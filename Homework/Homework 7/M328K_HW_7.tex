\documentclass[11pt]{article}
%\usepackage[spanish]{babel}
\usepackage[utf8]{inputenc}
\usepackage[OT1]{fontenc}
\usepackage{amsfonts, amsmath, amsthm, amssymb}
\usepackage{mathtools}
\usepackage{graphicx}
\usepackage{listings}
\usepackage[margin=1in]{geometry}
\usepackage{xcolor}
\usepackage{hyperref}
\hypersetup{
    colorlinks=true,
    linkcolor=blue,
    filecolor=magenta,      
    urlcolor=blue,
    pdftitle={Overleaf Example},
    pdfpagemode=FullScreen,
    }

\theoremstyle{definition}
\newtheorem{defn}{Definition}
\newtheorem{thm}{Theorem}
\newtheorem*{thm*}{Theorem}

\usepackage[Glenn]{fncychap}
% Sonny Lenny Glenn Conny Rejne Bjarne Bjornstrup

\newcommand{\ZZ}{\mathbb{Z}}
\newcommand{\NN}{\mathbb{N}}
\newcommand{\QQ}{\mathbb{Q}}
\newcommand{\ord}{\operatorname{ord}}


\newcommand{\Mod}[1]{\ (\mathrm{mod}\ #1)}
\newcommand{\legendre}[2]{\ensuremath{\left( \frac{#1}{#2} \right) }}

\DeclarePairedDelimiter\abs{\lvert}{\rvert}


\title{M328K: Homework 7}
\author{Katherine Ho}
\date\today
\begin{document}
\maketitle

\begin{enumerate}

    \item Determine whether the following quadratic congruences are solvable:
    \begin{enumerate}
        \item $x^2\equiv 219 \pmod{419}$
        \begin{proof}
            The quadratic congruence is solvable if 
            \[
                \legendre{219}{419} = 1
            \]
            \begin{align*}
                \legendre{219}{419}
                &= \legendre{3\cdot 73}{419}
                = \legendre{3}{419}\legendre{73}{419} \\\\
                \text{Since } &3\equiv 3\pmod{4} \text{ and } 419\equiv 3\pmod{4}, \\
                \legendre{3}{419}
                &= -\legendre{419}{3} 
                = -\legendre{2}{3} \\
                & \Longrightarrow 3\equiv 3\pmod{8}\rightarrow\legendre{2}{3}=-1\\
                &= -(-1) = 1 \\\\
                \text{Since } &73\not\equiv 3\pmod{4}, \\
                \legendre{73}{419}
                &= \legendre{419}{73} 
                = \legendre{54}{73}
                = \legendre{3}{73}\legendre{18}{73} \\
                &= \legendre{3}{73}\legendre{3}{73}\legendre{3}{73}\legendre{2}{73} \\
                &\Longrightarrow\legendre{3}{73} = \legendre{73}{3} = \legendre{1}{3} = 1 \\
                &\Longrightarrow\legendre{2}{73} = \legendre{73}{2} = \legendre{1}{2} = 1 \\
                &= 1 \\\\
                \legendre{219}{419} &= (1)(1) = 1
            \end{align*}
            The quadratic congruence \underline{is} solvable.
        \end{proof}

        \item $3x^2+6x + 5 \equiv 0 \pmod{89}$
        \begin{proof} First, we can use algebra to modify the left expression.
            \begin{align*}
                3x^2+6x+5 &\equiv 0 \pmod{89} \\ 
                3(x^2+2x+1) + 2 &\equiv 0 \pmod{89} \\ 
                3(x+1)^2 &\equiv -2 \pmod{89} \\ 
                90(x+1)^2 &\equiv -60 \pmod{89} \\
                (x+1)^2 &\equiv 29 \pmod{89}
            \end{align*}
            The congruence is solvable if $\legendre{29}{89} = 1$.
            \begin{align*}
                \legendre{29}{89} &= \legendre{89}{29} = \legendre{2}{29} \\
                &\Longrightarrow 29\equiv 5\pmod{8} \rightarrow \legendre{2}{29} = -1 \\
                &= -1
            \end{align*}
            The congruence \underline{is not} solvable. 
        \end{proof}

        \item $2x^2 + 5x - 9 \equiv 9 \pmod{101}$
        \begin{proof} First, we can use algebra to modify the left expression.
            \begin{align*}
                2x^2+5x-18 &\equiv 0 \pmod{101} \\
                16x^2 + 40x - 144 &\equiv 0\pmod{101} \\
                16x^2 + 40x + 25 - 25 - 144 &\equiv 0\pmod{101} \\
                (4x+5)^2 &\equiv 169\pmod{101} \\ 
                (4x+5)^2 &\equiv 68\pmod{101}
            \end{align*} 
            The congruence is solvable if $\legendre{68}{101} = 1$.
            \begin{align*}
                \legendre{68}{101} &= \legendre{4}{101}\legendre{17}{101} \\
                \legendre{4}{101} &= \legendre{2^2}{101} = \legendre{1}{101} = 1\\
                \legendre{17}{101} &= \legendre{101}{17} = \legendre{16}{17}
                = \legendre{2^4}{17} = \legendre{1}{17} = 1 \\
                \legendre{68}{101} &= (1)(1) = 1
            \end{align*}
            The congruence \underline{is} solvable.
        \end{proof}

    \end{enumerate}

    \item Let $g$ be a primitive root of an odd prime $p$. Show that the quadratic residues and quadratic nonresidues of $p$ are (modulo $p$) precisely the even powers and odd powers of $g$, respectively. In particular, $p$ has the same number $(p-1)/2$ of quadratic residues and quadratic nonresidues.
    \begin{proof}
        We say that $a$ is a quadratic residue of $p$ 
        if the congruence $x^2\equiv a\pmod{p}$ has a solution.
        Let $x = g^n$, ie. power of $g$. \\
        Then, the quadratic residues are of the form
        \[
            (g^n)^2 = g^{2n}
        \]
        We can see that the quadratic residues are the even powers of g.
        And so the non residues must be the odd powers of $g$. \\
    \end{proof}

    \item Let $p$ be an odd prime and let $M$ be the product of the quadratic residues of $p$ that belong to $\{1,2,\dots, p-1\}$. Show that $M$ is congruent modulo $p$ to 1 or $-1$ according to whether $p\equiv 3\Mod 4$ or $p\equiv 1\Mod 4$. (Hint: Use a primitive root of $p$.)
    \begin{proof}
        Every quadratic residue is of the form $k^2$ for 
        $k\in\{1,\dots,\frac{p-1}{2}\}$. The product of quadratic residues $M$ is 
        \[
            M = \prod_{k=1}^{\frac{p-1}{2}} k^2
        \]
        Since $k^2\equiv (-1)\cdot k\cdot (p-k)\pmod{p}$, we have
        \[
            \prod_{k=1}^{\frac{p-1}{2}} k^2 = (-1)^{\frac{p-1}{2}}\prod_{k=1}^{\frac{p-1}{2}} k \cdot \prod_{k=1}^{\frac{p-1}{2}} (p-k) = (-1)^{\frac{p-1}{2}}\cdot (p-1)! \\
            \equiv (-1)^{\frac{p+1}{2}}\pmod{p}
        \]
        Now consider the following.
        \begin{align*}
            (-1)^{\frac{p+1}{2}} = 
            \begin{cases}
                1   \quad\text{ if } \frac{p+1}{2} \text{ is even} \\
                -1 \quad\text{ if } \frac{p+1}{2} \text{ is odd} 
            \end{cases}
        \end{align*}
        $\frac{p+1}{2}$ is even for $p\equiv 3\pmod{4}$ and
        $\frac{p+1}{2}$ is odd for $p\equiv 1\pmod{4}$. So,
        \begin{align*}
            (-1)^{\frac{p+1}{2}} = 
            \begin{cases}
                1   \quad\text{ if } p\equiv 3\pmod{4} \\
                -1 \quad\text{ if } p\equiv 1\pmod{4} 
            \end{cases}
        \end{align*}
        Thus $M$ is congruent modulo $p$ to $1$ or $-1$
        according to whether $p\equiv 3\pmod{4}$ or $p\equiv 1\pmod{4}$. 
    
    \end{proof}

    \item Show that $(5/p)=1$ if and only if $p\equiv 1, 9, 11, \text{ or } 19 \pmod{20}$.
    \begin{proof}
        $\legendre{5}{p}=1$ if $5$ is a quadratic residue $\pmod{p}$. 
        So there must be an integer $x$ where $x^2\equiv 5\pmod{p}$. 

        By the law of quadratic reciprocity,
        \[
            \legendre{5}{p}\legendre{p}{5} = (-1)^{\frac{(5-1)(p-1)}{4}} = (-1)^{\frac{4(p-1)}{4}} = (-1)^{p-1}
        \]
        Since $p$ is an odd prime, then $p-1$ is even and $(-1)^{p-1} = 1$.
        So, we get 
        \[
            \legendre{5}{p} = \legendre{p}{5} = 1
        \]
        Since we know this, we can determine when $p$ is a quadratic residue modulo 5, ie.
        \[
            x^2\equiv p\pmod{5}
        \]
        The possible values are $x=0,1,2,3,4,5$. 
        Then the quadratic residues are $0,1,4$.
        So $p$ is a quadratic residue for $p\equiv 1$ or $p\equiv 4\pmod{5}$.
        We exclude $p\equiv 0\pmod{5}$ since $p$ is an odd prime.
        So, $\legendre{p}{5} = 1$ iff $p\equiv 1$ or $p\equiv 4\pmod{5}$. 
        \begin{itemize}
            \item If $p\equiv 1\pmod{5}$, the possible values modulo $20$ are
            $p\equiv 1,6,11,16\pmod{20}$. Excluding non-odd numbers, we get 
            $p\equiv 1,11\pmod{20}$.
            \item If $p\equiv 4\pmod{5}$, the possible values modulo $20$ are
            $p\equiv 4,9,14,19\pmod{20}$. Excluding non-odd numbers, we get 
            $p\equiv 9,19\pmod{20}$.
        \end{itemize}
        Thus $\legendre{5}{p}=1$ iff $p\equiv 1,9,11,$ or $19\pmod{20}$.

    \end{proof}

    \item Prove that $7$ is a primitive root of any prime of the form $p=2^{4n} + 1$ for $n\geq 1$.
    \begin{proof}
        Let $p=2^{4n}+1$ be a prime. So we have $p-1 = 2^{4n}$.
        Let $k$ be the order of $7\pmod{p}$. Then $7^k\equiv 1\pmod{p}$.
        Since $k$ is the order, it must divide $p-1$. 
        Therefore, $k$ must be of the form 
        $2^m$ for where $0\le m\le 4n$. \\
        Suppose $k<2^{4n}$. Then $k$ must divide $2^{4n}-1$. 
        Therefore, $7^{2^{4n-1}}\equiv 1\pmod{p}$.
        Then square both sides and we get $7^{2^{4n}}\equiv 1\pmod{p}$
        Since $p=2^{4n}+1$, we can use the fact that $a^{p-1}\equiv 1\pmod{p}$:
        \[
            7^{2^{4n}}\equiv 7^{p-1}\equiv 1\pmod{p}
        \]
        This implies that $7$ is a quadratic residue $\pmod{p}$.
        However by the quadratic reciprocity law, $7$ is not a 
        quadratic residue $\pmod{p}$. This is a contradiction, so 
        $k<2^{4n}$ is false.
        So, $k=2^{4n}=p-1$. 
        Thus $7$ is a primitive root of $p$ where $p$ is a prime of the form
        $p=2^{4n}+1$.
    \end{proof}
    
\end{enumerate}

\end{document}

