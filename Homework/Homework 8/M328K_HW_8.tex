\documentclass[11pt]{article}
%\usepackage[spanish]{babel}
\usepackage[utf8]{inputenc}
\usepackage[OT1]{fontenc}
\usepackage{amsfonts, amsmath, amsthm, amssymb}
\usepackage{mathtools}
\usepackage{graphicx}
\usepackage{listings}
\usepackage[margin=1in]{geometry}
\usepackage{xcolor}
\usepackage{hyperref}
\hypersetup{
    colorlinks=true,
    linkcolor=blue,
    filecolor=magenta,      
    urlcolor=blue,
    pdftitle={Overleaf Example},
    pdfpagemode=FullScreen,
    }

\theoremstyle{definition}
\newtheorem{defn}{Definition}
\newtheorem{thm}{Theorem}
\newtheorem*{thm*}{Theorem}

\usepackage[Glenn]{fncychap}
% Sonny Lenny Glenn Conny Rejne Bjarne Bjornstrup

\newcommand{\ZZ}{\mathbb{Z}}
\newcommand{\NN}{\mathbb{N}}
\newcommand{\QQ}{\mathbb{Q}}
\newcommand{\ord}{\operatorname{ord}}

\newcommand{\leg}[2]{\genfrac(){}{0}{#1}{#2}}
\newcommand{\legp}[1]{\leg{#1}{p}}

\newcommand{\legendre}[2]{\ensuremath{\left( \frac{#1}{#2} \right) }}

\newcommand{\Mod}[1]{\ (\mathrm{mod}\ #1)}


\DeclarePairedDelimiter\abs{\lvert}{\rvert}


\title{M328K: Homework 8}
\author{Katherine Ho}
\date\today
\begin{document}
\maketitle

\begin{enumerate}
    \item Let $n>1$ be odd and let $\gcd(a,n)=1$. Show that if $a$ is a quadratic residue of $n$, then the Jacobi symbol satisfies $\leg{a}{n} = 1.$ Give a counterexample to show that the converse is false.
    \begin{proof}
        If $a$ is a QR of $n$, then $x^2\equiv a\pmod{n}$ has a solution.
        Then, $x^2\equiv a\pmod{p_i}$ has a solution for every prime factor $p_i$ of $n$. 
        By definition of the Legendre symbol, 
        \[
            \legendre{a}{p_i} = 1
        \]
        for each prime factor $p_i$ of $n$.
        Then the product of the Legendre symbols for each $p_i$ is
        \[
            \legendre{a}{n} = \legendre{a}{p_1} \dots \legendre{a}{p_k} = 1
        \]
        Thus if $a$ is a quadratic residue of $n$, then the Jacobi symbol satisfies $\leg{a}{n} = 1.$ \\
    \end{proof}

        A counterexample is $a=2$ and $n=9$. 
        The Jacobi symbol satisfies $\legendre{2}{9} = 1$:
        \[
            \legendre{2}{9} = \legendre{2}{3}\cdot \legendre{2}{3} = (-1)(-1) = 1
        \]

        Now we check if 2 is a quadratic residue of 9 by calculating all of the QRs of 9. 
        \begin{align*}
            1^2 \equiv 1 \pmod{9} \\
            2^2 \equiv 4 \pmod{9} \\
            3^2 \equiv 0 \pmod{9} \\
            4^2 \equiv 7 \pmod{9} \\
            5^2 \equiv 7 \pmod{9} \\
            6^2 \equiv 0 \pmod{9} \\
            7^2 \equiv 4 \pmod{9} \\
            8^2 \equiv 1 \pmod{9}
        \end{align*}
        
        2 is not a quadratic residue of 9 and thus the converse is false.

    % \item Quadratic reciprocity tells us when solutions exist to $x^2\equiv a\Mod p$, but it does not tell us how to find them. However:
    % \begin{enumerate}
    %     \item Suppose that $p\equiv 3\Mod 4$ and $\legp{a}=1$. Show that $x=a^{(p+1)/4}$ is a solution to $x^2\equiv a\Mod{p}$.
    %     \item Suppose that $p\equiv 5\Mod 8$ and $\legp{a}=1$. Show that either $x = a^{(p+3)/8}$ OR $x=2a(4a)^{(p-5)/8}$ is a solution to $x^2\equiv a\Mod p$.
    % \end{enumerate}

    \item Prove that there are infinitely many primes of the form $8k+7$. (Hint: Emulate the proof that there are infinitely many primes of the form $4k+1$, using $N = (4p_1\cdots p_n)^2 - 2$.)
    \begin{proof}
        Suppose that there are only a finite number of primes of the form 
        $8k+7$. Let these primes be $p_1,\dots,p_n$ s.t. $p_i\equiv 7\pmod{8}$.
        Consider $N = (4p_1\dots p_n)^2 - 2$.
        Let $p$ be an odd prime dividing $N$. Note $p\ne p_i$ for any $i$, otherwise 
        $p\mid (N-(4p_1\dots p_n)^2) = -2$. 
        But since $p\mid ((4p_1\dots p_n)^2 - 2)$, we have
        \[
            (4p_1\dots p_n)^2 \equiv 2 \pmod{p}
        \]
        ie. $\legendre{2}{p} = 1$, so $p\equiv 1,7\pmod{8}$. 
        So we have constructed another prime $\equiv 7\pmod{8}$ not in the original list. 
        Thus there are infinitely many primes of the form $8k+7$.

    \end{proof}

    \item Let $n=341$.
    \begin{enumerate}
        \item Apply the Fermat primality test to $n$ using $a=2$. What is the conclusion?
        \begin{proof}
            Let $a=2$. 
            \[
                \text{If } 2^{340}\not\equiv 1 \pmod{341} \text{, then 341 is composite.}
            \]
            The binary expansion of $2^{340}$ is 
            \[
                2^{340} = 2^{256}\cdot 2^{64}\cdot 2^{16}\cdot 2^{4}
            \]

            Then find what each term is congruent to $\pmod{341}$.
            \begin{align*}
                2   &\equiv 2 \pmod{341} \\
                2^2 &\equiv 4 \\
                2^4 &\equiv 16 \\
                2^8 &\equiv 256 \\
                2^{16} &\equiv 256^2 \equiv 65536 \equiv 64  \\
                2^{32} &\equiv 64^2 \equiv 4096 \equiv 4 \\
                2^{64}  &\equiv 4^2\equiv 16 \\
                2^{128} &\equiv 16^2\equiv 256 \\
                2^{256}  &\equiv 256^2\equiv 64
            \end{align*}

            By substitution,
            \begin{align*}
                2^{340} &\equiv 64\cdot 16\cdot 64\cdot 16 \pmod{341}\\
                        &\equiv 4\cdot 256 \pmod{341}\\
                        &\equiv 1 \pmod{341}
            \end{align*}

            We have found that $2^{340}\equiv 1\pmod{341}$, so the test is indeterminate.

        \end{proof}

        \item Apply the Solovay--Strassen primality test to $n$ using $a=2$. What is the conclusion?
        \begin{proof}
            Let $a=2$.
            \[
                \text{If } 2^{\frac{341-1}{2}}\not\equiv \legendre{2}{341}\pmod{341} \text{, then 341 is composite.}
            \]
            Evaluate $\legendre{2}{341}$.
            Since  $341\equiv 5\pmod{8}, \legendre{2}{341} = -1$. \\

            Now by substitution, we can check 
            \begin{align*}
                2^{170} \equiv -1 \pmod{341}  
            \end{align*}

            The binary expansion of $2^{170}$ is
            \[
                2^{128}\cdot 2^{32}\cdot 2^8\cdot 2^2
            \]

            By substitution, 
            \begin{align*}
                2^{170} &= 2^{128}\cdot 2^{32}\cdot 2^8\cdot 2^2 \\
                        &\equiv 256\cdot 4\cdot 256\cdot 2\pmod{341} \\
                        &\equiv 64\cdot 4\cdot 2 \\
                        &\equiv 512 \pmod{341} \\
                        &\equiv 171 \pmod{341}
            \end{align*}
            
            $171 \not\equiv -1 \pmod{341}$, so we can conclude that 341 is composite. 

        \end{proof}

    \end{enumerate}

    
\end{enumerate}

\end{document}

