\documentclass[11pt]{article}
%\usepackage[spanish]{babel}
\usepackage[utf8]{inputenc}
\usepackage[OT1]{fontenc}
\usepackage{lmodern}
\usepackage{amsfonts, amsmath, amsthm, amssymb}
\usepackage{mathtools}
\usepackage{graphicx}
\usepackage{listings}
\usepackage[margin=1in]{geometry}
\usepackage{xcolor}
\usepackage{hyperref}
\hypersetup{
    colorlinks=true,
    linkcolor=blue,
    filecolor=magenta,      
    urlcolor=blue,
    pdftitle={Overleaf Example},
    pdfpagemode=FullScreen,
    }

\theoremstyle{definition}
\newtheorem{defn}{Definition}
\newtheorem{thm}{Theorem}
\newtheorem*{thm*}{Theorem}

\usepackage[Glenn]{fncychap}
% Sonny Lenny Glenn Conny Rejne Bjarne Bjornstrup

\newcommand{\ZZ}{\mathbb{Z}}
\newcommand{\NN}{\mathbb{N}}
\newcommand{\QQ}{\mathbb{Q}}
\newcommand{\ord}{\operatorname{ord}}

\newcommand{\leg}[2]{\genfrac(){}{0}{#1}{#2}}
\newcommand{\legp}[1]{\leg{#1}{p}}


\newcommand{\Mod}[1]{\ (\mathrm{mod}\ #1)}



\DeclarePairedDelimiter\abs{\lvert}{\rvert}


\title{M328K: Homework 10}
\author{Katherine Ho}
\date\today
\begin{document}
\maketitle

\begin{enumerate}
    \item In this problem we will investigate an important arithmetic function that is \emph{not} multiplicative. The \emph{Mangoldt function} $\Lambda$ is defined by
$$
\Lambda(n) = \begin{cases}
    \log p & \text{if $n=p^k$, where $p$ is prime and $k\geq 1$} \\
    0 & \text{otherwise.}
\end{cases}
$$
\begin{enumerate}
    \item Show that $\log n = \sum_{d\mid n} \Lambda(n)$. (Warning: In previous examples like this, it was sufficient to prove the equality for $n=p^k$ a prime power, but that is not enough here, since $\Lambda$ is not multiplicative.)
    \begin{proof}
        Let $n$ be a positive integer. The prime factorization of $n$ is 
        \[
            n = p_1^{e_1}\dots p_k^{e_k}
        \]
        Then taking the log of $n$ we get 
        \[
            \log(n) = \log(p_1^{e_1}\dots p_k^{e_k})
        \]
        By the properties of logarithms, we can rewrite this as 
        \[
            \log(n) = e_1\log(p_1) + \dots + e_k\log(p_k)
        \]
        By the definition of the Mangoldt function, we know that 
        \[
            \Lambda(n) = \log(p) \quad\text{if } n=p^k \text{, where p is prime and } k\ge 1
        \]
        In our equation, each term $e_k\log(p_k)$ divides $n$ and contains a prime $p_k$ raised to a positive exponent. 
        So, each term $e_k\log(p_k) = \Lambda(n)$. 
        
        The sum of all such terms is 
        \[
            e_1\log(p_1) + \dots + e_k\log(p_k) = \sum_{d\mid n}^{}\Lambda(n)
        \]
        By substitution, we get 
        \[
            \log(n) = \sum_{d\mid n}^{}\Lambda(n)
        \]

    \end{proof}
    \newpage

    \item Show that $\Lambda(n) = -\sum_{d\mid n} \mu(d)\log(d)$.
    \begin{proof}
        Using the Mobius inversion formula on $\log n = \sum_{d\mid n} \Lambda(n)$, we get 
        \[
            \Lambda(n) = \sum_{d\mid n}^{}\mu(d)\log(\frac{n}{d})
        \]
        By using algebra, we can rewrite this as 
        \begin{align*}
            \Lambda(n) &= \sum_{d\mid n}^{}\mu(d)(\log(n) - \log(d)) \\
            &= \sum_{d\mid n}^{}\mu(d)\log(n) - \sum_{d\mid n}^{}\mu(d)\log(d) \\
            &= \log(n)\sum_{d\mid n}^{}\mu(d) - \sum_{d\mid n}^{}\mu(d)\log(d)
        \end{align*}
        Consider the following property of the Mobius function:
        \[
            \sum_{d\mid n}^{} \mu(d) = 
            \begin{cases}
                1 \quad\text{if } n = 1 \\
                0 \quad\text{if } n > 1
            \end{cases}    
        \]
        If $n=1$, $\log(n)\sum_{d\mid n}^{}\mu(d) = 0$ since $\log(1) = 0.$
        
        So, $\log(n)\sum_{d\mid n}^{}\mu(d) = 0$ in all cases.

        This eliminates the term from the equation, leaving
        \[
            \Lambda(n) = -\sum_{d\mid n}^{}\mu(d)\log(d)
        \]
    \end{proof}

\end{enumerate}


    \item Consider the continued fraction $[2;5,1,3]$.
    \begin{enumerate}
        \item Calculate the convergents $C_0, C_1, C_2, C_3$.
        \begin{align*}
            C_0 &= 2 \\
            C_1 &= 2 + \frac{1}{5} = \frac{11}{5} \\
            C_2 &= 2 + \frac{1}{5 + \frac{1}{1}} = \frac{13}{6} \\
            C_3 &= 2 + \frac{1}{5 + \frac{1}{1 + \frac{1}{3}}} = 2 + \frac{1}{5 + \frac{3}{5}} = 2 + \frac{1}{5 + \frac{3}{5}} = 2 + \frac{5}{28} = \frac{61}{28}
        \end{align*}

        \item If $C_k = p_k/q_k$, calculate the continued fraction expansions of $p_k/p_{k-1}$ for $1\leq k\leq 3$.
        \begin{align*}
            p_0 &= a_0 = 2 \\
            p_1 &= a_1a_0 + 1 \\
            &= 10 + 1 = 11
        \end{align*}
        $\frac{p_1}{p_0} = \frac{11}{2}$
        \begin{align*}
            p_2 &= a_2p_1 + p_0 \\
            &= 1\cdot 11 + 2 = 13
        \end{align*}
        $\frac{p_2}{p_1} = \frac{13}{11}$
        \begin{align*}
            p_3 &= a_3p_2 + p_1 \\
            &= 3\cdot 13 + 11 \\
            &= 50
        \end{align*}
        $\frac{p_3}{p_2} = \frac{50}{13}$ \\
        
        \item Given a continued fraction $[a_0;a_1,\dots,a_n]$ with $a_0>0$, form a conjecture about the continued fraction expansion of $p_n/p_{n-1}$. Prove it.
        
        Conjecture: $\frac{p_n}{p_{n-1}}$ is the reversal of the continued fraction $[a_0;a_1,\dots,a_n]$. That is,
        \[
            \frac{p_n}{p_{n-1}} = [a_n; a_{n-1}, a_{n-2}, \dots, a_0]
        \]

        \begin{proof}
            We know that $p_n = a_np_{n-1}+p_{n-2}$, so we can write:
            \begin{align*}
                \frac{p_n}{p_{n-1}} &= \frac{a_np_{n-1} + p_{n-2}}{p_{n-1}} \\
                &= a_n + \frac{p_{n-2}}{p_{n-1}} \\
                &= a_n + \frac{1}{\frac{1}{\frac{p_{n-2}}{p_{n-1}}}} 
            \end{align*}
            Similarly, we know that $p_{n-1} = a_{n-1}p_{n-2}  + p_{n-3}$, so 
            \[
                \frac{p_{n-1}}{p_{n-2}} = a_{n-1}  + \frac{p_{n-3}}{p_{n-2}}
            \] 
            and 
            \[
                \frac{p_{n-2}}{p_{n-1}} = \frac{1}{a_{n-1}  + \frac{p_{n-3}}{p_{n-2}}}
            \]
            So, 
            \begin{align*}
                \frac{p_n}{p_{n-1}} &= a_n + \frac{1}{a_{n-1} + \frac{1}{\frac{p_{n-3}}{p_{n-2}}}} 
            \end{align*}
            We can see that if we continue to transform 
            $\frac{1}{\frac{p_{n-3}}{p_{n-2}}}$ and so on, we will see 
            the following pattern:
            \[
                \frac{p_n}{p_{n-1}} = a_n + \frac{1}{a_{n-1} + \frac{1}{a_{n-2} + \dots + \frac{1}{a_0}}} 
            \]
            This continued fraction is $[a_n; a_{n-1}, a_{n-2}, \dots, a_0]$, 
            thus proving the conjecture.

        \end{proof}
    
    \end{enumerate}
    \item Compute continued fraction expansions of the following:
    \begin{enumerate}
        \item $\sqrt 5$
        \begin{proof}
            Since $\sqrt{4} = 2 < \sqrt{5} < \sqrt{9} = 3$, 
            then the integer component of $\sqrt{5}$ is 2. We can rewrite 
            $\sqrt{5}$ as
            \begin{align*}
                \sqrt{5} &= 2 + (\sqrt{5} - 2) \\
                &= 2 + \frac{1}{\frac{1}{\sqrt{5} - 2}} 
            \end{align*}
            The fraction $\frac{1}{\sqrt{5} - 2} = \frac{\sqrt{5} + 2}{(\sqrt{5} - 2)(\sqrt{5} + 2)} = \frac{\sqrt{5} + 2}{1} = 4 + \sqrt{5} - 2 = 4 + \frac{1}{\frac{1}{\sqrt{5} - 2}}$.

            So by substitution, 
            \[
                \sqrt{5} = 2 + \frac{1}{4 + \frac{1}{\frac{1}{\sqrt{5} - 2}}}
            \]
            Since we know the value of $\frac{1}{\sqrt{5} - 2}$, then we also know that 
            this expression will continue to generate $4 + \frac{1}{\frac{1}{\sqrt{5} - 2}}$.

            So, the continued fraction expression is 
            \[
                \sqrt{5} = 2 + \frac{1}{4 + \frac{1}{4 + \frac{1}{ 4 + \frac{1}{\dots}}}} = [2; \overline{4}]
            \]
        \end{proof}

        \item $\displaystyle\frac{5+\sqrt{37}}{2}$
        \begin{proof}
            First we know that $\sqrt{36} = 6 < \sqrt{37} < \sqrt{49} = 7$, so 
            \begin{align*}
                \frac{5+6}{2} &< \frac{5+\sqrt{37}}{2} < \frac{5+7}{2} \\
                5.5 &< \frac{5+\sqrt{37}}{2} < 6
            \end{align*}
            So, the integer component is 5.
            \begin{align*}
                \frac{5+\sqrt{37}}{2} &= 5 + \frac{5+\sqrt{37}}{2} - 5 \\
                &= 5 + \frac{1}{\frac{1}{\frac{5+\sqrt{37}}{2} - 5}}
            \end{align*}
            The expression $\frac{5+\sqrt{37}}{2} - 5 = \frac{5+\sqrt{37}}{2} - \frac{10}{2} = \frac{\sqrt{37} - 5}{2}$

            The inverse is $\frac{1}{\frac{\sqrt{37} - 5}{2}} = \frac{2}{\sqrt{37} - 5} = \frac{2(\sqrt{37} + 5)}{12} = \frac{\sqrt{37} + 5}{6}$

            \begin{align*}
                \frac{6 + 5}{6} &< \frac{\sqrt{37} + 5}{6} < \frac{7 + 5}{6} \\
                1\frac{5}{6} &< \frac{\sqrt{37} + 5}{6} < 2
            \end{align*}
            So the integer component of this fraction is 1. Then, 
            \begin{align*}
                \frac{\sqrt{37} + 5}{6} &=  1 + \frac{\sqrt{37} + 5}{6} - 1 \\
                &= 1 + \frac{\sqrt{37}-1}{6}
            \end{align*}
            and 
            \[
                \frac{5+\sqrt{37}}{2} = 5 + \frac{1}{1 + \frac{\sqrt{37} - 1}{6}} = 5 + \frac{1}{1 + \frac{1}{\frac{1}{\frac{\sqrt{37} - 1}{6}}}}
            \]

            The expression $\frac{1}{\frac{\sqrt{37} - 1}{6}} = \frac{6}{\sqrt{37} - 1} = \frac{6(\sqrt{37} + 1)}{36} = \frac{\sqrt{37} + 1}{6}$

            \begin{align*}
                \frac{6 + 1}{6} &< \frac{\sqrt{37} + 1}{6} < \frac{7 + 1}{6} \\
                1\frac{1}{6} &< \frac{\sqrt{37} + 1}{6} < 1\frac{1}{3}
            \end{align*}
            The integer component is 1. So,
            \[
                \frac{\sqrt{37} + 1}{6} = 1 + \frac{\sqrt{37} + 1}{6} - 1 = 1 + \frac{\sqrt{37} - 5}{6}
            \]
            \[
                \frac{5+\sqrt{37}}{2} = 5 + \frac{1}{1 + \frac{1}{1 + \frac{\sqrt{37} - 5}{6}}} = 5 + \frac{1}{1 + \frac{1}{1 + \frac{1}{\frac{1}{\frac{\sqrt{37} - 5}{6}}}}}
            \]

            The expression $\frac{1}{\frac{\sqrt{37} - 5}{6}} = \frac{6}{\sqrt{37} + 5} = \frac{6(\sqrt{37} + 5)}{12} = \frac{\sqrt{37}+5}{2}$
            
            From an earlier step, we already know  
            \begin{align*}
                \frac{5+\sqrt{37}}{2} = 5 + \frac{1}{\frac{1}{\frac{5+\sqrt{37}}{2} - 5}}
            \end{align*}
            so by substitution, 
            \begin{align*}
                \frac{5+\sqrt{37}}{2} &= 5 + \frac{1}{1 + \frac{1}{1 + \frac{1}{\frac{1}{5 + \frac{1}{\frac{1}{\frac{5+\sqrt{37}}{2} - 5}}}}}}
            \end{align*}
            Now we can see that the pattern will continue. 

            Thus, the continued fraction expression is 
            \[
                \frac{5+\sqrt{37}}{2} = 5 + \frac{1}{1 + \frac{1}{1 + \frac{1}{\frac{1}{5 + \frac{1}{1 + \dots}}}}} = [5; \overline{1,1,5}]
            \]

        \end{proof}

        \item $\sqrt{n^2+1}$ for any $n>0$.
        \begin{proof}
            First, the integer component is $n$ and 
            \begin{align*}
                \sqrt{n^2 + 1} &= n + \sqrt{n^2 + 1} - n \\
                &= n + \frac{1}{\frac{1}{\sqrt{n^2 + 1} - n}}
            \end{align*}
            Then, $\frac{1}{\sqrt{n^2 + 1} - n} = \frac{\sqrt{n^2 + 1} + n}{n^2 + 1 - n^2} = \sqrt{n^2 + 1} + n$.

            The integer component is $2n$, so 
            \begin{align*}
                \sqrt{n^2 + 1} + n &= 2n + \sqrt{n^2 + 1} + n - 2n \\
                &= 2n + \sqrt{n^2 + 1} - n \\
                &= 2n + \frac{1}{\frac{1}{\sqrt{n^2 + 1} - n}}
            \end{align*}

            By substitution, 
            \[
                \sqrt{n^2 + 1} = n + \frac{1}{2n + \frac{1}{\frac{1}{\sqrt{n^2 + 1} - n}}}
            \]

            Since we already know the value of $\frac{1}{\sqrt{n^2 + 1} - n}$, 
            we know that the fraction will infinitely continue in this pattern.
            So, the continued fraction expression is 
            \[
                \sqrt{n^2 + 1} = n + \frac{1}{2n + \frac{1}{2n + \frac{1}{2n + \dots}}} = [n; \overline{2n}]
            \]

        \end{proof}

    \end{enumerate}
    \item Using the continued fraction of $\sqrt 5$ from the previous problem, find the first convergent that gives a rational approximation of $\sqrt 5$ accurate to four decimal places.
    \begin{proof}
        The actual value of $\sqrt{5} = 2.2360\dots$.
        \begin{align*}
            C_0 &= 2 \\
            C_1 &= 2 + \frac{1}{4} = 2.25 \\
            C_2 &= 2 + \frac{1}{4 + \frac{1}{4}} =  2 + \frac{1}{\frac{17}{4}} = 2 + \frac{4}{17} = \frac{38}{17} = 2.2352\dots \\
            C_3 &= 2 + \frac{1}{4 + \frac{1}{4 + \frac{1}{4}}} = 2 + \frac{1}{4 + \frac{4}{17}} = 2 + \frac{17}{72} = \frac{161}{72} = 2.2361\dots \\
            C_4 &= 2 + \frac{1}{4 + \frac{1}{4 + \frac{1}{4 + \frac{1}{4}}}} = 2 + \frac{1}{4 + \frac{17}{72}} = 2 + \frac{72}{305} = \frac{682}{305} = 2.2360\dots
        \end{align*}
        Thus the first convergent that gives a rational approximation of $\sqrt{5}$ is 
        \[
            C_4 = \frac{682}{305}
        \]
    \end{proof}

\end{enumerate}

\end{document}

