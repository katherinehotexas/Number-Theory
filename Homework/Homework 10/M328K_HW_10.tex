\documentclass[11pt]{article}
%\usepackage[spanish]{babel}
\usepackage[utf8]{inputenc}
\usepackage[OT1]{fontenc}
\usepackage{lmodern}
\usepackage{amsfonts, amsmath, amsthm, amssymb}
\usepackage{mathtools}
\usepackage{graphicx}
\usepackage{listings}
\usepackage[margin=1in]{geometry}
\usepackage{xcolor}
\usepackage{hyperref}
\hypersetup{
    colorlinks=true,
    linkcolor=blue,
    filecolor=magenta,      
    urlcolor=blue,
    pdftitle={Overleaf Example},
    pdfpagemode=FullScreen,
    }

\theoremstyle{definition}
\newtheorem{defn}{Definition}
\newtheorem{thm}{Theorem}
\newtheorem*{thm*}{Theorem}

\usepackage[Glenn]{fncychap}
% Sonny Lenny Glenn Conny Rejne Bjarne Bjornstrup

\newcommand{\ZZ}{\mathbb{Z}}
\newcommand{\NN}{\mathbb{N}}
\newcommand{\QQ}{\mathbb{Q}}
\newcommand{\ord}{\operatorname{ord}}

\newcommand{\leg}[2]{\genfrac(){}{0}{#1}{#2}}
\newcommand{\legp}[1]{\leg{#1}{p}}


\newcommand{\Mod}[1]{\ (\mathrm{mod}\ #1)}



\DeclarePairedDelimiter\abs{\lvert}{\rvert}


\title{M328K: Homework 10}
\author{Katherine Ho}
\date\today
\begin{document}
\maketitle

\begin{enumerate}
    \item In this problem we will investigate an important arithmetic function that is \emph{not} multiplicative. The \emph{Mangoldt function} $\Lambda$ is defined by
$$
\Lambda(n) = \begin{cases}
    \log p & \text{if $n=p^k$, where $p$ is prime and $k\geq 1$} \\
    0 & \text{otherwise.}
\end{cases}
$$
\begin{enumerate}
    \item Show that $\log n = \sum_{d\mid n} \Lambda(n)$. (Warning: In previous examples like this, it was sufficient to prove the equality for $n=p^k$ a prime power, but that is not enough here, since $\Lambda$ is not multiplicative.)
    \item Show that $\Lambda(n) = -\sum_{d\mid n} \mu(d)\log(d)$.
\end{enumerate}


    \item Consider the continued fraction $[2;5,1,3]$.
    \begin{enumerate}
        \item Calculate the convergents $C_0, C_1, C_2, C_3$.
        \item If $C_k = p_k/q_k$, calculate the continued fraction expansions of $p_k/p_{k-1}$ for $1\leq k\leq 3$.
        \item Given a continued fraction $[a_0;a_1,\dots,a_n]$ with $a_0>0$, form a conjecture about the continued fraction expansion of $p_n/p_{n-1}$. Prove it.
    \end{enumerate}
    \item Compute continued fraction expansions of the following:
    \begin{enumerate}
        \item $\sqrt 5$
        \item $\displaystyle\frac{5+\sqrt{37}}{2}$
        \item $\sqrt{n^2+1}$ for any $n>0$.
    \end{enumerate}
    \item Using the continued fraction of $\sqrt 5$ from the previous problem, find the first convergent that gives a rational approximation of $\sqrt 5$ accurate to four decimal places.
\end{enumerate}

\end{document}

