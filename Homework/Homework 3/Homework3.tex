\documentclass[11pt]{article}
%\usepackage[spanish]{babel}
\usepackage[utf8]{inputenc}
\usepackage[OT1]{fontenc}
\usepackage{amsfonts, amsmath, amsthm, amssymb}
\usepackage{mathtools}
\usepackage{graphicx}
\usepackage{listings}
\usepackage[margin=1in]{geometry}
\usepackage{xcolor}

\theoremstyle{definition}
\newtheorem*{defn*}{Definition}
\newtheorem{thm}{Theorem}
\newtheorem*{thm*}{Theorem}

\usepackage[Glenn]{fncychap}
% Sonny Lenny Glenn Conny Rejne Bjarne Bjornstrup

\newcommand{\ZZ}{\mathbb{Z}}
\newcommand{\NN}{\mathbb{N}}
\newcommand{\QQ}{\mathbb{Q}}

\DeclarePairedDelimiter\abs{\lvert}{\rvert}


\title{M328K: Homework 3}
\author{Katherine Ho}
\date\today
\begin{document}
\maketitle

\begin{defn*}
    A \emph{complete residue system modulo $n$} is a set of integers such that every integer is congruent modulo $n$ to exactly one integer in the set. For example, the ``canonical'' complete residue system modulo $n$ is the set of integers $\{0,1,2,\dots,n-1\}$.
\end{defn*}

\begin{enumerate}
    \item 
    \begin{enumerate}
        \item Prove that any set of $n$ incongruent integers modulo $n$ forms a complete residue system modulo $n$.
        \begin{proof}
            Suppose a set of $n$ integers does not form a complete residue system mod $n$. 
            Then it contains at least one integer $a$ that is not congreuent to another integer in the set.
            This means when $a$ is divided by $n$, then none of the other elements are equal to its remainder.
            There are at most $n-1$ remainders in the set. By the pigeonhole principle, at least 2 
            integers in the set have the same remainder $\pmod{n}$. However this contradicts the supposition 
            where $a$ is incongruent with all of the other integers in the set. Hence any set of $n$ incongruent 
            integers modulo $n$ forms a complete residue system modulo $n$. \\
        \end{proof}
        
        \item Suppose $\gcd(a,n)=1$. Prove that the integers $$c,c+a,c+2a,\dots,c+(n-1)a$$ form a complete residue system modulo $m$ for any $c$.\footnote{Note: With $c=0$, this is the fundamental fact we used in class to prove Fermat's Little Theorem.}
    \end{enumerate}
    \item Find a complete (up to congruence) set of solutions to the linear congruence $34x\equiv 60 \pmod{98}$.
    \begin{proof}
        We have that $\gcd(34,98)=2$. Also, $2\mid 60$, so there are 2 solutions $\pmod{98}$.
        First we can find a solution to
        \begin{align*}
            17x &\equiv 30\pmod{49} \\
            17x - 49y &= 30
        \end{align*}
        Then, by the Euclidean Algorithm:
        \begin{align*}
            49 &= 17(2) + 15 \\
            49 - 17(2) &= 15 \\
            49(2) - 17(4) &= 30 \\
            17(-4) - 49(-2) &= 30
        \end{align*}
        \begin{center}
            $x=-4$, $y=-2$
        \end{center}
        So, $x=-4+49=45$ is a solution. The other solution $\pmod{98}$ is 
        \[ x=45+49=94 \]
        $x=45,94$ is a complete set of solutions up to congruence $\pmod{98}$.
    \end{proof}

    \item This exercise illustrates a neat inductive proof of Fermat's Little Theorem using the binomial theorem.
    \begin{enumerate}
        \item Let $p$ be prime. Show that $p$ divides ${p\choose k}$ for $1\leq k\leq p-1$, where $${p\choose k} = \frac{p!}{k!(p-k)!} = \frac{p(p-1)\cdots(p-k+1)}{1\cdot2\cdot 3\cdots k}.$$ Hint: First show that $p$ divides $\displaystyle k! {p\choose k}$.
        \begin{proof}
            Let $n={p\choose k}$:
            \begin{align*}
                n &= {p\choose k} \\
                n &= \frac{p!}{k!(p-k)!} \\
                n\cdot k!(p-k)! &= p! 
            \end{align*}
            $p$ divides $p!$, so the left expression is also divisible by $p$. \\
            This means that at least one factor of the expression is divisible by $p$.
            \begin{enumerate}
                \item $k!$ is not divisible by $p$ since it is less than $p$ and $p$ is prime.
                \item $(p-k)!$ is not divisible by $p$ since $p-k$ is less than $p$ and $p$ is prime.
            \end{enumerate}
            This leaves $n$, which must be divisible by $p$. Therefore, $p$ divides ${p\choose k}$.
        \end{proof}

        \item Use induction on $a$ together with the binomial theorem\footnote{Binomial theorem: $\displaystyle(a+1)^p = a^p + {p\choose 1}a^{p-1}+\cdots + {p\choose k}a^{p-k} + \cdots + {p\choose p-1}a + 1$} to give another proof of Fermat's Little Theorem.
        \begin{proof}
            We aim to prove $a^{p-1}\equiv 1\pmod{p}$ for a prime $p$ and $p\nmid a$. 
            It can be rewritten as $a^p\equiv a\pmod{p}$. \\
            Base case ($a=1$): $1^{p}\equiv 1\pmod{p}$. 
            \[ 1^{p}-1 = px \quad\text{ for some } x\in\ZZ \]
            This is true for any prime $p$ and $x=0$. \\
            \underline{Inductive Hypothesis}: Assume $a^p\equiv a\pmod{p}$ for an integer 
            $a\in\ZZ$ is true. \\
            Consider $a+1$:
            \[
                (a+1)^{p} = a^p + {p\choose 1}a^{p-1}+\cdots + {p\choose k}a^{p-k} + \cdots + {p\choose p-1}a + 1 \\
            \]
            By (a), each binomial coefficient ${p\choose k}$ is divisible by $p$ since $p$ is prime. 
            So, if we take $\pmod{p}$ of this sum, we are left with:
            \begin{align*}
                (a+1)^{p} \equiv a^p + 1 \pmod{p}
            \end{align*}
            $a^p\equiv a\pmod{p}$ is true for $a+1$, thus proving Fermat's Little Theorem.
        \end{proof}
        
    \end{enumerate}

    \item A composite integer $n>1$ is called a \emph{Fermat pseudoprime to base $a$} if $a^{n-1}\equiv 1\pmod n$.
        \begin{enumerate}
            \item Prove the following: If $d,n\in \NN$ with $d\mid n$, then $2^d -1\mid 2^n-1$.
            
            Hint: Use the identity $$x^k-1 = (x-1)(x^{k-1}+x^{k-2}+\cdots+x+1).$$
            \begin{proof}
                Let $n=db$ for some $b\in\ZZ$.
                \begin{align*}
                    2^n-1 &= 2^{db}-1 \\
                    2^n-1 &= (2^d)^b-1 \\
                    2^n-1 &= (2^d-1)((2^d)^{b-1}+(2^d)^{b-2}+\dots+(2^d)^1+(2^d)^0)
                \end{align*}
                Thus if $d,n\in \NN$ with $d\mid n$, then $2^d -1\mid 2^n-1$.
            \end{proof}

            \item Prove that if $n$ is a Fermat pseudoprime to base $2$, then $M_n = 2^n - 1$ is also a Fermat pseudoprime to base $2$.
            \begin{proof}
                If $n\mid 2^{n-1}-1$, then $2^{n-1}-1 = nx$ for some $x\in\ZZ$. 
            \end{proof}

            \item Conclude that there are infinitely many Fermat pseudoprimes to base $2$.
            \begin{proof}
                
            \end{proof}
        \end{enumerate}
    
    \item A \emph{Carmichael} number is an integer $n>1$ that is a Fermat pseudoprime to base $a$ for all $a$ with $\gcd(a,n) = 1$.
    \begin{enumerate}
        \item Prove that if $n=p_1p_2\cdots p_r$ is a composite square-free integer such that $p_i-1\mid n-1$ for $i=1,2,\dots,r$, then $n$ is a Carmichael number.
        \begin{proof}
            
        \end{proof}

        \item Show that $6601$ is a Carmichael number.
        \begin{proof}
            
        \end{proof}
    \end{enumerate}

    \item Prove the converse to Wilson's Theorem: If $(m-1)! \equiv -1 \pmod m$, then $m$ is prime.
    \begin{proof}
        Let $m$ be composite. That is, $m=ab$ for some $1<a<b<m$. \\
        % ($a\mid m$ and $b\mid m$).  \\
        We can then say $a\mid (m-1)!$ since $1<a<m$. 
        If Wilson's Theorem holds for $m$, then
        \begin{align*}
            (m-1)! &\equiv -1\pmod{m} \\
            (m-1)! &= mx-1 \quad\text{for some x }\in\ZZ
        \end{align*}
        So, $m \mid (m-1)!+1$. Since $a|m$, then $a \mid (m-1)!+1$. 
        We can conclude that $a\mid 1$ since we also have $a\mid (m-1)!$. 
        If this is true, then $a=1$.
        However, this is a contradiction to $1<a<m$. Thus $m$ cannot be composite. 
        Therefore if $(m-1)! \equiv -1 \pmod m$, then $m$ is prime.
    \end{proof}

\end{enumerate}


\end{document}