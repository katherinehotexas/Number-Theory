\documentclass[11pt]{article}
%\usepackage[spanish]{babel}
\usepackage[utf8]{inputenc}
\usepackage[OT1]{fontenc}
\usepackage{amsfonts, amsmath, amsthm, amssymb}
\usepackage{mathtools}
\usepackage{graphicx}
\usepackage{listings}
\usepackage[margin=1in]{geometry}
\usepackage{xcolor}

\theoremstyle{definition}
\newtheorem{defn}{Definition}
\newtheorem{thm}{Theorem}
\newtheorem*{thm*}{Theorem}

\usepackage[Glenn]{fncychap}
% Sonny Lenny Glenn Conny Rejne Bjarne Bjornstrup

\newcommand{\ZZ}{\mathbb{Z}}
\newcommand{\NN}{\mathbb{N}}
\newcommand{\QQ}{\mathbb{Q}}

\DeclarePairedDelimiter\abs{\lvert}{\rvert}


\title{M328K: Homework 4}
\author{Katherine Ho}
\date\today
\begin{document}
\maketitle

\begin{enumerate}
    \item Show that if $d\mid n$, then $\phi(d)\mid \phi(n)$.
    \begin{proof}
        $d$ and $n$ can each be written as a product of primes. Given $d\mid n$:
        \begin{align*}
            d &= (p_1^{a_1})\dots(p_i^{a_i}) \\
            n &= (p_1^{b_1})\dots(p_i^{b_i})\cdot (q_1^{c_1})\dots(q_j^{c_j})
        \end{align*}
        where $b_i\ge a_i$. \\
        Then, since $\phi$ is multiplicative,
        \begin{align*}
            \phi(d) &= \phi(p_1^{a_1})\dots\phi(p_i^{a_i}) \\
            \phi(n) &= \phi(p_1^{b_1})\dots\phi(p_i^{b_i})\cdot \phi(q_1^{c_1})\dots\phi(q_j^{c_j})
        \end{align*}
        Since each factor is prime,
        \begin{align*}
            \phi(d) &= (p_1^{a_1} - p_1^{a_i-1})\dots(p_i^{a_i} - p_i^{a_i-1}) \\
            &= (p_1^{a_1}(1-\frac{1}{p_1}))\dots(p_i^{a_i}(1-\frac{1}{p_i})) \\
            \phi(n) &= (p_1^{b_1} - p_1^{b_i-1})\dots(p_i^{b_i} - p_i^{b_i-1})\cdot(q_1^{c_1} - q_j^{c_j-1})\dots(q^{c_j} - q^{c_j-1}) \\
            &= (p_1^{b_1}(1-\frac{1}{p_1}))\dots(p_i^{b_i}(1-\frac{1}{p_i}))\cdot(q_1^{c_1}(1-\frac{1}{q_1}))\dots(q_j^{c_j}(1-\frac{1}{q_j}))
        \end{align*}
        Since $b_i\ge a_i$, every factor of $\phi(d)$ divides a factor of $\phi(n)$.
        Thus $\phi(d)\mid \phi(n)$. \\
    \end{proof}

    \item Find the smallest positive integer $x$ satisfying the system of linear congruences
        \begin{alignat*}{3}
            3x &\equiv 10 &&\pmod{19} \\
            4x &\equiv 1 &&\pmod{23}.
        \end{alignat*}
    \begin{proof}
        First we can isolate x on the left side of each congruence by multiplying by the multiplicative inverse.
        \begin{enumerate}
            \item $3^{-1}\pmod{19}$:
            \begin{align*}
                3x &\equiv 1\pmod{19} \\
                3x-19y &= 1 \\
                19 &= 3(6)+1 \\
                1 &= 19-3(6) \\
                1 &= 3(-6) - 19(-1)
            \end{align*}
            We have $x=-6\equiv 13\equiv 3^{-1}\pmod{19}$. We can multiply by $3x\equiv 10\pmod{19}$ to get 
            \[ x\equiv 130\pmod{19} \]

            \item $4^{-1}\pmod{23}$
            \begin{align*}
                4x &\equiv 1\pmod{23} \\
                4x-23y &= 1 \\
                23 &= 4(5) + 3 \\
                4 &= 3(1) + 1 \\
                1 &= 4-3(1) \\
                1 &= 4-(23-4(5)) \\
                1 &= 4(6) - 23(1)
            \end{align*}
            We have $x=6\equiv 4^{-1}\pmod{23}$. We can multiply by $4x\equiv 1\pmod{23}$ to get
            \[ x\equiv 6\pmod{23} \]
        \end{enumerate}
        Now the system is 
        \begin{alignat*}{3}
            x &\equiv 130 &&\pmod{19} \\
            x &\equiv 6 &&\pmod{23}
        \end{alignat*}
        Since $\gcd(19,23)=1$, there is a unique solution for $x\pmod{19\cdot 23=437}$
        by the Chinese Remainder theorem. \\
        By Bezout's Theorem, there exist integers $y_1,y_2$ such that $19y_1+23y_2=1$. \\
        By the Euclidean Algorithm,
        \begin{align*}
            23 &= 19(1) + 4 \\
            19 &= 4(4) + 3 \\
            4 &= 3(1) + 1 \\
            1 &= 4-3(1) \\
            1 &= (23-19) - (19-4(4)) \\
            1 &= 23-2(19) + 4(4) \\
            1 &= 23-2(19) + 4(23-19) \\
            1 &= 23(5) - 6(19) \\
            1 &= 19(-6) + 23(5)
        \end{align*}
        So $y_1=-6,y_2=5$. Then, $x=a_2n_1y_1+a_1n_2y_2$ is a solution where $n_1=19,n_2=23$.
        \begin{align*}
            x &= (6)(19)(-6) + 130(23)(5) \\
            &= -684 + 14950 \\
            &= 14266 \\
            x &\equiv 282\pmod{437}
        \end{align*}
        Thus the smallest value of $x$ satisfying the system of congruences is $282$.

    \end{proof}
    \item Calculate $3^{434} \mod{1022}$ using binary exponentiation.
    \begin{proof}
        The binary expansion of $3^{434}$ is 
        \[
            3^{256} \cdot 3^{128} \cdot 3^{32} \cdot 3^{16} \cdot 3^2
        \]
        Then, we can find what each term is congruent to $\pmod{1022}$. 
        \begin{align*}
            3 &\equiv 3 \\
            3^2 &\equiv 9 \\
            3^4 &\equiv 81 \\
            3^8 &\equiv 81^2 \equiv 6,561 \equiv 429 \\
            3^{16} &\equiv 429^2 \equiv 184,041 \equiv 81 \\
            3^{32} &\equiv 81^2 \equiv 6,561 \equiv 429 \\
            3^{64} &\equiv 429^2 \equiv 184,041 \equiv 81 \\
            3^{128} &\equiv 81^2 \equiv 6,561 \equiv 429 \\
            3^{256} &\equiv 429^2 \equiv 184,041 \equiv 81
        \end{align*}
        Then substitute back into the binary expansion of $3^{434}$.
        \begin{align*}
            3^{434} &\equiv 81\cdot 429\cdot 429\cdot 81\cdot 9 \pmod{1022} \\
            3^{434} &\equiv 10,867,437,009 \pmod{1022}
        \end{align*}
        \begin{center}
            \boxed{3^{434}\equiv 9 \pmod{1022}}
        \end{center}
    \end{proof}

    \item Using RSA, decipher the critically important message someone is sending if $n =7417\cdot 8363$, exponent $E = 100019$, and you receive $23451141$. To fully decrypt, pair digits (padding at the beginning with 0 if necessary) and interpret as 01 = A, 26 = Z. Write down the intermediate steps you used. (Note: You may use computers/calculators to perform the divison algorithm and to reduce integers mod $n$. You may find the Wolfram Alpha command \texttt{Mod[a,n]} particularly useful.
    \begin{proof}
        First, we have the public key $(n,E) = (7417\cdot 8362,100019) = (62028371,100019)$. \\
        Then, we compute $\phi(n) = (7417-1)\cdot (8363-1) = 62012592$. \\
        We use this result to find the decryption exponent $D$.
        \[
            D = E^{-1}\pmod{\phi(n)} = 100019^{-1}\pmod{62012592} = 12142859
        \]
        Now we have the private key $(n,D) = (62028371,12142859)$. \\
        To recover the message $W$ from $Z=23451141$, we can use the private key to compute:
        \begin{align*}
            Z^D &\equiv W\pmod{n} \\
            23451141^{12142859} &\equiv W\pmod{62028371} \\
            &\equiv 12130115 
        \end{align*}
        To decrypt the message:
        \begin{align*}
            12\mid 13\mid 01\mid 15 \\
            L\mid M\mid A\mid O
        \end{align*} 
        The message is LMAO.
    \end{proof}

\end{enumerate}

\end{document}