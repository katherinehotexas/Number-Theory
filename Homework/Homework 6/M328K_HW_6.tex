\documentclass[11pt]{article}
%\usepackage[spanish]{babel}
\usepackage[utf8]{inputenc}
\usepackage[OT1]{fontenc}
\usepackage{amsfonts, amsmath, amsthm, amssymb}
\usepackage{mathtools}
\usepackage{graphicx}
\usepackage{listings}
\usepackage[margin=1in]{geometry}
\usepackage{xcolor}

\theoremstyle{definition}
\newtheorem{defn}{Definition}
\newtheorem{thm}{Theorem}
\newtheorem*{thm*}{Theorem}

\usepackage[Glenn]{fncychap}
% Sonny Lenny Glenn Conny Rejne Bjarne Bjornstrup

\newcommand{\ZZ}{\mathbb{Z}}
\newcommand{\NN}{\mathbb{N}}
\newcommand{\QQ}{\mathbb{Q}}

\newcommand{\ord}{\operatorname{ord}}
\newcommand{\ind}{\operatorname{ind}}

\DeclarePairedDelimiter\abs{\lvert}{\rvert}


\title{M328K: Homework 6}
\author{Katherine Ho}
\date\today
\begin{document}
\maketitle

\begin{enumerate}
    \item 
    \begin{enumerate}
        \item For $n>1$, let $a_1, a_2,\dots, a_{\phi(n)}$ be the positive integers less than $n$ that are relatively prime to $n$. Show that $$a_1 + a_2+\cdots + a_{\phi(n)} = \frac{1}{2} n\phi(n).$$ (Hint: $\gcd(a,n)=1 \iff \gcd(n-a,n)=1$)
        \begin{proof}
            For each integer $a_i$ coprime to $n$, $n-a_i$ is also coprime to $n$ because
            $\gcd(a,n)=1 \iff \gcd(n-a,n)=1$. We can pair these integers as so:
            $(a_1,n-a_1),(a_2, n-a_2),\dots$.
            Since there are $\phi(n)$ such integers $a_i$, there are $\frac{\phi(n)}{2}$ such pairs. \\
            Then, we know the sum of each pair is 
            \[ a_i + (n-a_i) = n \]
            So for all pairs, we have
            \[
                a_1 + (n-a_1) + a_2 + (n-a_2) + \dots + a_{\frac{\phi(n)}{2}} + (n-a_{\frac{\phi(n)}{2}}) = n\cdot \frac{\phi(n)}{2}
            \]
            This is equivalent to 
            \[
                a_1 + a_2+\cdots + a_{\phi(n)} = \frac{1}{2} n\phi(n)
            \]
        \end{proof}
        
        \item If $p\geq 5$ is prime, show that the product of the $\phi(p-1)$ primitive roots of $p$ is congruent to $1$ modulo $p$. (Hint: If $a$ is a primitive root of $p$, then $a^k$ is a primitive root of $p$ iff $\gcd(k,p-1) = 1$. Now use part (a).)
        \begin{proof}
            Let $g$ be a primitive root of $p$. The primitive roots of $p$ are the powers $g^k$
            where $1\le k\le p-1$ and $\gcd(k,p-1) = 1$. 
            The product $P$ of all the primitive roots of $p$ is: 
            \[ 
                P = g^{k_1}\cdot g^{k_1}\dots g^{k_{\phi(p-1)}}
            \]
            By the properties of exponents, this is equivalent to 
            \[
                P = g^{k_1+k_2+\dots+k_{\phi(p-1)}}
            \]
            From part (a), we can substitute:
            \[
                P = g^{\frac{1}{2}(p-1)\phi(p-1)}
            \]
            Then, since $g$ is a primitive root of $p$, we know that $g^{p-1}\equiv 1\pmod{p}$. 
            So, we can reduce the exponent $\pmod{p-1}$:
            \[
                P \equiv g^{\frac{1}{2}(p-1)\phi(p-1)\pmod{p-1}}\pmod{p}
            \]
            We can see that the exponent is a multiple of $p-1$. So, we can say:
            \[ 
                P\equiv 1\pmod{p}
            \]
            Thus the product of the $\phi(p-1)$ primitive roots of $p$ is congruent to $1\pmod{p}$.
        \end{proof}
    \end{enumerate}
    
    \item \begin{enumerate}
        \item Compute the table of indices modulo 17 relative to the primitive root 3.
        \begin{proof}
            First, compute all powers of $3\pmod{17}$.
            \begin{align*}
                3^1 &\equiv 3\pmod{17} \\
                3^2 &\equiv 9\pmod{17} \\
                3^3 &\equiv 10\pmod{17} \\
                3^4 &\equiv 13\pmod{17} \\
                3^5 &\equiv 5\pmod{17} \\
                3^6 &\equiv 15\pmod{17} \\
                3^7 &\equiv 11\pmod{17} \\
                3^8 &\equiv 16\pmod{17} \\
                3^9 &\equiv 14\pmod{17} \\
                3^{10} &\equiv 8\pmod{17} \\
                3^{11} &\equiv 7\pmod{17} \\
                3^{12} &\equiv 4\pmod{17} \\
                3^{13} &\equiv 12\pmod{17} \\
                3^{14} &\equiv 2\pmod{17} \\
                3^{15} &\equiv 6\pmod{17} \\
                3^{16} &\equiv 1\pmod{17}
            \end{align*}
            We can use these values to create the table of indices:
            \begin{center}
            \begin{tabular}{ |c|c c c c c c c c c c c c c c c c| } 
                \hline
                a & 1 & 2 & 3 & 4 & 5 & 6 & 7 & 8 & 9 & 10 & 11 & 12 & 13 & 14 & 15 & 16 \\ 
                \hline
                $ind_3a$ & 16 & 14 & 1 & 12 & 5 & 15 & 11 & 10 & 2 & 3 & 7 & 13 & 4 & 9 & 6 & 8 \\ 
                \hline
            \end{tabular}
            \end{center}
        \end{proof}

        Use the table to solve the following congruences:
        \item $x^{12}\equiv 13\pmod{17}$
        \begin{proof}
            We can first take the index of both sides of the congruence.
            \begin{align*}
                \ind(x^{12}) &\equiv \ind(13)\pmod{16} \\
                12\ind(x) &\equiv 4\pmod{16} \\
                3\ind(x) &\equiv 1\pmod{4}
            \end{align*}
            Then, find the multiplicative inverse of $3\pmod{4}$:
            \begin{align*}
                3x &\equiv 1\pmod{4} \\
                3x-4y &= 1 \\
                4 &= 3(1) + 1 \\
                1 &= 4(1)-3(1) \\
                1 &= 3(-1)-4(-1)
            \end{align*}
            We have $x\equiv -1\equiv 3\pmod{4}$. So $3^{-1}\pmod{4}=3$. 
            We can multiply both sides of the previous congruence by this value:
            \begin{align*}
                \ind(x) &\equiv 3\pmod{4}
            \end{align*}
            There are 4 solutions since we previously divided by a gcd of 4.
            \begin{align*}
                \ind(x) &\equiv 3,7,11,15 \pmod{16}
            \end{align*}
            \begin{center}
                \boxed{x\equiv 10,11,7,6 \pmod{17}}
            \end{center}
        \end{proof}
        \item $9x^8\equiv 8\pmod{17}$
        \begin{proof}
            We can first take the index of both sides of the congruence.
            \begin{align*}
                \ind(9x^8) &\equiv \ind(8)\pmod{16} \\
                \ind(9) + \ind(x^8) &\equiv 10\pmod{16} \\
                2 + 8\ind(x) &\equiv 10\pmod{16} \\
                8\ind(x) &\equiv 8\pmod{16} \\
                \ind(x) &\equiv 1\pmod{2} \\
                \ind(x) &\equiv 1,3,5,7,9,11,13,15\pmod{16}
            \end{align*}\begin{center}
                \boxed{x\equiv 3,10,5,11,14,7,12,6 \pmod{17}}
            \end{center}

        \end{proof}
        \item $7^x\equiv 7\pmod{17}$
        \begin{proof}
            Since $17$ is prime, the multiplicative group of integers $\pmod{17}$ has order $16$. 
            So the exponents $x$ and $1$ are congruent $\pmod{16}$.
            \[ 
                x\equiv 1\pmod{16}
            \]
            So, 
            \[
                x = 1 + 16y
            \]
            for some $y\in\ZZ$. \\
            The smallest solution is $x=1$. So, $x=1$ satisfies the congruence.

        \end{proof}
    \end{enumerate}

    \item Determine whether the congruences $x^{14}\equiv 3 \pmod{23}$ and $x^{14} \equiv 5 \pmod{23}$ are solvable.
    \begin{proof}
        $x^{14}\equiv 3 \pmod{23}$ is not solvable. 
        $x^{14} \equiv 5 \pmod{23}$ is solvable.
    \end{proof}

    \item In this problem, we will establish the existence of primitive roots for odd prime powers. Throughout, let $p$ be an odd prime. Prove the following assertions.
    \begin{enumerate}
        \item If $g$ is a primitive root of $p$ such that $g^{p-1}\not\equiv 1\pmod{p^2}$, then $g$ is a primitive root of $p^2$.
        \begin{proof}
            Let $d$ be the order of $g\pmod{p^2}$. We know
            \[
                g^d\equiv 1\pmod{p^2}
            \]
            This implies 
            \begin{align*}
                p^2 &\mid g^d - 1 \\
                p &\mid g^d - 1 \\
                g^d &\equiv 1\pmod{p}
            \end{align*}
            Since $p$ is prime, either $d\mid p-1$ or $d\mid p(p-1)$. 
            If $d\mid p-1$, then since $p-1\mid d$ we have $d=p-1$. However this can't be true 
            given $g^{p-1}\not\equiv 1\pmod{p^2}$. \\
            So, $d\mid p(p-1)$.
            Thus $g$ is a primitive root $\pmod{p^2}$.
        \end{proof}
        
        \item If $g$ is a primitive root of $p$, then either $g$ or $g+p$ (or both) is a primitive root of $p^2$.
        \begin{proof}
            If $g$ is a primitive root of $p$, then the order of $g\pmod{p^2}$ is either $p-1$
            or $p(p-1)$. If $g$ has order $p$, then $g+p$ will be a primitive root
            of $p^2$. 
        \end{proof}

        \item If $g$ is a primitive root of $p^2$, then for each positive integer $k\geq 2$, $$g^{p^{k-2}(p-1)}\not\equiv 1\pmod{p^k}.$$ (Hint: Argue by induction on $k$. For the induction step, assume it holds for $k$ and show that $g^{p^{k-2}(p-1)} = 1 + ap^{k-1}$, for some $a\in \ZZ$, where $p\nmid a$. Finish by raising both sides to the $p$th power and then reduce mod $p^{k+1}$.)
        \begin{proof}
            This holds for $k=2$. \\
            IH: Assume this is true for some $k\ge 2$. Since $\gcd(g, p^{k-1}) = \gcd(g, p^k) = 1$, 
            Euler's Theorem states:
            \[
                g^{p^{k-2}(p-1)} = g^{\phi(p^k-1)} \equiv 1\pmod{p^{k-1}}
            \]
            There exists an $a\in\ZZ$ such that
            \[
                g^{p^{k-2}(p-1)} = 1 + ap^{k-1}
            \]
            Raising both sides to the pth power, we get 
            \[
                g^{p^{k-1}(p-1)} = (1+ap^{k-1})^p \equiv 1+ap^k\pmod{p^{k+1}}
            \]
            Since $a$ is not divisible by $p$, we have
            \[
                g^{p^{k-1}(p-1)} \not\equiv 1\pmod{p^{k+1}}
            \]
            Thus proving the induction step.
        \end{proof}
        
        \item If $p$ is an odd prime and $k\geq 1$, then there exists a primitive root for $p^k$. In fact, there exists an integer $g$ that is a primitive root for all positive powers of $p$.
        \begin{proof}
            Let $g$ be a primitive root of $p$ and $g^{p^{k-2}(p-1)}\not\equiv 1\pmod{p^k}$.
            Let $n$ be the order of $g\pmod{p^k}$. $n$ divides $\phi(p^k)=p^{k-1}(p-1)$. \\
            Since $g^n\equiv 1\pmod{p^k}$ implies $g^n\equiv 1\pmod{p}$, we have $p-1\mid n$. 
            Then, $n=p^m(p-1)$ where $0\le m\le k-1$. If $n\ne p^{k-1}(p-1)$, then $p^{k-2}(p-1)$
            would be divisible by $n$ and we would have
            \[ g^{p^{k-2}(p-1)}\equiv 1\pmod{p^k} \]
            However this contradicts how we set $g$ above. Therefore,
            $n=p^m(p-1)$ and $g$ is a primitive root for $p^k$.

        \end{proof}
    \end{enumerate}

    
\end{enumerate}

\end{document}