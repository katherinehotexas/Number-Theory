\documentclass[11pt]{article}
%\usepackage[spanish]{babel}
\usepackage[utf8]{inputenc}
\usepackage[OT1]{fontenc}
\usepackage{amsfonts, amsmath, amsthm, amssymb}
\usepackage{mathtools}
\usepackage{graphicx}
\usepackage{listings}
\usepackage[margin=1in]{geometry}
\usepackage{xcolor}
\usepackage{hyperref}
\hypersetup{
    colorlinks=true,
    linkcolor=blue,
    filecolor=magenta,      
    urlcolor=blue,
    pdftitle={Overleaf Example},
    pdfpagemode=FullScreen,
    }

\theoremstyle{definition}
\newtheorem{defn}{Definition}
\newtheorem{thm}{Theorem}
\newtheorem*{thm*}{Theorem}

\usepackage[Glenn]{fncychap}
% Sonny Lenny Glenn Conny Rejne Bjarne Bjornstrup

\newcommand{\ZZ}{\mathbb{Z}}
\newcommand{\NN}{\mathbb{N}}
\newcommand{\QQ}{\mathbb{Q}}
\newcommand{\ord}{\operatorname{ord}}

\newcommand{\leg}[2]{\genfrac(){}{0}{#1}{#2}}
\newcommand{\legp}[1]{\leg{#1}{p}}

\newcommand{\legendre}[2]{\ensuremath{\left( \frac{#1}{#2} \right) }}

\newcommand{\Mod}[1]{\ (\mathrm{mod}\ #1)}


\DeclarePairedDelimiter\abs{\lvert}{\rvert}


\title{M328K: Homework 9}
\author{Katherine Ho}
\date\today
\begin{document}
\maketitle

\begin{enumerate}
    \item Let $p$ be an odd prime with $p\equiv 5\pmod{8}$. Find an explicit solution to the congruence $x^2 \equiv -1 \pmod p$. (Hint: You know $(2/p) = -1$. Apply Euler's criterion.)
    \begin{proof}
        Given $p\equiv 5\pmod{8}$, we know that $\legendre{2}{p} = -1$. By Euler's Criterion,
        \[
            2^{\frac{p-1}{2}} \equiv -1 \pmod{p}
        \]
        Consider $x = 2^{\frac{p-1}{4}}$. From the above property we have
        \[
            x^2 = {2^{\frac{p-1}{4}}}^2 \equiv 2^{\frac{p-1}{2}} \equiv -1
        \]
        Thus an explicit solution to $x^2\equiv -1\pmod{p}$ is $x = 2^{\frac{p-1}{4}}$.
    \end{proof}

    \item \begin{enumerate}
        \item Use the previous problem to find a solution $x$ to the congruence $x^2 \equiv -1 \pmod{541}$. (Reduce modulo $p$ so that $0<x<541$)
        \begin{proof}
            From the previous problem, $x = 2^{\frac{541-1}{4}} = 2^{135}$.
            Using binomial expansion:
            \[
                2^{135} = 2^{128}\cdot 2^4\cdot 2^2\cdot 2^1
            \]
            \begin{align*}
                2^1\equiv 2\pmod{541} \\
                2^2\equiv 4\pmod{541} \\
                2^4\equiv 16\pmod{541} \\
                2^8\equiv 256\pmod{541} \\
                2^{16}\equiv 75\pmod{541} \\
                2^{32}\equiv 215\pmod{541} \\
                2^{64}\equiv 240\pmod{541} \\
                2^{128}\equiv 254\pmod{541} \\
            \end{align*}
            By substitution,
            \[
                2^{135} = 254\cdot 16\cdot 4\cdot 2 \equiv 32512\equiv 52\pmod{541}
            \]
            Thus $x\equiv 52\pmod{541}$. 
        \end{proof}
    
        \item Use part (a) to express a multiple of $541$ as a sum of squares.
        \begin{proof}
            We can substitute $x\equiv 52\pmod{541}$ into $x^2\equiv -1\pmod{541}$. 
            \[
                52^2 \equiv -1 \pmod{541} 
            \]
            By the definition of congruence, 
            \[
                52^2 + 1^2 = 541\cdot k \quad\text{for some } k\in\ZZ
            \]
            Thus a multiple of 541 can be expressed as a sum of squares.
        \end{proof}
    
        \item Use Fermat's method of descent to express $541$ as a sum of squares.
        \begin{proof}
            Since $541\equiv 1\pmod{4}$, then 
            \[
                \legendre{-1}{541} = 1 \rightarrow x^2 + 1 = k\cdot 541 \rightarrow x = 52, k = 5
            \]
            Suppose
            \begin{align*}
                52^2 + 1^2 = 5\cdot 541
            \end{align*}
            Then find $u,v$ such that 
            \begin{align*}
                u\equiv 52\pmod{5} \rightarrow u = 2\\
                v\equiv 1\pmod{5} \rightarrow v = 1 
            \end{align*}
            Thus
            \begin{align*}
                \rightarrow 52^2+1^2 \equiv 2^2 + 1^2 \equiv 0 \pmod{5}
            \end{align*}
            Then,
            \begin{align*}
                (52^2+1^2)(2^2+1^2) = 5^2\cdot 1\cdot 541 \\
                (2\cdot 52+1\cdot 1)^2 + (52-2(1))^2 = 5^2\cdot 541 \\
                2(52)+(1) \equiv 52^2 + 1^2\equiv 52^2\cdot 1^2 \equiv 0\pmod{5} \\
                52-2\cdot1\equiv 52-52\equiv 0\pmod{5} \\
                (\frac{2(52) + 1}{5})^2 + (\frac{52-2(1)}{5})^2 = 541
            \end{align*}
            \begin{center}
                \boxed{21^2 + 10^2 = 541}
            \end{center}
        \end{proof}
    
    \end{enumerate}

    \item Let $\alpha = a + bi$ be a Gaussian integer. Show that if $N(\alpha) = a^2 + b^2$ is divisible by an odd prime $p$ with $p\equiv 3 \pmod 4$, then both $a$ and $b$ are divisible by $p$.
    
    (Hint: By contradiction, assume $a$ and $b$ are not divisible by $p$. Then the Legendre symbols $\legp{a}$ and $\legp{b}$ are well-defined. Now derive a contradiction.)
    \begin{proof}
        Assume $a$ and $b$ are both not divisible by $p$. 
        Then the Legendre symbols $\legp{a}$ and $\legp{b}$ are well-defined.

        Since $N(\alpha) = a^2 + b^2$ is divisible by $p$, then
        \[
            a^2 + b^2 \equiv 0\pmod{p}
        \]
        Then 
        \[
            a^2\equiv -b^2\pmod{p}
        \]
        Taking the Legendre symbol of both sides, we have 
        \[
            \legendre{a^2}{p} = \legendre{-b^2}{p}
        \]
        We see that $a^2$ is a QR of $p$ so $\legendre{a^2}{p} = 1$. 

        And, 
        \[
            \legendre{-b^2}{p} = \legendre{-1}{p}\legendre{b^2}{p} = \legendre{-1}{p}\cdot 1 = \legendre{-1}{p} = -1
        \]
        
        By substitution: 
        \[
            \legendre{a^2}{p} = \legendre{-b^2}{p} \rightarrow 1 = -1
        \]

        This is a contradiction. Hence if $N(\alpha) = a^2 + b^2$ is divisible by an odd prime $p$ 
        with $p\equiv 3 \pmod 4$, then both $a$ and $b$ are divisible by $p$.
        
    \end{proof}

    \item Show how to factor $27007$ if you know both $885$ and $7816$ are square roots of $22$ modulo $27007$.
    \begin{proof}
        If 885 and 7816 are square roots of 22 modulo 27007, then 
        \begin{align*}
            885^2 &\equiv 7816^2\pmod{27007} \\
            27007 &\mid (885^2-7816^2) \\
            27007 &\mid (885+7816)(885-7816)
        \end{align*}
        We can use this to find the factors of 27007. 
        \begin{align*}
            885+7816 &\equiv 8701\pmod{27007} \\
            885-7816 &\equiv -6931\equiv 20076\pmod{27007}
        \end{align*}
        Then find $\gcd(8701,27007)$ and $\gcd(20076,27007)$ to get the factors. 
        \begin{align*}
            27007 &= 8701*3 + 904 \\
            8701  &= 904*9 + 565 \\
            904   &= 565*1 + 339 \\
            565   &= 339*1 + 226 \\
            339   &= 226*1 + 113 \\
            226   &= 113*2 + 0
        \end{align*}
        $gcd(8701,27007) = 113$.
        \begin{align*}
            27007 &= 20076*1 + 6931 \\
            20076 &= 6931*2 + 6214 \\
            6931  &= 6214*1 + 717 \\
            6214  &= 717*8 + 478 \\
            717   &= 478*1 + 239 \\
            478   &= 239*2 + 0 
        \end{align*}
        $\gcd(20076,27007) = 239$.

        Thus $27007 = 113 \cdot 239$.
    \end{proof}

    \item Find the four incongruent solutions of the quadratic congruence $x^2 \equiv 30 \Mod{133}$.
    \begin{proof}
        133 can be factored as $7*19$. Then we can solve
        \[
            x^2 \equiv 30 \pmod{7} \quad\text{and}\quad
            x^2 \equiv 30 \pmod{19} 
        \]
        \begin{align*}
            x^2 &\equiv 30 \pmod{7} \\
            x^2 &\equiv 2 \pmod{7} \\
            x &\equiv \pm 3\pmod{7}
        \end{align*}
        \begin{align*}
            x^2 &\equiv 30 \pmod{19} \\
            x^2 &\equiv 11 \pmod{19} \\
            x &\equiv \pm 7\pmod{19}
        \end{align*}
        Now we can look at these four cases and solve with the Chinese Remainder Theorem:
        \begin{itemize}
            \item $x\equiv 3\pmod{7},  \quad x\equiv 7\pmod{19} \quad\rightarrow x\equiv 45\pmod{133}$
            \item $x\equiv -3\pmod{7}, \quad x\equiv 7\pmod{19} \quad\rightarrow x\equiv 102\pmod{133}$
            \item $x\equiv 3\pmod{7},  \quad x\equiv -7\pmod{19} \quad\rightarrow x\equiv 31\pmod{133}$
            \item $x\equiv -3\pmod{7}, \quad x\equiv -7\pmod{19} \quad\rightarrow x\equiv 88\pmod{133}$
        \end{itemize}
        Thus the four incongruent solutions are $x \equiv 31, 45, 88, 102$.
    \end{proof}

    \item We have seen that any prime of the form $p = 4k+1$ can be expressed as a sum of two squares. Prove that this representation is unique (except for swapping the order of the two summands).

    (Hint: Suppose that $p = a^2 + b^2 = c^2 + d^2$, where $a,b,c,d$ are all positive integers. First argue that $a^2d^2 \equiv b^2 c^2 \Mod p$, so then $ad \equiv bc \Mod p$ or $ad \equiv -bc \Mod p$. Next, argue that these two cases imply, respectively, that $ad - bc = 0$ or $ad + bc = p$. If $ad + bc = p$, use the product formula to write $p^2$ as a sum of squares and then use the resulting equation to conclude $ac - bd = 0$. Thus, it follows that either $ad = bc$ or $ac = bd$. Now draw the rest of the owl.)
    \begin{proof}
        Suppose this representation is not unique; that is $p = a^2 + b^2 = c^2 + d^2$, 
        where $a,b,c,d$ are all positive integers. 
        \begin{align*}
            p &= a^2 + b^2 \\
            pd^2 &= d^2(a^2+b^2) \\
            p &= c^2 + d^2 \\
            pb^2 &= b^2(c^2 + d^2) \\
        \end{align*}
        Subtracting these two equations, we get
        \begin{align*}
            pd^2 - pb^2 &= d^2(a^2+b^2) - b^2(c^2 + d^2) \\
            p(d^2 - b^2) &= a^2d^2 - b^2c^2
        \end{align*}
        Thus $a^2d^2 \equiv b^2c^2 \pmod{p}$. Taking the square root of both sides, we get 
        \[
            ad\equiv bc\pmod{p} \quad\text{or}\quad ad\equiv -bc\pmod{p}
        \]
        Case 1: $ad\equiv bc\pmod{p}$.
        \begin{align*}
            ad &\equiv bc\pmod{p} \\
            ad - bc &\equiv 0\pmod{p} \\
            ad - bc &= p\cdot x
        \end{align*}
        In this case, $ad-bc$ must be 0.

        Case 2: $ad\equiv -bc\pmod{p}$
        \begin{align*}
            ad &\equiv -bc\pmod{p} \\
            ad + bc &\equiv 0\pmod{p} \\
            ad + bc &= p\cdot x 
        \end{align*}
        Since $a,b,c,d<p$, then $ad+bc = p$. 

        If $ad + bc = p$, by the product formula we have
        \begin{align*}
            p^2 &= (ad+bc)(ad+bc) \\
            &= a^2d^2 + 2abcd + b^2c^2 \\
        \end{align*}
        From this we see that $ac-bd = 0$.

        Now we know that $ad = bc$ or $ac = bd$. 
        
        Since $a\ne b$,
        then this can only be true for $a=d$ and $b=c$. 
        Thus $p=a^2+b^2$, that is, there is a unique sum of 2 squares for any prime.
    \end{proof}

\end{enumerate}

\end{document}

