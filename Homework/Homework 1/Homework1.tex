\documentclass[letterpaper]{article}

\usepackage{amssymb}
\usepackage{amsthm}
\usepackage{amsmath}
\usepackage[left=1in, right=1in, top=1in, bottom=1in]{geometry}

\newtheorem{lemma}{Lemma}
\newtheorem{sublemma}{Lemma}[lemma]

\title{M 328K: Homework 1}
\author{Katherine Ho}
\date\today

\begin{document}
\maketitle

\begin{enumerate}
	\item Show that $(3!)^n\ |\ (3n)!$ for all $n\geq0$.
	\begin{proof} [Proof by Induction]
		We aim to show that 
		\[
			(3n)!=(3!)^n\cdot k \text{ for some } k\in\mathbb{Z}.
		\]
		Base case ($n=0$): $0! = (3!)^0\cdot k,\ k = 1$ \\
		Base case ($n=1$): $3! = (3!)\cdot k,\ k = 1 $

		Inductive Hypothesis: Assume that $(3n)!=(3!)^n\cdot k$ is true for some $k\in\mathbb{Z}$
		and all $n\geq 0$. \\
		Consider $n+1$:
		\begin{align*}
			(3(n+1))! &= (3n+3)! \\
			&= (3n+3)(3n+2)(3n+1)(3n)! \\
			&= (3n+3)(3n+2)(3n+1)(3!)^n\cdot k & \text{By the IH}
		\end{align*}

		\begin{lemma}
			The product of any two consecutive integers is even.
		\end{lemma}
			We want to show that $n(n+1)$ is even $\forall n \in \mathbb{Z}$.
			\begin{enumerate}
				\item Case 1: n is even.
				We have $n=2a$, where $a\in\mathbb{Z}$.
				\begin{equation*} \begin{split}
					n(n+1) = 2a(2a+1) 
					= 2(2a^2+a)
				\end{split} \end{equation*}
				$2a^2+a\in\mathbb{Z}$, thus n(n+1) is even.
				\item Case 2: n is odd.
				We have $n=2b+1$, where $b\in\mathbb{Z}$.
				\begin{equation*}
					\begin{split}
						n(n+1) 
						& = (2b+1)(2b+1+1) \\
						& = 4b^2+6b+2 \\
						& = 2(2b^2+3b+1)
					\end{split}
				\end{equation*}
				$2b^2+3b+1\in\mathbb{Z}$, thus $n(n+1)$ is even.
			\end{enumerate}
			The statement is true in both cases. 
			Therefore, the product of any two consecutive integers is even.\\\\
		By Lemma 1: $(3n+2)(3n+1)=2p$ for some $p\in\mathbb{Z}$.
		\begin{align*}
			(3(n+1))! &= (3n+3)(2p)(3!)^n\cdot k \\
			&= 3(n+1)(2)(p)(3!)^n\cdot k \\
			&= (3!)^{n+1}\cdot ((n+1)\cdot p\cdot k)
		\end{align*}
		where $(n+1)\cdot p\cdot k\in\mathbb{Z}$. Hence
		\[
			(3(n+1))! = (3!)^{n+1}\cdot k
		\]
		Thus $(3!)^n\ |\ (3n)!$ for all $n\geq0$.
			
	\end{proof}

	\item Show that if $a$ and $b$ are odd integers, then $8\ |\ a^2-b^2$.
	\begin{proof}
		% By the divisibility theorem, we have
		We aim to show that $a^2-b^2 = 8k,\text{ for some }k\in\mathbb{Z}$.
		
		Given a and b are odd integers, they can be rewritten as 
		$a=2m+1$ and $b=2n+1$ for some $m,n\in\mathbb{Z}$.\\\\
		Then, we have
		\begin{equation*}
			\begin{split}
				a^2-b^2 
				& = (2m+1)^2-(2n+1)^2 \\
				& = 4m^2+4m+1-(4n^2+4n+1) \\
				& = 4(m(m+1)-n(n+1))
			\end{split}
		\end{equation*}
		By Lemma 1: $4(m(m+1)-n(n+1)) = 4(2r-2s)$ for some $r,s\in\mathbb{Z}$.
		\[
			4(2r-2s) = 8(r-s)
		\]
		We now have $a^2-b^2 = 8(r-s)$, where $r-s$ is an integer. \\
		Thus if $a$ and $b$ are odd integers, then $8\ |\ a^2-b^2$.
		
	\end{proof}
	
	\item Consider the following sequence of integers:
	\[
		11, 111, 1111, \dots
	\]
	\begin{enumerate}
		\item Show by induction that each integer in the sequence can be written in the form $4k + 3$.
			\begin{proof} [Proof by Induction]
				Each element in the sequence can be described with:
				\[
					x_n = \sum_{i=0}^{n+1}10^i
				\]
				Let $P(n)$ be the statement that $x_n$ can be written in the form $4k+3$.\\\\
				Base case: $P(0)$
				\[
					x_0 
					= \sum_{i=0}^{0+1}10^i
					= 10^0 + 10^1
					= 11
					= 4(2)+3, \text{ where }2\in\mathbb{Z}
				\]
				Inductive Hypothesis: Assume $P(n)$ is true. That is,
				\[
					x_n = \sum_{i=0}^{n+1}10^i = 4k+3 \text{ for some }k\in\mathbb{Z}
				\]
				$P(n+1)$:
				\begin{align*}
					x_{n+1} & =\sum_{i=0}^{(n+1)+1}10^i \\
					& = 10^{n+2}+ \sum_{i=0}^{n+1}10^i \\
					& = 10^{n+2}+4k+3 && \text{By the IH} \\
					& = 100 \cdot 10^n+4k+3 \\
					& = 4(25 \cdot 10^n+k)+3
				\end{align*}
				where $(25 \cdot 10^n+k) \in \mathbb{Z}$. 
				Thus $P(n+1)$ is true, proving that each integer in the sequence can be written in the form $4k+3$.\\
			\end{proof}

		\item Use the previous result together with the division algorithm to show that no integer in
		the sequence is a perfect square.
			\begin{proof} [Proof by Contradiction]
				Given that each integer in the sequence can be written as $4k+3$, suppose 
				that each integer can be written as a perfect square. That is,
				\[ 4k+3=a^2, a\in\mathbb{Z} \]
				\begin{enumerate}
					\item Case 1: a is odd, ie. $a=2p+1, p\in\mathbb{Z}$
						\begin{align*}
							4k+3 &= (2p+1)^2 \\
							&= 4p^2+4p+1 \\
							&= 4(p^2+p)+1
						\end{align*}
						By the division algorithm, $\exists$ integers $q$ and $r$ such that 
						$a = bq + r$. In this instance, $q=4$ and $r=3$. If a is an odd 
						integer, $r=1$, a contradiction.
					\item Case 2: a is even, ie. $a=2p, p\in\mathbb{Z}$
						\begin{align*}
							4k+3 &= (2p)^2 \\
							&= 4(p^2)
						\end{align*}
						If a is an even integer, $r=0$, a contradiction.
				\end{enumerate}
				The supposition is false, thus no integer in the sequence is a 
				perfect square.
			\end{proof}
	\end{enumerate}

	\item Let $a$ and $b$ be coprime integers. Show that $ab$ and $a+b$ are also coprime.
	\begin{proof}
		The contrapositive of the given statement is as follows:
		\[
			\text{"If } ab \text{ and } a+b \text{ are not coprime, then } 
			a \text{ and } b \text{ are not coprime."}
		\]
		Suppose that $ab$ and $a+b$ are not coprime. That is, suppose 
		$d|ab$ and $d|a+b$, for some $d\in\mathbb{Z}$. Given that $a$ and $b$
		are coprime,
		\[
			d|ab \implies d|a \text{ OR } d|b
		\]
		Say $d|a$. This implies that $a=dc$ for some $c\in\mathbb{Z}$. \\
		Next, we have
		\begin{align*}
			d|a+b \implies a+b &= dk \text{ for some } k\in\mathbb{Z} \\
			b &= dk - a \\
			b &= dk - dc \\
			b &= d(k-c) 
		\end{align*}
		This suggests that $a$ and $b$ are not coprime due to sharing a common factor $d$.
		Thus the contrapositive statement is true and so the given statement is true.

	\end{proof}

	\item Prove the following properties of the greatest common divisor (without appealing to prime
	factorization):
	\begin{enumerate}
		\item If $\gcd(a, b) = \gcd(a, c) = 1$, then $\gcd(a, bc) = 1$.
			\begin{proof}
				We want to show that $ax+bcy = 1$ for some $x,y\in\mathbb{Z}$.
				Given $\gcd(a, b) = \gcd(a, c) = 1$ and Bezout's Theorem, we can write
				\[
					ax_1 + by_1 = ax_2 + cy_2 = 1 \text{ for some } x_1,y_1,x_2,y_2 \in\mathbb{Z}
				\]			
				\begin{enumerate}
					\item Multiply the first expression by c. We get
					\begin{align*}
						acx_1+bcy_1 &= c \\
						a(cx_1)+bc(y_1) &= c
					\end{align*}
					\item Multiply the second expression by b. We get
					\begin{align*}
						abx_2+bcy_2 &= c \\
						a(bx_2)+bc(y_2) &= c
					\end{align*}
				\end{enumerate}
				Now, we have 
				\[ a(cx_1)+bc(y_1) = a(bx_2)+bc(y_2) = c \] 
				Since $ax_1+by_1=ax_2+cy_2=1$,
				\[ a(cx_1)+bc(y_1) = a(bx_2)+bc(y_2) = 1 \] 
				Thus, $ax+bcy = 1$ and $\gcd(a, bc) = 1$.
			\end{proof}

		\item If $\gcd(a, b) = 1$, then $\gcd(ac, b) = \gcd(c, b)$.
			\begin{proof}
				Suppose $\gcd(ac,b)=z_1$ and $\gcd(c,b)=z_2$. We have
				\begin{align*}
					ac(x_1)+b(y_1) &= z_1 \\
					c(x_2)+b(y_2) &= z_2
				\end{align*}
				Given $ax+by=1$ for some $x,y\in\mathbb{Z}$, we have 
				\begin{enumerate}
					\item $axz_1+byz_1=z_1$
						\begin{align*}
							ax(ac(x_1)+b(y_1))+byz_1 &= z_1 \\
							c(a^2xx_1)+b(axy_1+y_1z_1) &= z_1
						\end{align*}
						$\therefore z_2 | z_1$ since $z_2$ can divide any linear combination of c and b.
					\item $axz_2+byz_2=z_2$
						\begin{align*}
							ax(cx_2+by_2)+by(z_2) &= z_2 \\
							ac(xx_2)+b(axy_2+yz_2) &= z_2
						\end{align*}
						$\therefore z_1 | z_2$ since $z_2$ can divide any linear combination of ac and b.
				\end{enumerate}
				Now, we have $z_1|z_2$ and $z_2|z_1$. 
				\begin{lemma}
					For some integers $x_1,x_2\in\mathbb{Z}$, if $x_1|x_2$ and $x_2|x_1$, 
					then $x_1=x_2$ or $x_1=-x_2$.
				\end{lemma}
						Given $x_1|x_2$ and $x_2|x_1$, we have 
						\begin{align*}
							x_2 &= x_1\cdot a \text{ for some } a\in\mathbb{Z} \\
							x_1 &= x_2\cdot b \text{ for some } b\in\mathbb{Z} \\
							x_1 &= (x_1\cdot a)\cdot b \\
							1 &= a\cdot b
						\end{align*} 
						This tells us $a=b=1$ or $a=b=-1$.
						\begin{enumerate}
							\item $a=b=1 \implies x_1=x_2$
							\item $a=b=-1 \implies x_1=-x_2$
						\end{enumerate}
						Thus if $x_1|x_2$ and $x_2|x_1$, then $x_1=x_2$ or $x_1=-x_2$. \\\\
				By Lemma 2, $z_1=z_2$ since $z_1$ and $z_2$ both must be positive integers. \\
				Thus $\gcd(ac, b) = \gcd(c, b)$.
			\end{proof}

		\item If $\gcd(a, b) = 1$, $d | ac$, and $d | bc$, then $d | c$.
			\begin{proof}
				Given $d | ac$ and $d | bc$, we have
				\begin{align*}
					ac &= dp \text{ for some } p\in\mathbb{Z} \\
					bc &= dq \text{ for some } q\in\mathbb{Z}
				\end{align*} 
				Given a and b are coprime, we have
				\begin{align*}
					ax+by &= 1 \text{ for some } x,y\in\mathbb{Z} \\
					acx+bcy &= c \\
					dpx+dqy &= c \\
					d(px+qy) &= c
				\end{align*}
				Since $x,y,p,q,\in\mathbb{Z}$, $d|c$.
			\end{proof}

		\item If $\gcd(a, b) = 1$, then $\gcd(a^2, b^2) = 1$.
			\begin{proof}
				Given a and b are coprime, there exist some $x,y\in\mathbb{Z}$ such that
				$ax+by=1$.
				\begin{align*}
					1 &= ax + by \\
					1^3 &= (ax+by)^3 \\
					1 &= (ax+by)^2\cdot (ax+by) \\
					1 &= (a^2x^2+2axby+b^2y^2)\cdot (ax+by) \\
					1 &= a^3x^3 + a^2x^2by + 2a^2x^2by + 2axb^2y^2 + b^2y^2ax + b^3y^3 \\
					1 &= a^2(ax^3 + x^2by + 2x^2by) + b^2(2axy^2 + y^2ax + by^3)
				\end{align*}
				This equation can be rewritten as 
				\[ a^2x_1 + b^2y_2 = 1 \]
				where 
				\begin{align*}
					x_1 &= ax^3 + x^2by + 2x^2by \in\mathbb{Z} \\
					y_1 &= 2axy^2 + y^2ax + by^3 \in\mathbb{Z}
				\end{align*}
				Therefore, by Bezout's Theorem, $\gcd(a^2, b^2) = 1$.
			\end{proof}

	\end{enumerate}

	\item Use the Euclidean Algorithm to obtain integers $x$ and $y$ satisfying
	\[
		119x+272y=\gcd(119,272)
	\]
	\begin{align*}
		272 & = 2\cdot119+34 \\
		119 & = 3\cdot34+17 \\
		34 & = 2\cdot17+0 
	\end{align*}
	\begin{center}
		\boxed{GCD = 17}
	\end{center}
	\begin{align*}
		17 & = 119-3\cdot34 \\
		& = 119-3(272-2\cdot119) \\
		& = 119-(3\cdot272)+6(119) \\
		& = 7(119)-3(272)
	\end{align*}
	\begin{center}
		\boxed{x = 7,\ y = -3}
	\end{center}
\end{enumerate}

\end{document}
