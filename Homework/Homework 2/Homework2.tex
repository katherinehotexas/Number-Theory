\documentclass[11pt]{article}
\usepackage[utf8]{inputenc}
\usepackage[OT1]{fontenc}
\usepackage{amsfonts, amsmath, amsthm, amssymb}
\usepackage{mathtools}
\usepackage{graphicx}
\usepackage{listings}
\usepackage[margin=1in]{geometry}
\usepackage{xcolor}
\newcommand{\ZZ}{\mathbb{Z}}
\newcommand{\NN}{\mathbb{N}}
\newcommand{\QQ}{\mathbb{Q}}

\DeclarePairedDelimiter\abs{\lvert}{\rvert}

\title{M328K: Homework 2}
\author{Katherine Ho}
\date\today

\begin{document}
\maketitle
    \begin{enumerate}
    \item Let $\mathbb G$ denote the set of rational numbers that are greater than or
    equal to 1. Call an element $x\in \mathbb G$ a $\mathbb G$-\emph{prime} if it
    cannot be factored as $x=yz$, where $y,z\in \mathbb G$, unless $y=1$ or $z=1$. Find
    all $\mathbb G$-primes. (Note: everything you do on homework should be assumed
    to be ``with proof'' unless otherwise specified.) Is it the case that every
    element of $\mathbb G$ can be factored as a product of $\mathbb G$-primes?
    \begin{proof}
        Any rational number $x$ can be expressed as the product of two rational numbers.
        \[ 
            x=\frac{p}{q}=\frac{p\cdot r}{1}\cdot\frac{1}{q\cdot r} \quad\text{ where } p,q,r\in\ZZ \text{ and } q\ne 0
        \]
        This is true since any integer $r$ can be multiplied by the first factor and divided by the 
        second factor to create new rational numbers.
        $r$ is then cancelled out during multiplication to yield the same $x$. \\
        However, the only number that cannot be factored as $x=yz$ unless $y=1$ or $z=1$ is 1.
        \begin{align*}
            1 &= \frac{p}{1}\cdot\frac{1}{q} \quad\text{where } p=1 \text{ and } q=1
        \end{align*}
        So, the only $\mathbb{G}$-prime is 1.
        \[ \{x\mid x\text{ is a }\mathbb{G}\text{-prime} \} = \{ 1 \} \]
    \end{proof}
    Every element in $\mathbb{G}$ can be factored as a product of $\mathbb G$-primes since 
    the only element is 1, which can be factored as a product of itself.

    \item Prove each of the following assertions:
        \begin{enumerate}
        \item Any prime of the form $3n+1$ is also of the form $6m+1$.
        \begin{proof}
            First, consider two cases.
            \begin{enumerate}
                \item n is odd. ie. $n=2a+1$ for some $a\in\ZZ$ \\
                \begin{align*}
                    3n+1 &= 3(2a+1) + 1 \\
                    &= 6a+4 \\
                    &=2(3a+2)
                \end{align*}
                Thus we have $3n+1$ is even.                
                \item n is even. ie. $n=2a$ for some $a\in\ZZ$
                \begin{align*}
                    3n+1 &= 3(2a)+1 \\
                    &= 2(3a)+1
                \end{align*}
                Thus we have $3n+1$ is odd.
            \end{enumerate}
            We know that 2 is the only even prime number since all even numbers 
            greater than 2 are divisible by 2. Also, 2 cannot be expressed in the 
            form $3n+1$. Thus any prime of the form $3n+1$ must be odd, where
            n is even. So, suppose $n=2m$ for some $m\in\ZZ$.
            \[ 3n+1 = 3(2m)+1 = 6m+1 \]
            Thus any prime of the form $3n+1$ is also of the form $6m+1$.
        \end{proof}

        \item If $p$ is a prime and $p\mid a^n$, then $p^n\mid a^n$.
        \begin{proof}
            Since $p\mid a^n$, $\exists a_k\in a^n \text{ such that }p\mid a_k$.
            Since $a_k=a$, we have
            \begin{align*}
                a &= px &\text{for some }x\in\ZZ \\
                a^n &= p^nx^n &\text{By algebra}
            \end{align*}
            Thus $p^n|a^n$.
        \end{proof}

        \item If $p\neq 5$ is an odd prime, then either $p^2-1$ or $p^2 + 1$ is
        divisible by 10.
        \begin{proof}
            If $p\neq 5$ is odd, then $p^2$ is odd. Also, all odd prime numbers must 
            end with $1,3,7,$ or $9$ so that they can't be divided by 2 or 5.
            Knowing this, 
            \[
                p \equiv 1,3,7,9 \pmod{10}
            \]
            This means that
            \[
                p^2 \equiv 1,-1,-1,1 \pmod{10}
            \]
            Consider the following cases:
            \begin{enumerate}
                \item $p\equiv 1 \text{ or }9 \pmod{10}$
                \begin{align*}
                    p^2 &\equiv 1 \pmod{10} \\
                    p^2-1 &\equiv 0 \pmod{10} \\
                    10 &\mid p^2-1
                \end{align*}
                \item $p\equiv 3 \text{ or }7 \pmod{10}$
                \begin{align*}
                    p^2 &\equiv -1 \pmod{10} \\
                    p^2+1 &\equiv 0 \pmod{10} \\
                    10 &\mid p^2+1
                \end{align*}
            \end{enumerate}
            Thus for any odd prime $p\neq 5$, either $p^2-1$ or $p^2 + 1$ is
        divisible by 10.
        \end{proof}

        \end{enumerate}

    \item
        \begin{enumerate}
            \item Find all prime numbers that divide $50!$. Prove that your list of primes
            is complete.
            \begin{proof}
                First, we have
                \[ 50! = (50)(49)\dots(2)(1) \]
                Each factor $n$ in this product is an integer from 1 to 50. 
                Each $n$ can be written as a product 
                of primes by the Fundamental Theorem of Arithmetic. 
                \begin{align*}
                    % n\mid 50! \text{ where } 1\leq n\leq 50 \\
                    n = p_1^{a_1}p_2^{a_2}\dots p_{k_n}^{a_{k_n}} \qquad\text{ for } 1\le n\le 50
                \end{align*}
                The prime factorization of $50!$ is the product of the prime factorizations
                of each n.
                \[
                    50! = \prod_{n=1}^{50} p_1^{a_1}p_2^{a_2}\dots p_{k_n}^{a_{k_n}}
                \]
                Then, it can be said that for each integer from 1 to 50, none of their
                prime factorizations will contain a prime greater than 50. 
                Thus the prime numbers that divide $50!$ are all of the prime numbers between 1 and 50.
                \[
                    \{p\mid p \text{ prime and } p \text{ divides }50!\}=\{2,3,5,7,11,13,17,19,23,29,31,37,41,43,47\}
                \]
            \end{proof}

            \item Prove that $n>4$ is composite, then $n\mid (n-1)!$
            \begin{proof}
                First, we know that
                \[
                    (n-1)! = (1)(2)\dots(n-1)
                \]
                Since $n>4$ is composite, we can say $n=ab$ for some $a,b\in\ZZ$ and $1<a<b<n$. \\
                Also, $a$ and $b$ must be factors in $(n-1)!$.
                \[
                    (n-1)! = (1)(2)\dots(a-1)(a)(a+1)\dots(b-1)(b)(b+1)\dots(n-1)
                \]
                Then, substitute $(a)(b)=n$.
                \[
                    (n-1)! = (1)(2)\dots(a-1)(a+1)\dots(b-1)(b+1)\dots(n-1)(n) 
                \]
                Thus $n\mid(n-1)!$.
            \end{proof}
        \end{enumerate}

    \item An integer is called \emph{square-free} if it is not divisible by the square
    of any integer greater than 1. Prove the following:
        \begin{enumerate}
            \item An integer $n>1$ is square-free if and only if $n$ can be factored into a
            product of distinct primes.
            \begin{proof} [Proof by Contradiction]
                First, assume $n$ is square-free. It can be represented as:
                \[
                    n = (p_1^{a_1})(p_2^{a_2})\dots(p_k^{a_k})
                \]
                If some $a_i\ge 2$, then $p_i^2\mid n$. 
                However, this is a contradiction as $n$ does not contain distinct primes. 
                So, all $a_i$ must be 1.
                Thus $n$ is square-free iff $n$ can be factored into a product of distinct primes.
            \end{proof}

            \item Every integer $n>1$ is the product of a square-free integer and a perfect
            square. (Hint: Use the canonical factorization of $n$.)
            \begin{proof} 
                Every integer $n$ can be expressed as a product of primes:
                \[
                    n = (p_1^{a_1})(p_2^{a_2})\dots(p_k^{a_k})
                \]
                where each $p_i$ is prime and each $a_i$ is a positive integer.
                $a_i=2q_i+r_i$, where $r_i=0$ for even values of $a_i$ and $r_i=1$ for odd values of
                $a_i$. For odd values of $a$, we have
                \[
                    p_i^{a_i} = p_i^{2q_i+1} = p_i^{2q_i}p_i^1        
                \]
                Now, we can write $n$ as the product of primes either to 
                the power of 1 or $2q_i$. Then, by the commutative and associative 
                properties n can be rearranged to be a product of two groups of 
                primes.
                \[
                    n = ((p_i)\dots(p_j))((p_k^{2q_k})\dots(p_l^{2q_l}))
                \]
                The first factor is the product of distinct primes. The second factor
                is a perfect square since it is the product of primes with even exponents. 
                Thus every integer $n>1$ is the product of a square-free integer and a 
                perfect square.

            \end{proof}
        \end{enumerate}

    \item 
    \begin{enumerate}
        \item Suppose $a\equiv b\pmod m$ and $n\mid m$. Prove that $a\equiv b\pmod n$.
        \begin{proof}
            First, we have $m=nx$ for some $x\in\mathbb{Z}$. By definition,
            \begin{align*}
                m &\mid (a-b) \\
                nx &\mid (a-b) \\
                a-b &= nx\cdot y \text{ for some } y\in\mathbb{Z} \\
                a-b &= n(xy) \\
                n &\mid (a-b)
            \end{align*}
            Thus $a\equiv b\pmod n$.
        \end{proof}

        \item Let $p$ be prime. Show that if $x^2\equiv 1\pmod p$, then $x\equiv \pm 1\pmod p$. Find a counterexample when $p$ is not prime.
        \begin{proof}
            First, we have
            \begin{align*}
                p &\mid x^2-1 \\
                p &\mid (x+1)(x-1) 
            \end{align*}
            Given $p$ is prime, $p\mid(x+1)$ or $p\mid(x-1)$. Now, we have
            \begin{align*}
                x-1&\equiv 0\pmod{p} \quad\text{and}\quad x+1\equiv 0\pmod{p} \\
                x&\equiv 1\pmod{p} \quad\text{and}\quad x\equiv -1\pmod{p}
            \end{align*}
            Thus if $x^2\equiv 1\pmod{p}$, then $x\equiv\pm 1\pmod{p}$
        \end{proof}
        A counterexample is $p=8$.
        \begin{align*}
            3^2 &\equiv 1 \pmod{8} \\
            3 &\not\equiv 1 \pmod{8} \\
            3 &\not\equiv -1 \pmod{8}
        \end{align*}

        \item Suppose $a\equiv b\pmod m$. Prove that $\gcd(a,m) = \gcd(b,m)$.
        \begin{proof}
            First, we have $m\mid (a-b)$. Thus $a-b = mx$ for some
            $x\in\mathbb{Z}$. \\
            Suppose the following:
            \begin{enumerate}
                \item $\gcd(a,m) = ax_1+my_1 = z_1$ for some $x_1,y_1\in\mathbb{Z}$
                \begin{align*}
                    z_1 &= ax_1+my_1 \\
                    z_1 &= (mx+b)x_1 + my_1 \\
                    z_1 &= mxx_1+bx_1+my_1 \\
                    z_1 &= b(x_1) + m(xx_1+y_1) 
                \end{align*}
                \item $\gcd(b,m) = bx_2+my_2 = z_2$ for some $x_2,y_2\in\mathbb{Z}$
            \end{enumerate} 
            $z_1$ and $z_2$ are both linear combinations of b and m, so $z_1=z_2$. \\
            Thus $\gcd(a,m) = \gcd(b,m)$.
        \end{proof}
    \end{enumerate}

    \end{enumerate} 

\end{document}