\documentclass[11pt]{article}
%\usepackage[spanish]{babel}
\usepackage[utf8]{inputenc}
\usepackage[OT1]{fontenc}
\usepackage{amsfonts, amsmath, amsthm, amssymb}
\usepackage{mathtools}
\usepackage{graphicx}
\usepackage{listings}
\usepackage[margin=1in]{geometry}
\usepackage{xcolor}

\theoremstyle{definition}
\newtheorem{defn}{Definition}
\newtheorem{thm}{Theorem}
\newtheorem*{thm*}{Theorem}

\usepackage[Glenn]{fncychap}
% Sonny Lenny Glenn Conny Rejne Bjarne Bjornstrup

\newcommand{\ZZ}{\mathbb{Z}}
\newcommand{\NN}{\mathbb{N}}
\newcommand{\QQ}{\mathbb{Q}}

\newcommand{\ord}{\operatorname{ord}}

\DeclarePairedDelimiter\abs{\lvert}{\rvert}


\title{M328K: Homework 5}
\author{Katherine Ho}
\date\today
\begin{document}
\maketitle

\begin{enumerate}
    \item Calculate $\operatorname{lcm}(140,520)$.
    \begin{proof}
        First, \[ \operatorname{lcm}(140,520) = \frac{140\cdot 520}{\gcd(140,520)} \]
        We can use the Euclidean Algorithm to solve for $\gcd(140,520)$.
        \begin{align*}
            520 &= 140(3) + 100 \\
            140 &= 100(1) + 40 \\
            100 &= 40(2) + 20 \\
            40 &= 20(2) + 0             
        \end{align*}
        \begin{center}
            $\gcd(140,520) = 20$
        \end{center}
        Then by substitution,
        \[
            \operatorname{lcm}(140,520) = \frac{140\cdot 520}{\gcd(140,520)} = \frac{72800}{20} = 3640
        \]
    \end{proof}

    \item \begin{enumerate}
        \item Let $m$ and $n$ be relatively prime positive integers. Show that every divisor $d$ of $m n$ can be uniquely written as a product of a divisor $d_1$ of $m$ and a divisor $d_2$ of $n$, where $\gcd(d_1,d_2) = 1$.
        \begin{proof}
            Let $m = (p_1^{a_1})\dots(p_i^{a_i})$ be the prime factorization of $m$. 
            Let $n = (q_1^{b_1})\dots(q_i^{b_i})$ be the prime factorization of $n$.
            Given $m$ and $n$ are relatively prime, $m$ and $n$ do not share any factors.
            Thus each term $p_i$ and $q_i$ is distinct. So, we can say the prime factorization 
            of $mn$ is
            \[ mn = (p_1^{a_1})\dots(p_i^{a_i}) \cdot (q_1^{b_1})\dots(q_i^{b_i}) \]
            If $d$ is a divisor of $mn$, then 
            \[ d = (p_1^{c_1})\dots(p_i^{c_i}) \cdot (q_1^{e_1})\dots(q_i^{e_i}) \]
            where $0\le c_i \le a_i$ and $0\le e_i < b_i$. 
            Let $d_1 = (p_1^{c_1})\dots(p_i^{c_i})$ and let $d_2 = (q_1^{e_1})\dots(q_i^{e_i})$. \\
            $d_1$ is a divisor of $m$ and $d_2$ is a divisor of $n$ since $0\le c_i \le a_i$ and $0\le e_i < b_i$.
            They are coprime since every term $p_i$ and $q_i$ is distinct.
            From this we can see that $d$ is the product of a divisor of $m$ and a divisor of $n$. 
            This product is unique since it is the unique prime factorization of $d$.
            Thus every divisor $d$ of $m n$ can be uniquely written as a product of a 
            divisor $d_1$ of $m$ and a divisor $d_2$ of $n$, where $\gcd(d_1,d_2) = 1$. \\
        \end{proof}
        
        \item Show that if $f$ is a multiplicative function and $F$ is defined by $$F(n) = \sum_{d\mid n} f(d)$$ then $F$ is also multiplicative.

        Hint: Here's a concrete case that illustrates how to use the result from part (a) in the proof.
        \begin{align*}
            F(3\cdot 4) &= \sum_{d\mid 12} f(d) \\
            &= f(1) + f(2) + f(3) + f(4) + f(6) + f(12) \\
            &= f(1\cdot 1) + f(1\cdot 2) + f(3\cdot 1) + f(1\cdot 4) + f(3\cdot 2) + f(3\cdot 4) \\
            &= f(1)f(1) + f(1)f(2) + f(3)f(1) + f(1)f(4) + f(3)f(2) + f(3)f(4) \\
            &= \Big( f(1) + f(3) \Big) \Big(f(1) + f(2) + f(4)\Big) \\
            &= F(3)F(4).
        \end{align*}

        \begin{proof}
            To show that $F$ is multiplicative, we want to show that $F(mn) = F(m)F(n)$.
            \[
                F(mn) = \sum_{d\mid mn} f(d)
            \]
            We can say $d=d_1\cdot d_2$ where $d_1$ is a divisor of $n$ and $d_2$ is a divisor of $m$. 
            And since $f$ is multiplicative, we have
            \begin{align*}
                \sum_{d\mid mn} f(d) &= \sum_{d_1\mid n}\sum_{d_2\mid m} f(d_1)f(d_2) \\
                &= \sum_{d_1\mid n}[f(d_1)\cdot \sum_{d_2\mid m} f(d_2)] \\ 
                &= \sum_{d_1\mid n}[f(d_1)\cdot F(m)] \\
                &=  F(m)\sum_{d_1\mid n}f(d_1) \\
                &=  F(m)F(n)
            \end{align*}
            Now we have
            \[
                F(mn) = F(m)F(n)
            \]
            Thus if $f$ is a multiplicative function and $F$ is defined by 
            $$F(n) = \sum_{d\mid n} f(d)$$ then $F$ is also multiplicative.
        \end{proof}
    \end{enumerate}

    \item Show that $\displaystyle\sum_{d\mid n} \frac{1}{d} = \frac{\sigma(n)}{n}$ for every positive integer $n$.
    
    Hint: If you want to show that two multiplicative functions $f$ and $g$ are the same, it is enough to show that $f(p^k) = g(p^k)$ for prime powers $p^k$.
    \begin{proof}
        \begin{align*}
            \sigma(n) &= \displaystyle\sum_{d\mid n}^{}d \\
            &= \displaystyle\sum_{d\mid n}^{}\frac{n}{d} \\
            &= n\displaystyle\sum_{d\mid n}^{}\frac{1}{d} \\
            \frac{\sigma(n)}{n} &= \displaystyle\sum_{d\mid n}^{}\frac{1}{d}
        \end{align*}
    \end{proof}

    \item Given that $3$ is a primitive root of $43$, find all positive integers $n$ less than 43 such that
    \begin{enumerate}
        \item $\ord(n) = 6$.
        \begin{proof}
            Given that $3$ is a primitive root of $43$, then $\ord_{43}(3) = \phi(43)$.
            Since $43$ is prime, $\phi(43) = 43-1 = 42$. So, $\ord_{43}(3) = 42$. Then,
            \[ 
                \ord_{43}(3^h) = \frac{42}{\gcd(h,42)}
            \]
            Since we want to find $n$ such that $\ord(n)=6$, we need to find the integers $h$ that
            satisfy the following where $n=3^h\pmod{43}$.
            \begin{align*}
                \frac{42}{\gcd(h,42)} &= 6 \\
                7 &= \gcd(h,42) \\
                h &= 7,35
            \end{align*}
            With substitution, $n = 3^7\pmod{43}$ and $n = 3^{35}\pmod{43}$.
            \begin{align*}
                n &= 3^7\pmod{43} = 2187\pmod{43}\equiv 37 \\
                n &= 3^{35}\pmod{43} = (3^{7})^5\pmod{43}\equiv 37^5\pmod{43}\equiv 7
            \end{align*}
            Thus $n = 7,37$.
        \end{proof}
        \item $\ord(n) = 21$.
        \begin{proof}
            Since we want to find $n$ such that $\ord(n)=21$, we need to find the integers $h$ that
            satisfy the following where $n=3^h$.
            \begin{align*}
                \frac{42}{\gcd(h,42)} &= 21 \\
                2 &= \gcd(h,42) \\
                h &= 2,4,8,10,16,20,22,26,32,34,38,40
            \end{align*}
            By substitution, we have
            \begin{align*}
                n &= 3^2\pmod{43}\equiv 9 \\
                n &= 3^4\pmod{43}\equiv 38 \\
                n &= 3^8\pmod{43}\equiv 25 \\
                n &= 3^{10}\pmod{43}\equiv 10 \\
                n &= 3^{16}\pmod{43}\equiv 23 \\
                n &= 3^{20}\pmod{43}\equiv 14 \\
                n &= 3^{22}\pmod{43}\equiv 40 \\
                n &= 3^{26}\pmod{43}\equiv 15 \\
                n &= 3^{32}\pmod{43}\equiv 13 \\
                n &= 3^{34}\pmod{43}\equiv 31 \\
                n &= 3^{38}\pmod{43}\equiv 17 \\
                n &= 3^{40}\pmod{43}\equiv 24
            \end{align*}
            Thus $n=9,10,13,14,15,17,23,24,25,31,38,40$.
        \end{proof}
        
        \item $n$ is a primitive root of $43$.
        \begin{proof}
            Given $3$ is a primitive root, $\ord_{43}(3) = \phi(43) = 42$. Also, 
            \[ 
                \ord_{43}(3^h) = \frac{42}{\gcd(h,42)}
            \]
            $3^h$ is a primitive root $\pmod{43}$ if $\ord_{43}(3^h) = 42$. So, find 
            all values of $h$ such that $\gcd(h,42)=1$. 
            \[ h = 1,5,11,13,17,19,23,25,29,31,37,41 \]
            Now find $3^h\pmod{43}$ for all values of $h$.
            \begin{align*}
                3^1\pmod{43} &= 3 \\
                3^5\pmod{43} &= 28 \\
                3^{11}\pmod{43} &= 30 \\
                3^{13}\pmod{43} &= 12 \\
                3^{17}\pmod{43} &= 26 \\
                3^{19}\pmod{43} &= 19 \\
                3^{23}\pmod{43} &= 34 \\
                3^{25}\pmod{43} &= 5 \\
                3^{29}\pmod{43} &= 18 \\
                3^{31}\pmod{43} &= 33 \\
                3^{37}\pmod{43} &= 20 \\
                3^{41}\pmod{43} &= 29
            \end{align*}
            So, the primitive roots of $43$ are $3,5,12,18,19,20,26,28,29,30,34,33$.
        \end{proof}

    \end{enumerate}

    \item Suppose $n = pq$ is the product of two distinct odd primes. Show that $\ord_n(a)$ divides $\phi(n)/2$ for all $a$ relatively prime to $n$ and hence that there is no ``primitive root of $pq$''.
    \begin{proof}
        First, $ \phi(n) = \phi(p)\phi(q) = (p-1)(q-1) $.
        A positive integer $x$ has a primitive root iff $x = 2,4,p^k$, or $2p^k$. 
        Since $n$ is the product of distinct odd primes $p$ and $q$, $n$ cannot fit the 
        criteria to have a primitive root.

    \end{proof}

\end{enumerate}

\end{document}