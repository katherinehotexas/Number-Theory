\documentclass[letterpaper]{article}

\usepackage{amssymb}
\usepackage{amsthm}
\usepackage{amsmath}
\usepackage[left=1in, right=1in, top=1in, bottom=1in]{geometry}

\newtheorem{lemma}{Lemma}
\newtheorem{sublemma}{Lemma}[lemma]
\newtheorem{theorem}{Theorem}[section]
\newtheorem{corollary}{Corollary}[section]
\newtheorem{definition}{Definition}[section]
\newtheorem{proposition}{Proposition}[section]
\newtheorem{example}{Example}[theorem]

\title{M 328K: Lecture 5}
\author{Katherine Ho}
\date\today

\begin{document}
\maketitle

\section{Modular Congruences}
    Recall: We often use arguments like "n is of the form $4k,4k+1,4k+2,
    \text{or }4k+3\dots$"

    \begin{definition} [Precise]
        Let $a,b,n\in\mathbb{Z}$ and $n>0$. We say that a is congruent to 
        b mod n if $n|(a-b)$. We write
        \[
            a \equiv b \pmod{n}
        \]
    \end{definition}

    \begin{definition} [Informal]
        $a\equiv b$ mod n if a and b give the same remainder after division
        by n. \\
        Examples:
        \begin{itemize}
            \item $7\equiv 2 \pmod 5$
            \item $-31\equiv 11 \pmod 7$
            \item $10^{2024}+1\equiv 1 \pmod 10$
            \item $a\equiv b \pmod 2$ iff a and b are both even or both odd
            \item a can be written in the form 
            \[ a=nk+r \]
                iff $a\equiv r \pmod n$
        \end{itemize}
    \end{definition}

    \begin{proposition}
        Every integer is congruent modulo n to exactly one of $0,1,2,\dots,n-1$
        \begin{proof}
            Let $a\in\mathbb{Z}$. By the division algorithm, we can write
            \[ a=nq+r,\ 0\leq r<n \]
            Then $a-r=nq$, so $n|a-r$, ie.
            \[ a\equiv r \pmod{n} \]
            Uniqueness follows from uniqueness of division algorithm remainder.
        \end{proof}
    \end{proposition}

    \begin{theorem}
        Let $a,b,c\in\mathbb{Z}, n>0$. Then
        \begin{enumerate}
            \item $a\equiv a \pmod{n}$
            \item if $a\equiv b \pmod{n}$ then $b\equiv a \pmod{n}$
            \item if $a\equiv b \pmod{n}$ and $b\equiv c \pmod{n}$, then
            $a\equiv c \pmod{n}$
        \end{enumerate}
        \begin{proof} [Proof (3)]
            By definition, $n|a-b$ and $n|b-c$. Recall that if $n|r, n|s$, 
            then $n|(rx+sy)$ for any $x,y\in\mathbb{Z}$. In particular,
            \[
                n|((a-b)+(b-c)) \Leftrightarrow n|(a-c)
            \]
            So $a\equiv c \pmod{n}$.
        \end{proof}
    \end{theorem}

    \begin{theorem}
        Let $a,b,c,d\in\mathbb{Z}$ and assume $a\equiv b \pmod{n}$.
        \begin{enumerate}
            \item if $c\equiv d \pmod{n}$, then $a+c\equiv b+d\pmod{n}$.
            \item if $c\equiv d \pmod{n}$, then $ac\equiv bd \pmod{n}$.
            \item $a^k\equiv b^k \pmod{n}\ \forall k\in\mathbb{Z}$.
        \end{enumerate}
        \begin{proof} [Proof (1)]
            Suppose $a\equiv b \pmod{n}$ and $c\equiv d \pmod{n}$. By definition,
            $n|a-b$ and $n|c-d$. \\
            But, $(a+c)-(b+d)=(a-b)+(c-d)$ which is divisible by n, so 
            $a+c\equiv b+d \pmod{n}$.
        \end{proof}
        \begin{proof} [Proof (3) by Induction]
            Base case: $k=1$. Tautology \\
            Inductive step: Assume for some $k>1$ that $a^k\equiv b^k \pmod{n}$
            (WTS: $a^{k+1}\equiv b^{k+1}$) \\
            Note by (2) we have 
            \begin{align*}
                a^k & \equiv b^k \pmod{n} & [IH]\\
                a^k\cdot a & \equiv b^k\cdot b \pmod{n} & [2]\\
                a^{k+1} & \equiv b^{k+1} \pmod{n}
            \end{align*}
        \end{proof}

    \end{theorem}
    
    \textbf{WARNING}: In general, if $ac\equiv bc\pmod{n}$, it is not true that 
    $a\equiv b\pmod{n}$. 
    Ex: $2\cdot 3\equiv 2\cdot 0\pmod{6}$

    \begin{example}
        Show $41|(2^{20}-1) \Leftrightarrow$ Show $2^{20}\equiv 1 \pmod{41}$. \\
        First, 
        \begin{align*}
            2^5 & \equiv 32 \pmod{41} \\
            (2^5)^2 & \equiv (-9)^2 \\
            2^{10} & \equiv 81 \pmod{41} \\
            2^{10} & \equiv -1 \pmod{41} \\
            2^{20} & \equiv (-1) \equiv 1 \pmod{41} 
        \end{align*}
    \end{example}

    \begin{proposition}
        A decimal integer is divisible by 3 iff the sum of its digits is 
        divisible by 3.
        \begin{proof}
            Let n be an integer whose decimal representation is 
            \[ (a_na_{n-1}\dots a_1a_0)_{10} \]
            Then 
            \[ a=a_0+a_1\cdot 10+a_2\cdot 100+\dots+a_n\cdot 10^n \]
            Then
            \[ a=a_0+a_1\cdot 10+\dots+a_n\cdot 10^n \pmod{n}\]
            Since $10\mod 3\equiv 1$, we have
            \[ a\equiv a_0+a_1+\dots+a_n\pmod{3} \]
        \end{proof}
    \end{proposition}

\section{Congruences with Unknowns}
    \begin{example}
        Solve
        \begin{align*}
            x+12 &\equiv 5\pmod{8} \\
            x &\equiv -7\pmod{8}
        \end{align*}
        We also have
        \begin{itemize}
            \item $x\equiv 1\pmod{8}$ is also a solution
            \item $x\equiv 9$
            \item $x\equiv 17$
        \end{itemize}
        But we consider these to be the "same" since they are congruent.
    \end{example}

    \begin{example}
        Solve
        \begin{align*}
            4x &\equiv 3 \pmod{19} \\
            20x &\equiv 15 \pmod{19} \\
            x &\equiv 15 \pmod{19} \\
            \text{Since } 20 &\equiv 1 \pmod{19}
        \end{align*}
    \end{example}

    \begin{example}
        Solve
        \[
            6x\equiv 15 \pmod{514}
        \]
        This has no solutions. \\
        Why?! $6x-15$ is always odd. \\
        In particular, $514\nmid (6x-15)$. \\
        In general, we want to understand when $ax\equiv b$ has solutions 
        and how to find them.
    \end{example}

    \begin{example}
        $18x\equiv 8 \pmod{22}$ has incongruent solutions \\
        $x\equiv 20 \pmod{22}$ and $x\equiv a\pmod{22}$
    \end{example}

\end{document}
