\documentclass[letterpaper]{article}

\usepackage{amssymb}
\usepackage{amsthm}
\usepackage{amsmath}
\usepackage[left=1in, right=1in, top=1in, bottom=1in]{geometry}

\newcommand{\ZZ}{\mathbb{Z}}
\newcommand{\NN}{\mathbb{N}}
\newcommand{\QQ}{\mathbb{Q}}

\newtheorem{lemma}{Lemma}
\newtheorem{sublemma}{Lemma}[lemma]
\newtheorem{theorem}{Theorem}[section]
\newtheorem{corollary}{Corollary}[section]
\newtheorem{definition}{Definition}[section]
\newtheorem{proposition}{Proposition}[section]
\newtheorem{example}{Example}[theorem]

\title{M 328K: Lecture 6}
\author{Katherine Ho}
\date\today

\begin{document}
\maketitle

\section{From Last Time}
    Solve $ax \equiv b \pmod{n}$. \\
    It's possible for there to be no solutions OR a single solution OR 
    multiple incongruent solutions.
    \begin{theorem}
        \begin{enumerate}
            \item $a\equiv a \pmod{n}$
            \item if $a\equiv b \pmod{n}$ then $b\equiv a \pmod{n}$
            \item if $a\equiv b \pmod{n}$, $b\equiv c \pmod{n}$, then
            $a\equiv c \pmod{n}$
        \end{enumerate}
        \begin{example} $20 \equiv 1 \pmod{19}$
            \begin{align*}
                20 & \equiv 1 \pmod{19} \\
                20x & \equiv x \pmod{19} \\
                20x & \equiv 15 \pmod{19} & \text{We also have this}\\
                x & \equiv 20x \pmod{19} & \text{By (2)} \\
                x & \equiv 15 \pmod{19} & \text{By (3)}
            \end{align*}
        \end{example}
    \end{theorem}

\section{Solving stuff}
    \textbf{WARNING}: If $ac\equiv bc\pmod{n}$, we can't conclude $a\equiv b \pmod{n}$.
    \begin{theorem}
        If $\gcd(c,n)=1$, then $ac\equiv bc\pmod{n}$ implies $a\equiv b\pmod{n}$.
        \begin{proof}
            By definition, we  have
            \[ 
                n \mid (a-b)c 
            \] 
            By Euclid's Lemma, since $\gcd(n,c)=1$, we have
            $n\mid (a-b)$, hence $a\equiv b\pmod{n}$.
        \end{proof}
    \end{theorem}
    \begin{proposition}
        Let $d=\gcd(a,b)$ for some $a,b\in\ZZ$. 
        Then $\gcd(\frac{a}{d}, \frac{b}{d})=1$. 
        \begin{proof}
            By Bezout, there exist integers $x$ and $y$ such that $ax+by=d$.
            Then, \[ (\frac{a}{d}x+\frac{b}{d}y)=1 \]
            So $\frac{a}{d}, \frac{b}{d}$ are relatively prime.
        \end{proof}
    \end{proposition}
    \begin{theorem}
        Consider $ac\equiv bc \pmod{n}$ and let $d=\gcd(c,n)$. 
        Then $a\equiv b\pmod{\frac{n}{d}}$. \\
        \underline{Note}: If $d=1$, this is the same statement as before.
        \begin{proof}
            $n\mid (a-b)c$ as before. So there exists $k\in\ZZ$ such that
            $(a-b)c=nk$. Then,
            \[
                (a-b)\frac{c}{d} = \frac{n}{d}k 
            \]
            So, 
            \[ 
                \frac{n}{d}\mid (a-b)\frac{c}{d} 
            \]
            By Proposition 2.1, $\gcd(\frac{n}{d}, \frac{c}{d})=1$, so 
            Euclid's Lemma says 
            \[ 
                \frac{n}{d}\mid (a-b) \text{, ie. } a\equiv b\pmod{\frac{n}{d}}
            \]
        \end{proof}
        \begin{example}
            \begin{align*}
                2\cdot 3 &\equiv 2\cdot 0 \pmod{6} & \gcd(2,6)=2 \\
                3 &\equiv 0 \pmod{3}
            \end{align*}
        \end{example}
    \end{theorem}
    \begin{theorem} [Build-a-theorem]
        Let $a,b,n\in\ZZ$ with $n>1$, let $d=\gcd(a,n)$.
        Then the linear congruence $ax\equiv b\pmod{n}$. 
        \begin{itemize}
            \item has no solution if $d\nmid b$
            \item has exactly d incongruent solutions $\pmod{n}$ if $d\mid b$
        \end{itemize}
        In particular, if $x_0$ is a solution, then
        \[ 
            x_0, x_0+\frac{n}{d}, x_0+2\frac{n}{d},\dots, x_0+(d-1)\frac{n}{d}
        \]
        is a complete set of solutions $\pmod{n}$, ie. 
        if x is a solution, then x is congruent modulo n to exactly one of 
        \[
            x_0+t(\frac{n}{d}) \text{ for } 0\leq t\leq d-1
        \]
        Study $ax\equiv b\pmod{n}$. If this has a solution x, then
        $n\mid(ax-b)$. Then there exists $y\in\ZZ$ such that
        \[ ax-b=ny \]
        So, 
        \[ ax-ny=b \]
        This linear diophantine equation has a solution exactly when 
        $\gcd(a,n)=d\mid b$. \\\\
        \underline{Recall}: $6x\equiv 15\pmod{512}$.
        $\gcd(6,512)=(1,2,3,\text{or } 6)$. Note $3\nmid 512$ since 
        $3+(5+1+2)$. \\ But $2\nmid 15$, so there are no solutions.

        \begin{example}
            Solve \[ 9x\equiv 21\pmod{30} \]
            $d=\gcd(9,30)=3\mid 21$
            Either write down
            \[ 9x-30y=21 \]
            dividing,
            \[ 3x-10y=7 \]
            OR apply Theorem 2.2 to yield 
            \[ 3x\equiv 7\pmod{10} \] 
            leading to 
            \[ 3x-10y=7 \]
            \underline{Extended Euclidean algorithm}
            \begin{align*}
                10 &= 3\cdot 3+1 \\
                10-3\cdot 3 &= 1 \\
                10\cdot 7 - 3\cdot 21 &= 7 \\
                -10(-7)+3(-21) &= 7
            \end{align*}
            \begin{center}
                \boxed{$x=-21$, $y=-7$}
            \end{center}
            But $x\equiv (-21)+30\pmod{30}$. $x\equiv 9\pmod{30}$. 
            So we have found one solution (up to congruence). \\
            \underline{Note}: $x=9$ is a solution to $3x\equiv 7\pmod{10}$. 
            So, $x=19$ and $x=29$ are also soolutions to $3x\equiv 7\pmod{10}$ 
            that are distrinct $\pmod{30}$.
        \end{example}

        \begin{example}
            Solve \[ 18x\equiv 8\pmod{22} \]
            $d=\gcd(18,22)=2$. First find a solution to
            \[ 9x\equiv 4\pmod{11} \]
            Solve 
            \[ 9x-11y = 4\]
            this has a solution $x=-2$, $y=-22$. \\
            Choose $x=-2+11 = 9$ is one solution. \\
            The other distinct solution $\pmod{22}$ is 
            \[ x=9+11=20 \] 
            $x=9,20$ is a complete set of solutions up to congruence $\pmod{22}$.

        \end{example}

    \end{theorem}

\end{document}
