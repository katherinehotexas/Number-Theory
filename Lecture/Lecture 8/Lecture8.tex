\documentclass[letterpaper]{article}

\usepackage{amssymb}
\usepackage{amsthm}
\usepackage{amsmath}
\usepackage[left=1in, right=1in, top=1in, bottom=1in]{geometry}

\newcommand{\ZZ}{\mathbb{Z}}
\newcommand{\NN}{\mathbb{N}}
\newcommand{\QQ}{\mathbb{Q}}

\newtheorem{lemma}{Lemma}
\newtheorem{sublemma}{Lemma}[lemma]
\newtheorem{theorem}{Theorem}[subsection]
\newtheorem{corollary}{Corollary}[subsection]
\newtheorem{definition}{Definition}[subsection]
\newtheorem{proposition}{Proposition}[subsection]
\newtheorem{example}{Example}[subsection]

\title{M 328K: Lecture 8}
\author{Katherine Ho}
\date\today

\begin{document}
\maketitle

\section{Last Time}
    \subsection{Fermat's Little Theorem}
        Let $p$ be prime, $a\in\ZZ$, $p\nmid a$, then 
        \[ a^{p-1}\equiv 1\pmod{p} \]
        \[ ax\equiv 1\pmod{n} \quad\text{has a solution whenever}\quad \gcd(a,n)=1 \]
        \begin{align*}
            4x &\equiv 3\pmod{19} \\
            4^{17}(4x) &\equiv 4^{17}\cdot 3\pmod{19} \\
            4^{18}x &\equiv 5\cdot 3\pmod{19} \\
            x &\equiv 15\pmod{19}
        \end{align*}
        \underline{Note}: Definitely need $p$ to be prime.
    \begin{example}
        \[ 3^9\equiv 3\pmod{10} \]
    \end{example}

\section{Generalization to composite modulus}
    \subsection{Euler Totient Function (Euler's Phi Function)}
    \begin{definition}
        The Euler totient function $\phi$ is 
        the function $\phi$ $\NN\rightarrow\NN$ defined by 
        \[ 
            \phi(n) = \#\{ a\mid 1\le a\le n-1, \gcd(a,n)=1 \} 
        \]
        \begin{example}
            \begin{align*}
                \phi(1) &= 1 \\
                \phi(2) &= 1 \\
                \phi(3) &= 2 \\
                \phi(4) &= 2 \\
                \phi(20) &= 8
            \end{align*}
        \end{example}

        \begin{proposition}
            If $p$ is prime, then
            \[ \phi(p) = p-1 \]
        \end{proposition}

        \begin{proposition}
            If $p$ is prime and $k>1$, then
            \[ \phi(p^k) = p^k-p^{k-1} \]
            Exclude all multiples of $p$ between 1 and $p^k$:
            \[ p,2p,3p,\dots,(p^{k-1})p,p^{k-1}p \]
        \end{proposition}

        \underline{Note}: $\phi(n)=n-1$ iff $n$ is prime.
        Intuition: $\phi$ is how close $n$ is to being prime.
    \end{definition}

    \subsection{Euler's Theorem}
    \begin{theorem} [Euler's Theorem]
        Let $\gcd(a,n)=1$. Then 
        \[ a^{\phi(n)}\equiv 1\pmod{n} \]
        Note: If $n=p$ is prime, then $\phi(n)=p-1$, so we get 
        \[ a^{p-1}\equiv 1\pmod{p} \]
        \begin{proof} [Proof of Euler's Theorem]
            Let $0<b_1<b_2<\dots<b_{\phi(n)}$ be the integers between 
            1 and $n$ that are coprime to $n$. The claim: The integers 
            $ab_1, ab_2, \dots, ab_{\phi(n)}$ are the same as 
            $b_1,b_2,\dots,b_{\phi(n)}\pmod{n}$ but maybe in a different order.
            \begin{example}
                $n=10$; $a-3$
                \[
                \begin{matrix}
                    b_1 & b_2 & b_3 & b_4 \\
                    1 & 3 & 7 & 9 \\
                    ab_1 & ab_2 & ab_3 & ab_4 & \pmod{10} \\
                    3 & 9 & 1 & 7
                \end{matrix}
                \]
                Proof is same from HW. \\
                So
                \begin{align*}
                    (ab_1)(ab_2) &\equiv b_1b_2\dots b_{\phi(n)} \pmod{n} \\
                    a^{\phi(n)}(b_1b_2\dots b_{\phi(n)}) &\equiv b_1b_2\dots b_{\phi(n)}
                \end{align*}
                Since each $b_i$ is coprime to $n$, we can cancel to get 
                \[ a^{\phi(n)}\equiv 1\pmod{n} \]
            \end{example}
        \end{proof}
    \end{theorem}

    \subsection{More on $\phi$}
        \begin{align*}
            \phi(p) &= p-1 \quad\text{for } p \text{ prime} \\
            \phi(p^k) &= p^k - p^{k-1} 
        \end{align*}

        \begin{theorem}
            Let $a,b$ be coprime positive integers. Then,
            \[ \phi(a,b)=\phi(a)\cdot\phi(b) \]
            "$\phi$ is multiplicative." \\\\
            \textbf{WARNING}: We need $\gcd(a,b)=1$.
            Ex. $\phi(4)=2$, $\phi(2)\phi(2)=1$
        \end{theorem}

        \begin{corollary}
            If $n=p_1^{r_1}\dots p_k^{r_k}$, then
            \[ 
                \phi(n) = \phi(p_1^{r_1})\dots\phi(p_k^{r_k})
                = (p^{r_1}-p^{r_{k-1}})\dots(p^{r_k}-p^{r_{k-1}})
            \]
        \end{corollary}

        To prove this, we first need to understand how to solve this 
        problem from 4th century China: 
        \begin{align*}
            x &\equiv 2\pmod{3} \\
            x &\equiv 3\pmod{5} \\
            x &\equiv 2\pmod{7}
        \end{align*}
        We will solve this using the \underline{Chinese Remainder Theorem}.

        \begin{theorem} [Chinese Remainder Theorem]
            Suppose $\gcd(n_1,n_2)=1$ for pos integers $n_1$ and $n_2$. 
            Then for any $a_1,a_2\in\ZZ$, the system
            \begin{align*}
                x\equiv a_1\pmod{n_1} \\
                x\equiv a_2\pmod{n_2}
            \end{align*} 
            has a unique solution $0\le x<n_1n_2$.

            \begin{proof} [Proof (Existence)]
                By Bezout, there exist $m_1,m_2\in\ZZ$ such that
                \[ n_1m_1+n_2m_2 = 1 \]
                Now let $x=a_2n_1m_1+a_1n_2m_2$.
                Then reducing $\pmod{n_1}$, we have
                \begin{align*}
                    x=a_2n_1m_1+a_1n_2m_2 &\equiv a_1n_2m_2\pmod{n_1} \\
                    &\equiv a_1(1-n_1m_1)\pmod{n-1} \\
                    &\equiv a_1-a_1n_1m_1\pmod{n-1} \\
                    &\equiv a_1\pmod{n_1}
                \end{align*}
                By the same argument, 
                \[ x\equiv a_2\pmod{n_2} \]
                Take $x\pmod{n_1n_2}$ to be a solution between 0 and $n_1n_2$.
            \end{proof}
        \end{theorem}

        \begin{example} 
            Going back to this problem, 
            \begin{align*}
                x &\equiv 2\pmod{3} \\
                x &\equiv 3\pmod{5} \\
                x &\equiv 2\pmod{7}
            \end{align*}
            First use Bezout:
            \begin{align*}
                3\cdot 2+5(-1)=1 \\
                x = 3(6)+2(-5)\pmod{15} = 8 \\
            \end{align*} 
            \begin{align*}
                x\equiv 8\pmod{15} \\
                x\equiv 2\pmod{7} \\
                15\cdot 1 + 7(-2) = 1 \\
                x = 2(15) + 8(-14)\pmod{105} \\
                -82\pmod{105} = 23
            \end{align*}
        \end{example}   

        \underline{Relationship with $\phi$}: To show
        \[ \phi(ab) = \phi(a)\phi(b) \]
        when $\gcd(a,b)=1$, we need to count two things: 
        \[ 
            \{ x\mid 0\le x<ab, \gcd(x,ab)=1 \} 
        \]
        \begin{center}Size: $\phi(ab)$\end{center}
        \[ 
            \{ (y_1,y_2)\mid 0\le y_1 < a, \gcd(y_1,a)=1,
            0\le y_2<b, \gcd(y_2,b)=1 \} 
        \]
        \begin{center}Size: $\phi(a)\phi(b)$\end{center}

\end{document}