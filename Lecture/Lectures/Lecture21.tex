\chapter{Lecture 21}
\date{November 12, 2024}

\section{Midterm 2}
    \subsection{Question 1}
    \begin{enumerate}
        \item $g$ prim root of $p$, $d\nmid p-1 \longrightarrow g^d$ prim root $\gcd(d,p-1) = 1$. FALSE
        \item if $\exists a, 1\le a\le n-1$ s.t.
            \[
                a^{\frac{n-1}{2}}\ne \pm 1 \pmod{n}
            \]
            then $n$ is composite. TRUE
        \item If $\gcd(a,n) = 1$, then $x^2\equiv a\pmod{n}$
            has $e$, then 0 or 2 incongruent solutions. FALSE \\
            Example: $x^2\equiv 1\pmod{8}, x\equiv 1,3,5,7$
        \item If $\legendre{a}{n} = -1$, then $a$ is a NR of $n$. TRUE
    \end{enumerate}

    \subsection{Congruence solutions for $\legendre{3}{p}$}
    \[
        \legendre{3}{p} = \\
        \begin{cases}
            -\legendre{p}{3} \text{ if } p\equiv 3\pmod{4} \\
            \legendre{p}{3} \text{ if } p\equiv 1\pmod{4} 
        \end{cases}
    \]
    \begin{enumerate}
        \item if $p\equiv 1\pmod{4}$, then 
        \[
            \legendre{3}{p} = \legendre{p}{3} = \\
            \begin{cases}
                1 \text{ if } p\equiv 1\pmod{3} \\
                -1 \text{ if } p\equiv 2\pmod{3}
            \end{cases}
        \]
        \item if $p\equiv 3\pmod{4}$, then 
        \[
            \legendre{3}{p} = -\legendre{p}{3} = \\
            \begin{cases}
                1 \text{ if } p\equiv 2\pmod{3} \\
                -1 \text{ if } p\equiv 1\pmod{3}
            \end{cases}
        \]
    \end{enumerate}
    \[
        \legendre{3}{p} = \\
        \begin{cases}
            1 \\
            -1 
        \end{cases}
    \]
    
    \subsection{$p,q=2p+1$ odd primes}
    WTS: -4 is a prime root of q. 
    \[
        \ord(-4) \mid (q-1) = 2p \longrightarrow \ord(-4) = 1, 2, p, \text{ or } 2p
    \]
    Rule out $\ord(-4) = p$. 
    Compute $(-4)^p = -4^{\frac{q-1}{2}} \equiv \legendre{-4}{q}\pmod{q}$.
    \[
        \legendre{-4}{q} = \legendre{-1}{q}\legendre{4}{q} = \legendre{-1}{q}
    \]
    So if $\ord(-4) = p$, then $\legendre{-1}{q} = 1$, so $q\equiv 1\pmod{4}$.
    But $q\equiv 3\pmod{4}$ since $q=2p+1$, Sophie Germain

    \subsection{}
    Let p be an odd prime, $(p-1)\nmid n$. 
    Show $1^n+2^n+\dots+(p-1)^4\equiv 0\pmod{p}$. 
    g = prim root. 
    \[
        g,g^2,\dots,g^{p-1}\equiv 1,2,\dots,p-1
    \]
    in some order.
    \[
        \longrightarrow 1^n+\dots+(p-1)^n \equiv g^n + g^{2n} + \dots + g^{(p-1)n} \pmod{p}
    \]

    \begin{align*}
        (g^n - 1)(g^{n(p-1)} + \dots + g^n + 1) &= g^{np-1} \\
        g^{n(p-1)} + \dots + g^n &= \frac{g^{np} - 1}{g^n - 1} - 1 \\
        &\equiv 0
    \end{align*}

\section{Cryptography Stuff}
\subsection{Remote Coin Flipping}
    Instead of H/T, we will use roots of $x^2\equiv a\pmod{n}$ where $n=pq$.

    Procedure: 
    \begin{enumerate}
        \item Alice chooses 2 odd primes $p,q (p\equiv q\equiv 3\pmod{4})$ 
            and computes $n=pq$ and tells Bob $n$. 
        \item Bob choose randomly some $1\le x\le n-1$, compute $a=x^2\pmod{n}$
            and tell Alice $a$. 
        \item Alice computes the square roots of $a\pmod{n}$, $\pm x_1$, $\pm x_2$
            Choose either $\pm x_1$ or $\pm x_2$ (Heads or Tails), 
            tell Bob $\pm x_1$ or $\pm x_2$.
        \item If Bob's $x$ is different from Alice's then, Bob can factor $n$. 
    \end{enumerate}

    \begin{align*}
        x^2\equiv 324\pmod{391}&, 391 = 17.23 \\
        x^2\equiv 324\equiv 1\pmod{17} &,\quad x^2\equiv 324\equiv 2\pmod{23} \\
        x\equiv \pm 1\pmod{17} &,\quad x^2\equiv 2\pmod{23}
    \end{align*}

    If $p\equiv 3\pmod{4}$ and $a$ is QR of $p$, then $x=a^{\frac{p+1}{4}}$ is a solution to $x\equiv a\pmod{p}$
    
    \begin{proof}
        $x^2=(a^{\frac{p+1}{4}})^2 = a^{\frac{p+1}{2}} = a\cdot a^{\frac{p-1}{2}} \equiv a\cdot 1 \equiv a\pmod{p}$ \\
        $x = 2^{\frac{23+1}{4}} = 2^6 = 64 \equiv -5$ \\

        Solutions are $x\equiv \pm 5\pmod{23}$. 
        $\rightarrow$ 4 systems. 
        \begin{align*}
            x\equiv 1\pmod{17},\quad x\equiv 5\pmod{23} \rightarrow x_1\pmod{391} \\
            x\equiv 1\pmod{17},\quad x\equiv -5\pmod{23} \rightarrow x_2\pmod{391} \\
            x\equiv -1\pmod{17},\quad x\equiv 5\pmod{23} \rightarrow -x_2\pmod{391} \\
            x\equiv -1\pmod{17},\quad x\equiv -5\pmod{23} \rightarrow x_1\pmod{391} 
        \end{align*}
    \end{proof}

    Back to (4), How does Bob factor n when he has knowledge of all 4 roots 
    $\pm x_1, \pm x_2$ of $a$?
    Idea: 
    \begin{align*}
        x_1^2\equiv a\equiv x_2^2\pmod{pq} \\
        \rightarrow pq\mid x_1^2 - x_2^2 = (x_1 - x_2)(x_1 + x_2) \\
        p\mid (x_1 - x_2)(WLOG)
    \end{align*}
    Then $q\nmid (x_1-x_2)$, $pq=n\mid (x_1-x_2)$  so $x_1\equiv x_2\pmod{n}$. 

    $\rightarrow$ Bob computes $\gcd(x_1 - x_2, n) = p$ or $q$. 