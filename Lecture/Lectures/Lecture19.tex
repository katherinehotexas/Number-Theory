\chapter{Lecture 19}
\date{October 31, 2024}

\section{Exam Review}
    \subsection{HW7 Q4}
    Show that $\legendre{5}{p}=1$ iff $p\equiv 1,9,11,19\pmod{20}$. \\
    Since $5\equiv 1\pmod{4}$
    \[
        \legendre{5}{p} = \legendre{p}{5} = 1 \quad\text{where P is QR of 5}
    \]
    \begin{align*}
        1^2 &= 1 \\
        2^2 &= 4 \\
        3^2 &= 9 \equiv 4 \\
        4^2 &= 16 \equiv 1
    \end{align*}
    So, 
    \[
        \legendre{5}{1} = 1 \text{ iff } p\equiv 1,4\pmod{5}
    \]

    \subsection{Determine congruence conditions for $\legendre{-5}{p}=1$}
    \[
        \legendre{-5}{p} = \legendre{-1}{p}\legendre{5}{p} = \\
        \begin{cases}
            1 \text{ whenever } \legendre{-1}{p}=\legendre{5}{p}=1 \text{ or } \legendre{-1}{p} = \legendre{5}{p} = 1
        \end{cases}
    \]
    \[
        \legendre{-1}{p} = \\
        \begin{cases}
            1 \text{ when } p\equiv 1\pmod{4} \\
            -1 \text{ when } p\equiv 3\pmod{4}
        \end{cases}
    \]
    \[
        \legendre{5}{p} = \legendre{p}{5} \\
        \begin{cases}
            1 \text{ when } p\equiv 1,4\pmod{5} \\
            -1 \text{ when } p\equiv 2,3\pmod{5}
        \end{cases}
    \]
    Hence we have $\legendre{-1}{p} = \legendre{5}{p} = 1$ iff 
    \[
        (p\equiv 1\pmod{4}) \text{ AND } (p\equiv 1\pmod{5} \text{ or } p\equiv 4\pmod{5})
    \]
    Equivalently, 
    \[
        p\equiv 1\pmod{4}, p\equiv 1\pmod{5} \quad\text{ OR }\quad p\equiv 1\pmod{4}, p\equiv 4\pmod{5}
    \]

    Using Chinese Remainder Theorem, 
    \[
        p\equiv 1\pmod{20} \quad\text{ OR }\quad p\equiv 9\pmod{20}
    \]

    On the other hand, we have $\legendre{-1}{p} = \legendre{5}{p} = -1$ iff
    \begin{align*}
        p\equiv 3\pmod{4} \quad\text{OR}&\quad p\equiv 3\pmod{4} \\
        p\equiv 2\pmod{5} \quad&\quad  p\equiv 3\pmod{5} \\
        \Longleftrightarrow\quad & \quad\Longleftrightarrow \\
        p\equiv 7\pmod{20} \quad&\quad p\equiv 3\pmod{20}
    \end{align*}
    So,
    \[
        \legendre{-5}{p} = 1 \text{ iff } p\equiv 1,3,7,9\pmod{20}
    \]

\section{Last Time: Complex Numbers}
    \subsection{Gaussian Integers}
    \[
        \ZZ[i] = \{ a+bi\mid a,b\in\ZZ \}
    \]
    We saw that 2 is not "prime" in $\ZZ[i]$ since $2=(1+i)(1-i)$. 
    But what does it mean to be prime in $\ZZ[i]$?

    $3 = (3i)(-i)$, so is 3 "composite" in $\ZZ[i]$? \\
    \underline{Idea}: This isn't a "real" factorization, just like $3 = (-3)(-1)$. \\

    Why/how do we exclude $\pm i$?
    Are there other elements of $\ZZ[i]$ we should exclude from factorization?

    \underline{Answer}: Only need to exclude $1,-1,i,-i$.

    For each $a\in\{ 1,-1,i,-i \}, \exists b\in\ZZ[i]$ such that $ab=1$.
    Ex: $(-1)(-1) = 1, (i)(-i) = 1$ \\

    \section{Units}
    \begin{definition}
        A Gaussian integer $z$ is called a \underline{unit} if there exists some $w\in\ZZ[i]$
        such that 
        \[
            zw=1
        \]
    \end{definition}

    \begin{theorem}
        The only units in $\ZZ[i]$ are $1,-1,i,-i$.
    \end{theorem}

    Use geometry of $\CC$ to answer.

    Recall: Multiplication has a geometric meaning in polar coordinates
    \[
        z=a+bi \rightarrow (a,b) \leftrightarrow (r,\theta)
    \]
    \[
        zw \leftrightarrow (r_1, \theta_1)(r_2, \theta_2) = (r_1r_2, \theta_1 + \theta_2)
    \]

    $z=a+bi$ has polar coords $(r,\theta)$. Then $r\sqrt{a^2+b^2}$.
    We can interpret $r$ as an absolute value of $\CC$. 
    The fact that multiplication works geometrically like this means $|zw| = |z||w|$
    where $|a+bi| = \sqrt{a^2+b^2}$.

    \begin{definition}
        For $z\in\ZZ[i]$, define the \underline{norm} of $z$. 
        \[
            N(z) = |z|^2 = a^2 + b^2 \quad\text{if } z=a+bi
        \]
        Note: $N(zw) = |zw|^2=|z|^2|w|^2=N(z)N(w)$
    \end{definition}

    Let $z=a+bi, w = c+di$. then
    \begin{align*}
        zw &= (a+bi)(c+di) \\
        &= (ac-bd) + (ad+bc)i
    \end{align*}
    Hence $N(zw) = (ac-bd)^2 + (ad+bc)^2$. 
    On the other hand, $N(z)N(w) = (a^2+b^2)(c^2+d^2)$.
    We obtain the identity:
    \begin{theorem}
        For any $a,b,c,d\in\RR$, we have
        \[
            (a^2+b^2)(c^2+d^2) = (ac-bd)^2 + (ad+bc)^2
        \]
    \end{theorem}

    \subsection{Back to units}
    Suppose $u$ is a unit. Then there exists a unit $v$ such that 
    \[
        uv=1
    \]
    Then 
    \[
        N(u)N(v) = N(1) = 1
    \]
    Hence $N(u)$ and $N(v) = 1$. 
    If $u=a+bi$ is a unit, then $a^2+b^2 = 1$. \\
    Solutions are $(a,b) = (1,0), (-1,0), (0,1), (0,-1)$. \\
    Each correspond to 
    \begin{align*}
        (1,0) &\rightarrow 1 + 0i = 1 \\
        (-1,0) &\rightarrow -1 + 0i = -1 \\
        (0,1) &\rightarrow 0 + i = i \\
        (0,-1) &\rightarrow o - i = -i
    \end{align*}
    So these are all the units. Unit circle.

\section{Sum of 2 Squares}
    To answer which primes in $\ZZ$ are still prime in $\ZZ[i]$,
    we need to first answer the following: 

    \underline{Q}: Which primes can be written as a sum of two squares?
    \begin{align*}
        p &= 3 \\
        &= 5 = 1^2+2^2 \\
        &= 7 \\
        &= 11 \\
        &= 13 = 2^2 + 3^2 \\
        &= 17 = 1^2 + 4^2 \\
        &= 19 \\
        &= 23
    \end{align*}
    \begin{theorem}
        If $p$ is an odd prime and the sum of 2 squares, 
        then $p\equiv 1\pmod{4}$.
        \begin{proof}
            Suppose $p=a^2+b^2$. then
            \begin{align*}
                a^2+b^2 &\equiv 0\pmod{p} \\
                a^2 &\equiv -b^2\pmod{p}
            \end{align*}
            Thus
            \begin{align*}
                \legendre{a^2}{p} &= \legendre{-b^2}{p} \\
                1 &= \legendre{-1}{p}\legendre{b^2}{p} = \legendre{-1}{p}\cdot 1
            \end{align*}
            Thus, $\legendre{-1}{p} = 1$ so $p\equiv 1\pmod{4}$.
        \end{proof}
    \end{theorem}

    In fact: 
    \begin{theorem}
        An odd prime $p$ is the sum of two swuares iff 
        $p\equiv 1\pmod{4}$.
        \begin{proof} [Proof (Fermat)]
            Let $p\equiv 1\pmod{4}$. then
            \[
                \legendre{-1}{p} = 1
            \]
            So there exists $a\in\ZZ$ such that $a^\equiv -1\pmod{p}$. 
            Hence $a^2+1=Mp$ for some $M\in\ZZ$. 
        \end{proof}
        \begin{lemma} [Fermat]
            If $Mp, M\ge 2$ can be written as a sum of two squares,
            then there exists $1\le m< M$ such that $mp$ can be written as a sum of two squares.
            \begin{example}
                $p=881$
                \begin{align*}
                    387^2 + 1^2 = 170\cdot 881 \qquad (M=170) \\
                \end{align*}
                Reduce $\pmod{M}$ to lie in $\{\frac{-M}{2}, \frac{M}{2}\}$
                \begin{align*}
                    387 &\equiv 47\pmod{170} \\
                    1 &\equiv 1\pmod{170}
                \end{align*}
                Then 
                \begin{align*}
                    387^2 + 1^2 &\equiv 0 \pmod{170} \\
                    47^2 + 1^2  &\equiv 0 \pmod{170}
                \end{align*}
                Note: $(u^2+v^2)(A^2+B^2) = (uA+vB)^2+(vA-uB)^2$. \\
                Multiply $387^2 + 1^2$ and $47^2 + 1^2$ to get 
                \[
                    (387^2 + 1^2)(47^2 + 1^2) = (47\cdot 387 + 1\cdot 1)^2 + (1\cdot 387 - 47\cdot 1)^2 
                    = (18190)^2 + (340)^2
                \]
                But also 
                \begin{align*}
                    387^2 + 1^2 = 170\cdot 881 \\
                    47^2 + 1^2 = 170\cdot 13
                \end{align*}
                So 
                \begin{align*}
                    170^2\cdot 13\cdot 881 &= 18190^2 + 340^2 \\
                    13\cdot 881 &= 107^2 + 2^2
                \end{align*}
                Keep doing this process and eventually you can write 881 as a sum of 2 squares.
            \end{example}
        \end{lemma}
    \end{theorem}