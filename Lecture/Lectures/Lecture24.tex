\chapter{Lecture 24}
\date{November 21, 2024}

\section{Recall: Continued Fractions}
    pi example:
    \begin{align*}
        \pi &= [3;7,15,1,292,\dots] \\
        C_0 &= 3, C1 = \frac{22}{7}, \\
        C_2 &= \frac{333}{106}, C_3 = \frac{355}{113}, \\
        C_4 &= \frac{103993}{33102}
    \end{align*}

    Continued Fractions
    \begin{align*}
        [a_0; a_1, a_2, \dots] = a_0 + \frac{1}{a_1 + \frac{1}{a_2 + \dots}}
    \end{align*}

    nth convergent:
    \[
        C_n = [a_0; a_1, a_2, \dots, a_n]
    \]

    \begin{theorem}
        \begin{align*}
            p_0 &= a_0, q_0 = 1, \\
            p_1 &= a_1a_0+1, q_1 = a_1, \\
            p_n &= a_np_{n_1}+p_{n-2}, q_n=a_nq_{n-1}+q_{n-2}
        \end{align*}
        Then,
        \[
            C_n = \frac{p_n}{q_n}
        \]
    \end{theorem}

    \underline{Ex}: $\quad e = [2;1,2,1,1,4,1,1,6,1,\dots]$
    \begin{align*}
        C_0 &= \frac{p_0}{q_0} = \frac{a_0}{1} = 2 \\
        C_1 &= \frac{p_1}{q_1} = \frac{a_1a_0+1}{a_1} = \frac{2\cdot 2 + 1}{1} = \frac{3}{1} \\
        C_2 &= \frac{2\cdot p_1 + p_0}{2\cdot q_1 + q_0} = \frac{2\cdot 3 + 2}{2\cdot 1 + 1} = \frac{8}{3} \\
        C_3 &= \frac{1\cdot 8 + 3}{1\cdot 3 + 1} = \frac{11}{4}
    \end{align*}
        
    \begin{theorem}
        \[
            p_kq_{k-1} - q_kp_{k-1} = (-1)^{k-1} \quad (p_{k+1}q_k - q_{k+1}p_k = (-1)^k)
        \]
        \begin{proof}
            Base case: $k=1$
            \[
                p_1q_0 = q_1p_0 = (a_1a_0+1)(1)-(a_1)(a_0) = 1 = (-1)^0
            \]
            Now assume for induction that 
            \[
                p_mq_{m-1} - q_mp_{m-1} = (-1)^{m-1}
            \]
            Now consider 
            \begin{align*}
                p_{m+1}q_m - q_{m+1}p_m &= (a_{m+1}p_m + p_{m-1})q_m - (a_{m+1}q_m + q_{m-1})p_m \\
                &= a_m + p_mq_m + p_{m+1}q_m - a_{m+1}q_mp_m - q_{m-1}p_m \\
                &= -(p_mq_{m-1} - q_mp_{m-1}) \\
                &= -(-1)^{m-1} = (-1)^m
            \end{align*}
        \end{proof}
    \end{theorem}
    
    \underline{Note}: This says that 
    \[
        p_kx+q_ky = \pm 1
    \]
    has an integer solution. 

    So Bezout $\rightarrow \gcd(p_k,q_k = 1)$

    \begin{corollary}
        $C_k = \frac{p_k}{q_k}$ is in lowest terms.
    \end{corollary}

    \begin{corollary}
        $C_{k+1} - C_k = \frac{(-1)^k}{q_kq_{k+1}}$
        \begin{proof}
            \begin{align*}
                C_{k+1} - C_k &= \frac{p_{k+1}}{q_{k+1}} \\
                &= \frac{p_{k+1}q_k - q_{k+1}p_k}{q_{k+1}q_k} \\
                &= \frac{(-1)^k}{q_{k+1}q_k}
            \end{align*}
        \end{proof}
    \end{corollary}
    
    Note: The relation $q_k = a_kq_{k-1} + q_{k-2}$
    implies that $0 < q_o \le q_1 < q_2 < q_3 <\dots$

    \begin{corollary}
        All infinite (simple) continued fractions converge.
    \end{corollary}

    \begin{theorem}
        \dots
        \begin{itemize}
            \item $C_0 < C_2 < C_4 < \dots$
            \item $C_1 > C_3 > C_5 > \dots$
        \end{itemize}
        \begin{proof}
            \begin{align*}
                C_{k+2} - C_k &= (C_{k+2} - C_{k+1}) + (C_{k+1} - C_k) \\
                &= \frac{(-1)^{k+1}}{q_{k+2}q_{k+1}} + \frac{(-1)^k}{q_{k+1}q_k} \\
                &= \frac{(-1)^k(q_{k_2} - q_k)}{q_{k+2}q_{k+1}q_k}
            \end{align*} 
        \end{proof}
    \end{theorem}

    \begin{theorem} [Dirichlet's Approximation]
        Let $x$ be irrational. 
        Then there exist infinitely many $\frac{a}{b}\in\QQ \quad(\gcd(a,b)=1)$
        such that 
        \[
            |x-\frac{a}{b}| < \frac{1}{b^2}
        \]
        \begin{proof}
            Let $x=[a_0; a_1, \dots]$

            We want to bound $|x-C_k|$.
            \begin{align*}
                |x-C_k| &\le |C_{k+1} - C_k| \\
                &= |\frac{(-1)^k}{q_{k+1}q_k}| \\
                &= \frac{1}{q_{k+1}q_k} \\
                &< \frac{1}{q_k^2}
            \end{align*}
            bc $q_{k+1} > q_k$.
        \end{proof}
    \end{theorem}

    \underline{Remark}: (Thue-Siegel-Roth Theorem)

    If $\alpha > 2$ then there exist at most finitely many $\frac{a}{b}\in\CC\quad(\gcd(\frac{a}{b}) = 1)$
    such that 
    \[
        |x-\frac{a}{b}| < \frac{1}{b^\alpha}
    \]

    \begin{theorem}
        $C_k = \frac{p_k}{q_k}$ approximates $x$ "the best" in the sense that 
        if $1\le b\le q_k$, then 
        \[
            |x-\frac{p_k}{q_k}| \le |x - \frac{a}{b}|
        \]
        for any $a\in\ZZ$. 
    \end{theorem}

    \begin{lemma}
        If $\frac{a}{b} \in\QQ$ with $1\le b\le q_k$, then 
        \[
            |q_kx-p_k| \le |bx-a|
        \]
        \begin{proof}
            Consider the system of equations
            \begin{align*}
                p_k\alpha + p_{k+1}\beta = a \\
                q_k\alpha + q_{k+1}\beta = b
            \end{align*}
            has a solution iff
            \[
                det\begin{bmatrix}
                    p_k & p_{k+1} \\
                    q_k & q_{k+1}
                \end{bmatrix} \ne 0
            \]
            has an integer solution iff 
            \[
                det\begin{bmatrix}
                    p_k & p_{k+1} \\
                    q_k & q_{k+1}
                \end{bmatrix} = \pm 1
            \]
        \end{proof}
    \end{lemma}

    Hence 7 integer solutions $\alpha, \beta$

    Details: 
    \begin{itemize}
        \item $\alpha \ne 0$
        \item $\beta = 0$ then Thm is true.
    \end{itemize}
    Now assume both $\alpha,\beta \ne 0$.
    We want to show that $\alpha$ and $\beta$ have opposite signs.

    Why? 
    
    If $\beta < 0$, then $q_k\alpha = b - q_{k+1}\beta$

    If $\beta > 0$, then same equations shows $\alpha < 0$.

    Thus, 
    \begin{align*}
        |bx-a| &= |(q_k\alpha + q_{k+1}\beta)x - (p_k\alpha + p_{k+1}\beta) \\
        &= |\alpha(q_kx - p_k) + \beta(q_{k+1}x-p_{k+1})| 
    \end{align*}
    If $q_kx-p_k > 0$, then $x-\frac{p_k}{q_k} > 0 \rightarrow x-\frac{p_{k+1}}{q_{k+1}} < 0$, 
    then $\alpha(q_kx - p_k)$ and $\beta(q_{k+1}x-p_{k+1})$ have the same sign, so
    \begin{align*}
        &= |\alpha(q_kx - p_k)| + |\beta(q_{k+1}x-p_{k+1})| \\
        &\ge |\alpha||q_kx-p_k| \\
        &\ge |q_kx-p_k|
    \end{align*}

    \begin{proof} [Proof Thm]
        If $1\le b \le q_k$, then $\frac{a}{b}$ satisfies $|x-\frac{p_k}{q_k}| < |x-\frac{a}{b}|$

        Suppose $|x-\frac{p_k}{q_k}| > |x-\frac{a}{b}|$. Then 
        \[
            |q_kx-p_k > q_k| |x-\frac{a}{b}|
        \]
        But by the technical result, 
        \[
            |bx-a| > q_k|x-\frac{a}{b}| \ge b|x-\frac{a}{b}|
        \]
    \end{proof}