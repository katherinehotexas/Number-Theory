\chapter{Lecture 20}
\date{November 7, 2024}

\section{Last Time}
    Which primes can be written as the sum of 2 squares?
    Ans: $p=2, p\equiv 1\pmod{4}$ \\
    If $p$ is odd prime and $p=a^2 + b^2$ for some $a,b\in\ZZ$, 
    then $a^2\equiv -b^2\pmod{p}$
    \begin{align*}
        \legendre{a^2}{p} &= \legendre{b^2}{p} = \legendre{-1}{p}\legendre{b^2}{p} \\
        1 &= \legendre{-1}{p}\longrightarrow p\equiv 1\pmod{4}
    \end{align*}

\section{Sum of 2 Squares}
    Now suppose $p\equiv 1\pmod{4}$ want to write $p$ as a sum of 2 squares.
    Use 
    \[
        (u^2+v^2)(A^2+B^2) = (uA+vB)^2 + (vA+uB)^2
    \]

    \subsection{Fermat's Method of Infinite Descent}
    Since $p\equiv 1\pmod{4}$, we have $\legendre{-1}{p} = 1$ \\
    ie. $x^2\equiv -1\pmod{p}$ has a solution.
    ie. $x^2+1=kp$ for some $k\in\ZZ$ \\
    .$\quad x^2+1^2=kp$ is a sum of squares

    Suppose now that $A^2+B^2=Mp$. We will conduct a smaller multiple of $p$ 
    that is a sum of squares.

    Find integers $u,v$ susch that
    \begin{align*}
        u\equiv A\pmod{M} \\
        v\equiv B\pmod{M}
    \end{align*}
    so that 
    \[
        -\frac{1}{2}M \le u,v \le \frac{1}{2}M
    \]
    Thus $A^2+B^2\equiv u^2+v^2\equiv 0\pmod{M}$ \\
    Thus 
    \begin{align*}
        A^2+B^2 = Mp \\
        u^2 + v^2 = Mp
    \end{align*}
    Then 
    \begin{align*}
        (A^2+B^2)(u^2+v^2) = M^2rp \\
        (uA+vB)^2+(rA-uB)^2 = M^2rp \\
        uA+vB\equiv AA + BB \equiv A^2rB^2\equiv 0\pmod{M} \\
        vA-uB\equiv BA - AB\equiv 0\pmod{M} \\
        (\frac{uA+vB}{M})^2 + (\frac{vA-uB}{M})^2 = rp 
    \end{align*}

    \subsection{Example}
    Choose $p=13$. 
    \[
        \legendre{-1}{13} = 1 \rightarrow x^2+1=k\cdot 13 \rightarrow x=5,k=2
    \]
    \begin{align*}
        5^2+1^2 = 2\cdot 13 \\
        5\equiv 1\pmod{2} \\
        1\equiv 1\pmod{2} \\
        1^2+1^2 = 2\cdot 2
    \end{align*}
    \begin{align*}
        (5^2+1^2)(1^2+1^2) = 2^2 \cdot 1\cdot 13 \\
        (5+1)^2 + (5-1)^2 = 2^2\cdot 13 \\
        \frac{5+1}{2}^2 + \frac{5-1}{2}^2 = 13 \\
        3^2 + 2^2 = 13
    \end{align*}
        
\section{Gaussian Integers}
    \[
        \ZZ[i] = \{ a+bi\mid a,b\in\ZZ \}
    \]
    Primes sometimes factor in $\ZZ[i]$. \\
    eg. $5=(1+2i)(1-2i)$ but 3 is "prime" in $\ZZ[i]$ 

    Suppose $p\equiv 1\pmod{4}$. Then $p$ can be written as $p=a^2+b^2$
    for some $a,b\in\ZZ$ But then 
    \[
        p=a^2+b^2 = (a+bi)(a-bi)
    \]
    \underline{Claim}: Neither $a+bi$ nor $a-bi$ is a unit in $\ZZ[i] (1,-1,i,-i)$.
    Hence $p$ is composite in $\ZZ[i]$. 

    \subsection{When is $a+bi\in\ZZ[i]$?}
    Prime is a Gaussian integer?
    
    Ex: $\alpha = 1+2i$ is prime. \\
    Suppose $\alpha = 1+2i = (a+bi)(c+di)$ \\
    Could write out (ac-bd)+(bc+ad)i \\

    Another way? Use $N(a+bi) = a^2+b^2$. 
    Then 
    \begin{align*}
        N(1+2i) = N(c+bi)N(c+di) \\
        N = (a^2+b^2)(c^2+d^2)
    \end{align*}
    WLOG
    \begin{align*}
        a^2+b^2 = 1 \rightarrow (a,b) = 
        \begin{cases}
            (1,0),(ai) \\
            (-1,0),(a-i)
        \end{cases}
        \Longleftrightarrow a+bi = 
        \begin{cases}
            1, -1, \\
            i, -i
        \end{cases}
    \end{align*}

    \begin{corollary}
        If $N(a+bi) = a^2+b^2$ is prime, then $a+bi$ is prime in $\ZZ[i]$
    \end{corollary}

    \begin{theorem} [Gaussian Primes]
        Let $\alpha = a+bi$.
        \begin{enumerate}
            \item If $\alpha\in\ZZ (b=0)$, then $\alpha$ is prime in $\ZZ[i]$ 
            iff $\alpha = p$ is an odd prime with $p\equiv 3\pmod{4}$. 
            \item If $\alpha\in i\ZZ$ then $\alpha$ is $\dots \alpha = ip \dots p\equiv 3\pmod{4}$
            \item If both $a$ and $b$ are nonzero, then $\alpha$ is prime in $\ZZ[i]$ iff $N(\alpha)$ is a prime in $\ZZ$. 
        \end{enumerate}
        Ex. of 3: Suppose $N(\alpha)$ is even so $2\mid N(2)$. 
        Claim: $(1+i)\mid\alpha$
        \begin{proof}
            WTS 
            \[
                \frac{a+bi}{1+i} \in\ZZ[i]
            \]
            \begin{align*}
                \frac{a+bi}{1+i}\frac{1-i}{1-i} = \frac{(a+b)+(b-a)i}{2}
            \end{align*}
            Since $a^2+b^2$ is even, $a,b$ are both even or both odd.
            So $a+b$ and $b-a$ are both even.

            So \[ \frac{a+bi}{a+i} = \frac{a+b}{2} + \frac{b-a}{2}i \in\ZZ[i] \]
            So, $(1+i\mid (a+bi))$.
        \end{proof}
    \end{theorem}