\chapter{Lecture 12}
\date{October 8, 2024}

\section{Miscellaneous}
    \subsection{Least Common Multiple}
    \begin{definition}
        Let $a,b$ be positive integers. 
        The least common multiple of $a$ and $b$ denoted by 
        $\text{lcm}(a,b)$ is the smallest positive integer divisible by 
        $a$ and $b$. \\
        Examples
        \begin{itemize}
            \item $\text{lcm}(2,3)=6$
            \item $\text{lcm}(4,6)=12$
            \item $\text{lcm}(1,n)=n$
            \item $\text{lcm}(n,n)=n$
        \end{itemize}
        \[ 4\cdot 6 = 24, \gcd(4,6) = 2, \text{lcm}(4,6) = 12 \]
        \[ 3\cdot 9 = 27, \gcd(3,9) = 3, \text{lcm}(3,9) = 9 \]
        \begin{theorem}
            For positive integers $a,b$ we have 
            \[ ab = \gcd(a,b) \cdot \text{lcm}(a,b) \]
        \end{theorem}
    \end{definition}

    \subsection{More about $\phi$ (and number-theoretic functions)}
    \begin{definition}
        A number theoretic function (or arithmetic function) is a function
        \[ f: \NN \leftrightarrow \NN \quad(\text{or } \ZZ \leftrightarrow \ZZ) \]
        that has "number theory properties" \\
        Ex: 
        \begin{itemize}
            \item $\phi$
            \item $\tau(n) = \#$ of divisors of $n$
            \begin{align*}
                10:&\quad 1,2,5,10 \\
                \tau(10) &= 4 \\
                12:&\quad 1,2,3,4,6,12 \\
                \tau(12) &= 6
            \end{align*}
            \item  $\sigma(n) =$ \text{sum of divisors of n}
            \begin{align*}
                \sigma(10) &= 1 + 2 + 5 + 10 = 18 \\
                \sigma(12) &= 1+2+3+4+6+12=28
            \end{align*}            
        \end{itemize}
        Facts: $\phi,\tau,\sigma$ are all multiplicative.
        \begin{align*}
            \phi(ab) &= \phi(a)\phi(b) \\
            \sigma(ab) &= \sigma(a)\sigma(b) \quad\text{if } \gcd(a,b)=1\\
            \tau(ab) &= \tau(a)\tau(b)
        \end{align*}
        Notice: $\sigma(n) = \sum_{d\mid n}^{}d, \quad\tau(n) = \sum_{d\mid n}^{}1$ \\
        ($d\mid n$ is sum over positive divisors of $n$)        
    \end{definition}
    \begin{example}
        Define $F(n) = \sum_{d\mid n}^{}\phi(d)$
        \begin{align*}
            F(12) &= \sum_{d\mid 12}^{}\phi(d) \\
            &= \phi(1)+\phi(2)+\phi(3)+\phi(4)+\phi(6)\phi(12) \\
            &= 1+1+2+2+2+4 \\
            F(12) &= 12
        \end{align*}
        \begin{align*}
            F(15) &= \phi(1)+\phi(3)+\phi(5)\phi(15) \\
            &= 1+2+4+8 \\
            F(15) &= 15
        \end{align*}
    \end{example}
    \begin{theorem}
        For all pos  integers $n$,
        \[ n = \sum_{d\mid n}^{}\phi(d) \]
        \begin{proof}
            (Step 1) Lemma: If $f: \NN\leftrightarrow\NN$ is multiplicative, 
            then the function 
            \[ F(n) = \sum_{d\mid n}^{}f(d) \]
            is multiplicative. (Proof: HW) \\\\
            (Step 2) We know that $F(n) = \sum_{d\mid n}^{}\phi(d)$
            is multiplicative, since $\phi$ is multiplicative. \\
            Lets show $F(n)=n$ for primes and prime powers. \\
            If $p$ is prime, then $F(p) = \sum_{d\mid p}^{}\phi(d)
            = \phi(1) + \phi(p) = 1 + (p-1) = p$ \\
            Now calculate for $k\ge 1$
            \begin{align*}
                F(p^k) &= \sum_{d|p^k}^{}\phi(d) \\
                &= \phi(1) + \phi(p) + \phi(p^2) + \dots + \phi(p^k) \\
                &= 1 + (p-1) + (p^2-p) + \dots + (p^j-p^{j-1})+(p^k-p^{k-1}) \\
                F(p^k) &= p^k
            \end{align*}
            Now let $n=p_1^{k_1}\dots p_r^{k_r}$
            \begin{align*}
                F(n) &= F(p_1^{k_1})\dots F(p_r^{k_r}) \\
                &= p_1^{k_1}\dots p_r^{k_r} \\
                &= n
            \end{align*}
        \end{proof}
    \end{theorem}

    \subsection{Lagrange's Theorem}
    Recall $x^2\equiv 1\pmod{8}$ has $x\equiv 1,3,5,7$ (4 solutions). But\dots
    \begin{theorem} [Lagrange's Theorem]
        Let $f(x)$ be a polynomial of degree $d$ with integer coefficient
        and $p$ be prime. Suppose $p\nmid $(leading coefficient). \\
        Then $f(x)\equiv 0\pmod{p}$ has at most $d$ incongruent solutions.
        \begin{proof} 
            By induction on the degree d. \\
            Base case: $d=1$, $f(x) = a_1x+a_o$ and $p\nmid a_1$. 
            Then 
            \begin{align*}
                f(x) &\equiv 0\pmod{p} \\
                a_1x+a_0 &\equiv 0\pmod{p} \\
                a_1x &\equiv a_0\pmod{p}
            \end{align*}
            has a unique solution since $\gcd(a_1,p) = 1\le d$. \\\\
            Induction step: Let's assume the statement is true for all 
            polynomials of degree $\le k$. \\
            Now let 
            $f(x)\equiv a_{k+1}x^{k+1}+\dots+a_1x+a_0$ where $p\nmid a_{k+1}$.
            If $f(x)\equiv 0\pmod{p}$ has no solutions, then we are done since
            $0<k+1$. Hence suppose $x=a$ is a solution. \\
            By the division algorithm applied to $f(x)$ and $x-a$, we have
            \begin{align*}
                f(x) &= (x-a)\cdot q(x) + r, \quad r\in\ZZ  \\
                f(a) &\equiv 0\pmod{p} \\
                r &\equiv 0\pmod{p}
            \end{align*}
            Thus, $f(x)\equiv (x-a)\cdot q(x)\pmod{p}$. By IH, $q(x)\equiv 0\pmod{p}$
            has at most k solutions. Thus $f(x)\equiv 0\pmod{p}$ has at most $k+1$ solutions. \\
        \end{proof}
    \end{theorem}

\section{Order}
    \subsection{}
    \begin{definition}
        Let $\gcd(a,n)=1$. Then the smallest positive integer $k$ such that
        $a^k\equiv 1\pmod{n}$ is called the order of a modulo n and is denoted by 
        $\text{ord}_n(a)$ or just $\text{ord}(a)$ is it's unambiguous.
    \end{definition}
    \begin{example}
        $a^k\pmod{7}$
    \end{example}
    \begin{theorem}
        Suppose $\gcd(a,n)=1$ and $a^k\equiv 1\pmod{n}$. Then ord$(a)\mid k$.
        \begin{proof}
            By division algorithm, write
            \[ k = \text{ord}(a)\cdot q + r, \quad 0\le r<\text{ord}(a) \]
            Then 
            \begin{align*}
                a^k &\equiv 1\pmod{n} \\
                a^{\text{ord}(a)\cdot q}\cdot a^r &\equiv 1\pmod{n} \\
                a^{\text{ord}(a)^q}\cdot a^r &\equiv 1\pmod{n} \\
                a^r &\equiv 1\pmod{n}
            \end{align*}
            Then $r=0$, otherwise r is a smaller exponent for $a^r\equiv 1\pmod{n}$
            contradicting ord$(a)$ being the smallest. 
            Thus $k=$ ord$(a)\cdot q$ so ord$(a)\mid k$.
        \end{proof}
    \end{theorem}