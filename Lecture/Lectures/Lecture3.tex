\chapter{Lecture 3}
\date{September 3, 2024}

\section{Problem - Diophantine Equations}
    If a rooster is worth 5 coins, a hen 3 coins, and 3 chicks together 1 coin,
    how many roosters, hens, and chicks, totaling 100, can be bought for 100 coins?
    \begin{align*}
        x & = \# roosters \\
        y & = \# hens \\
        z & = \# chicks 
    \end{align*}
    \begin{align*}
        x+y+z & = 100 \\
        5x + 3y + \frac{1}{3}z & = 100
    \end{align*}
    \begin{center}
        \underline{Diophantine Equations}
    \end{center}
    \begin{align*}
        x^n+y^n & = z^n \\
        x^2+y^2+z^2+w^2 & = n
    \end{align*}

\section{Bezout's Theorem}
    Let $a,b\in\mathbb{Z}$ (not both zero). The gcd of $a$ and $b$ is the smallest
    positive integer $d$ that can be written as $ax+by=d, x,y\in\mathbb{Z}$.
    \begin{proof}
        Let $S = \{ax+by>0 | x,y\in\mathbb{Z}\}$. 
        Note that S is nonempty since for $x=a$, $y=b$ we have $ax+by= a^2+b^2>0$.
        By WOP, S has a smallest element, call it d. WTS:
        \begin{enumerate}
            \item $d|a$, $d|b$
            \item if $c|a$, $c|b$, then $c\leq d$
        \end{enumerate}
        To show $d|a$, apply the division algo to obtain $a=d\cdot q + r, 0\leq r<d$. \\
        Writing $d=ax_0 + by_0$ for $x_0,y_0\in\mathbb{Z}$, we have
        \begin{align*}
            r & = a-d\cdot y \\
            r & = a(ax_0 + by_0)\cdot q \\
            r & = a(1-x_0q) + b(-y_0q)
        \end{align*}
        Hence, if $r>0$ then $r\in S$ which is smaller than d, contradicting d being
        the smallest element. Then, $r=0$ and $d|a$. (Same argument for $d|b$). \\
        Now suppose that $c\in\mathbb{Z}$ such that $c|a$ and $c|b$. Recall that if x and y
        are integers, then $c|(cx+by)$. Hence, $c|(ax_0+by_0) <=> c|d$. 
        Then $c\leq |d| = d$.
        Therefore, $d=\gcd(a,b)$.
    \end{proof}
    \begin{corollary}
        Every common divisor of $a$ and $b$ divides $\gcd(a,b)$.
    \end{corollary}
    \begin{corollary}
        The linear Diophantine equation $ax+by=c$ has a solution iff $d|c$.
    \end{corollary}
    \begin{proof}
        First assume that $ax+by=c$ has a solution: $c=ax_0+by_0$. 
        Since $d|a$, and $d|b$, we have $d|(ax_0+by_0)$. \\
        One the other hand, suppose $d|c$. By definition, $c=d|k$ for some k. \\
        By Bezout's theorem, we can write 
        \[
            d=ax+by \text{ for some } x,y\in\mathbb{Z}
        \]
        Then,
        \begin{align*}
            d\cdot k & = a(x\cdot k) + b(y \cdot k) \\
            c & = a(x\cdot k)+b(y\cdot k)
        \end{align*}
        So c is an integer linear combo $a < b$ as desired.
    \end{proof}

    \begin{definition}
        We say that integers a and b (not both zero) are relatively prime or coprime if
        \[
            \gcd(a,b) = 1
        \]
    \end{definition}
    \begin{corollary}
        Integers a and b are relatively prime iff there exist $x,y\in\mathbb{Z}$ such that
        $ax+by=1$.
    \end{corollary}
    \begin{corollary}
        If a, b are coprime, then $ax+by=c$ has a solution for any $c\in\mathbb{Z}$.
    \end{corollary}

\section{Euclidean Algorithm}
    \begin{enumerate}
        \item Start with (a,b) (assume $|a|\geq |b|$)
        \item Apply DA: $a = bq+r, 0 \leq r < |b|$
        \item If $r=0$, then $b|a$ and $\gcd(a,b) = |b|$.
        \item Otherwise, replace $(a,b)$ with $(b,r)$.
        \item Repeat.
        \item The final nonzero r is gcd.
    \end{enumerate}
    \begin{example}
        $\gcd(12378,3054)$
        \begin{align*}
            12378 & = 3054\cdot 4 + 162 \\
            3054 & = 162\cdot 18 + 138 \\
            162 & = 138\cdot 1 + 24 \\
            138 & = 24\cdot 5 + 18 \\
            24 & = 18\cdot 1 + 6 \\
            18 & = 6\cdot 3 + 0 
        \end{align*}
        \begin{center}
            $\gcd = 6$
        \end{center}
        Note: if you allow for negative remainders, that can be more efficient.
        \begin{align*}
            3054 & = 162\cdot 19 - 24 \\
            162 & = (-24)(-7) - 6 \\
            -24 & = (-6)(4) + 0
        \end{align*}
    \end{example}

    \begin{example}
        Solve $1237x+3054y = 6$ via "Extended Euclidean Algorithm".
        \begin{align*}
            6 & = 24-18\cdot 1 \\
            & = 24-(138-24*5) \\
            & = 24\cdot 6 - 138 \\
            & = (162-138)\cdot 6 - 138 \\
            & = 162\cdot 6 - 138\cdot 7 \\
            & = 162\cdot 6 - (3054-162\cdot 18) \cdot 7 \\
            & = (12378-3054\cdot 4)\cdot 6 - (3054-(12378-3054))\cdot 7
        \end{align*}
    \end{example}
        
    \begin{example}
        Solve
        \begin{align*}
            x+y+z & = 100 \\
            5x+3y+\frac{1}{3}z & = 100 \\
        \end{align*}
        Using $z=100-x-y$, we have $7x+4y = 100$. \\
        Note: $7(-1) + 4(2) = 1$. \\
        So $7(-100) + 4(200) = 100$
        
        \begin{align*}
            7 & = 4\cdot 1 + 3 \\
            4 & = 3\cdot 1 + 1 \\
            1 & = 4-3 \\
            1 & = 4 - (7-4) \\
            1 & = -7 + 4(2)
        \end{align*}
    \end{example}

    \begin{theorem}
        If $ax+by=c$ has a solution $x_0,y_0\in\mathbb{Z}$. Then any other solution
        $x,y\in\mathbb{Z}$ is given by 
        \[
            x=x_0 + \frac{b}{d}k, y = y_0-\frac{a}{d}k        
        \]
        where $k\in\mathbb{Z}$ and $d=\gcd(a,b)$. \\
        If $x,y,z >0$, then k must satisfy
        \[
            \frac{200}{7} > k > 25
        \]
        So
        \[
            k = 26,27,28 \text{, so the only solutions are}
        \]
        \begin{align*}
            x & = 4, y = 18, z = 78 \\
            x & = 8, y = 11, z = 81 \\
            x & = 12, y = -1, z = 89
        \end{align*}
        
    \end{theorem}

