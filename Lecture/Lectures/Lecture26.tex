\chapter{Lecture 26}
\date{December 5, 2024}

\section{Square-triangular numbers}
    \[
        1,3,6,10,15,21,28,36,45,55,66,78,91,\dots
    \]
    Each number is $1 + 2 + 3 + \dots$
    \[
        T_n = 1+2+3+\dots+n = \frac{n(n+1)}{2}
    \]
    \[
        T_49 = 1225 = 35^2
    \]
    1225 is a square-triangular number 

    \underline{Q}: Are there infinitely many square-triangular numbers? YES! 

    A \textbf{square-triangular number} must satisfy
    \[
        n^2 = \frac{m(m+1)}{2} \quad\text{for some pos integers } m,n
    \]
    \begin{align*}
        2n^2 &= m^2 + m \\
        8n^2 &= (2m+1)^2 - 1 \\
        (2m+1)^2 - 8n^2 &= 1
    \end{align*}
    Let $x = 2m+1$, $y=2n$. 
    \[
        \rightarrow x^2-2y^2 = 1
    \]
    (Pell's Equation)

    Find the fundamental solution by computing the components of $\sqrt{2}$.
    \[
        \sqrt{2} = [1; \overline{2}]
    \]
    Convergents: $\frac{1}{1}, \frac{3}{2}, \frac{7}{5}, \dots$ 
    \begin{align*}
        (x=1, y=1) &\quad 1^2 - 2\cdot 1^2 = -1 \\
        (x=3, y=2) &\quad 3^2 - 2\cdot 2^2 = 1 
    \end{align*}
    
    So $x = 3, y = 2$ is the fundamental solution. 

    $\leftrightarrow m = 1, n = 1$ corresponds to the square-triangular number $T_1 = 2^2 = 1$

    Recall: Every other positive solution is of the form $x_n, y_n$, 
    \[
        x_n = y_n\sqrt{2} = (3+2\sqrt{2})^n
    \]

    Ex: 
    \begin{align*}
        (3+2\sqrt{2})^2 &= 9+8+12\sqrt{2} \\
        &= 17 + 12\sqrt{2} \\
        \rightarrow m = \frac{17-1}{8}, n = \frac{12}{2} = 6 \\
        \leftrightarrow T_8 = 6^2 = 36
    \end{align*}

    Ex: 
    \begin{align*}
        (3+2\sqrt{2})^3 &= (3 + 2\sqrt{2})(17+12\sqrt{2}) \\
        &= 99+70\sqrt{2} \\
        \rightarrow m = 49, n = 35 \\
        T_{49} = 35^2 = 1225
    \end{align*}

\section{Square-pyramid numbers}
    "Stacking cannonballs"
    \[
        9+4+1
    \]
    (Sum of consecutive squares)

    When is 
    \[
        1^2 + 2^2 + \dots + m^2 = n^2?
    \]
    \[
        1^2 + 2^2 + \dots + 24^2 \quad\text{is a square}
    \]

\section{Wiener's attack on RSA}
    \subsection{RSA Recap}
    Pick two primes $p,q$, compute $n=pq$. 
    \[
        \phi(n) = (p-1)(q-1) = n-p-q+1
    \]
    Choose $E,D$ such that 
    \[
        ED = 1\pmod{\phi(n)}
    \]
    $\rightarrow$ Public key (n, E) $\rightarrow$ Encode $Z\rightarrow Z^E\pmod{n}$ \\
    $\rightarrow$ Private key (n, D) $\rightarrow$ Decode $W\rightarrow W^D\pmod{n}$

    \subsection{Attack Theorem}
    \begin{theorem}
        If $D < 3n^{\frac{1}{4}}$, then it is possible to efficiently recover D (and factor n)
        only given the public key (n, E).
    \end{theorem}
    Ex: If n has 1024 binary digits, then you must have D at least 
    around 256 binary digits long

    \begin{proof}
        Since $DE = 1\pmod{\phi(n)}$, we have 
        \[
            DE = 1 + k\phi(n)\quad\text{for some } k
        \]
        \begin{align*}
            |DE - k\phi(n)| = 1 \\
            |\frac{E}{\phi(n) - \frac{k}{D}} = \frac{1}{D\phi(n)} \\
            \rightarrow \frac{k}{D} \text{ approximates } \frac{E}{\phi(n)} 
        \end{align*}
    \end{proof}
    Idea: $\phi(n)$ is "close" to n. 

    Now $\phi(n) = n-p-q+1$. 
    \[
        |n-\phi(n)| = p+q-1 < 3\sqrt{n}
    \]
    Bound 
    \begin{align*}
        |\frac{E}{n} -\frac{k}{D}| &= |\frac{ED-kn}{nD}| \\
        &= |\frac{ED = k\phi(n) - kn + k\phi(n)}{nD}| \\
        &= |\frac{1-k(n-\phi(n))}{nD}| \\
        &= \frac{1}{nD} + \frac{k(n-phi(n))}{nD} \le \frac{k(n-\phi(n))}{nD}
    \end{align*}

    \begin{align*}
        |\frac{E}{n} -\frac{k}{D}| &< \frac{k(n-\phi(n))}{nD} \\
        &< \frac3{k}{\sqrt{n}D} < \frac{3D}{\sqrt{n}D}
    \end{align*}

    Claim: $k < D$
    \begin{proof} [Proof of Claim]
        \[
            k\phi(n) = DE-1 < DE
        \]
        On the other hand, $E<\phi(n)$.
    \end{proof}

    Assume $D < \frac{1}{3}n^{\frac{1}{4}}$. Now use
    \begin{align*}
        D < \frac{1}{3}n^{\frac{1}{4}} \\
        D^2 < \frac{1}{9}\sqrt{n} \\
        \frac{1}{D^2} > \frac{9}{\sqrt{n}} 
    \end{align*}
    Hence 
    \[
        |\frac{E}{n} - \frac{k}{D}| < \frac{3}{\sqrt{n}} < \frac{3}{9D^2} = \frac{1}{3D^2} < \frac{1}{2D^2}
    \]

    By thm from last time, $\frac{k}{D}$ must be a convergent of $\frac{E}{n}$. 
    In fact, $\gcd(k, D) = 1$ and convergents are always in lowest terms. 

    Ex: $n = 101. 107 = 10807$. 

    $\phi(n) = 10600, D = 3, E = 7067$. 

    Public key: $(10807, 7067)$

    $\frac{E}{n} = \frac{7067}{10807} = [0;1,1,1,8,17,1,2,2]$

    Convergents: $\frac{1}{1}, \frac{1}{2}, \frac{2}{3}, \frac{17}{26}$

    $3\cdot 7067 = 2; 10600 + 1$

    \[
        (x-p)(x-q) = x^2 - (p+q)x + n
    \]

\section{Final}
    Exam is cumulative 

    Half is from this unit (material from homeworks; same topics)

    1 sheet of paper cheat sheet 

    Need a calculator