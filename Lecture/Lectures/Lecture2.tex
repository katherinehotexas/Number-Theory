\chapter{Lecture 2}
\date{August 29,2024}

\section{Proof by Contradiction}
    To prove a statement p, assume p is false and derive a contradiction.
    \begin{theorem}
        $\sqrt{2}$ is irrational.
        \begin{proof}
            Assume $\sqrt{2}$ is rational.
            So there exist integers a,b s.t. 
            \[
                \sqrt{2} = \frac{a}{b}, \text{ where a and b have no common factors.}
            \]
            Thus $2b^2=a^2$. ie. $2|a^2$.
            Hence also $2|a$. By definition, we can write $a=2k$ for some $k\in\mathbb{Z}$.
            Then,
            \begin{align*}
                2b^2 &= (2k)^2 = 4k^2 \\
                b^2 &= 2k^2
            \end{align*}
            So $2|b^2$, hence $2|b$.
            Thus, 2 is a common factor of a and b, a contradiction. \\
            Therefore, $\sqrt{2}$ is irrational.
        \end{proof}
    \end{theorem}

\section{Proof by Induction}
    Use to prove an infinite number of statements. 
    Ex: Prove that the sum of the first n odd integers is $n^2$. \\
    Strategy: 
    \begin{itemize}
        \item Prove base case(s) n=0,1
        \item Prove that if the statement is true for n, then it is true for n+1
    \end{itemize}

    \begin{proof} [Proof by Induction]
        Base case: For n=1, the sum of the first n positive odd integers is 1, which is $n^2$. \\
        Induction step: Assume that the sum of the first n odd integers is $n^2$.
        Consider the sum of the first n+1 odd integers.
        \[
            \sum_{k=1}^{n+1}2k-1 = 1+3+5+\dots+2n-1+2(n+1)-1
        \]
        By the induction hypothesis, we have
        \begin{align*}
            \sum_{k=1}^{n+1}2k-1 &= n^2+2(n+1)-1 \\
            &= n^2 + 2n + 2 - 1 \\
            &= n^2 + 2n + 1 \\
            &= (n+1)^2, \text{ as desired}
        \end{align*}
    \end{proof}

    \begin{theorem}
        For $n\geq 1, \frac{d}{dx}x^n = nx^{n-1}$.
        \begin{proof} [Proof by Induction]
            Base case: n=1. $\frac{d}{dx}x^1=1=1\cdot x^0$. \\
            Induction step: Assume $\frac{d}{dx}x^n = nx^{n-1}$ is true for some $n>1$. 
            Using the power rule, we have
            \begin{align*}
                \frac{d}{dx}x^{n+1} &= x(nx^{n-1})+x^n \\
                &= n\cdot x^{1+(n-1)}+x^n \\
                &= x^n(n+1) \\
                &= (n+1)x^n, \text{ as desired.}
            \end{align*}
        \end{proof}
    \end{theorem}

\section{Well Ordering Principle (WOP)}
    Every nonempty subset of $\mathbb{N}$ has a smallest element.
    \begin{theorem} [Division Algorithm]
        For any $a,b\in\mathbb{Z}$ with $b\neq 0$, there exist unique integers q,s s.t.
        $a=bq+r, 0\leq r<|b|$.
        \begin{proof}
            Consider the set 
            \[ S=\{a-bx|x\in\mathbb{Z}, a-bx\geq 0\} \]
            For simplicity, assume $b>0$. Note that S is nonempty since for $x=-|a|$, we have
            \begin{align*}
                a-bx = a-b-(-|a|) &= a+b|a| \\
                & \geq a+|a| \\
                & \geq 0
            \end{align*}
            So, $a-bx\in S$. \\\\
            By WOP, S has a smallest element r. Call the corresponding value of x by q. \\
            So $r=a-bq \Leftrightarrow a=bq+r$. \\\\
            Now, we want to show that $0\leq r\leq |b|\ (=b)$ since $b>0$. \\
            By way of contradiction, assume $r\geq b$. 
            Consider
            \begin{align*}
                a-b(q+1) &= a-bq-b \\
                &= r-b \\
                & \geq 0
            \end{align*}
            Thus, $a-b(q+1)$ is an element of S that is smaller than r, a contradiction. \\\\
            Suppose there exist $q_1,r_1,q_2,r_2\in\mathbb{Z}$ such that 
            \[ a=bq_1+r_1 = bq_2+r_2 \]
            where $0\leq r_1,r_2 < b$ (still assuming $b > 0$).
            We want to show $q_1=q_2, r_1=r_2$.
            We have 
            \begin{align*}
                bq_1-bq_2 &= r_1-r_2 \\
                b(q_1-q_2) &= r_1-r_2 \\
                b|q_1-q_2| &= |r_1-r_2| < b
            \end{align*}
            But $b|q_1-q_2| < b$ implies (since $b > 0$) that 
            \[ 0\leq |q_1-q_2|<1 \]
            So, $q_1-q_2$ since $q_1,q_2\in\mathbb{Z}$
            Thus also $r_1=r_2$.
        \end{proof}
    Note: The division algorithm lets us make statements like "Every integer can be 
    expressed uniquely in the form $4k, 4k+1, 4k+2, or 4k+3$"
    \end{theorem}

    \begin{theorem}
        The square of every odd integer is of the form $8k+1$.
        \begin{proof}
            By the division algorithm, any odd integer n is of the form $n=4k+1$ or 
            $4k+3$. \\
            In the 1st case,
            \begin{align*}
                n^2 &= (4k+1)^2 \\
                &= 16k^2+8k+1 \\
                &= 8(2k^2+3k+1)
            \end{align*}
            In the 2nd case,
            \begin{align*}
                n^2 &=(4k+3)^2 \\
                &= 16k^2+24k+9 \\
                &= 8(2k^2+3k+1)+1
            \end{align*}
        \end{proof}
    \end{theorem}

    \begin{definition}
        For $a,b,c\in\mathbb{Z}$, if $c|a$ and $c|b$, we say that c is a common divisor 
        and has the property that for any other common c of a and b that $d\geq c$, 
        we call d the greatest common divisor of a and b, and write $d=\gcd(a,b)$.
    \end{definition}
