\chapter{Lecture 22}
\date{November 14, 2024}

\section{Recall: Arithmetic Functions}
    \[
        f: \NN \rightarrow \RR (\CC)
    \]
    that have some "number theory" property. 

    Ex. $\phi = $ totient, $\sigma = $ divisor sum, $\tau = $ divisor count 

    \begin{definition}
        $f$ is multiplicative if $f(ab) = f(a)f(b)$ whenever $\gcd(a,b) = 1$.
    \end{definition}

    We can express 
    \[
        \tau(n) = \sum_{d\mid n}^{} 1
    \]
    \[
        \sigma(n) = \sum_{d\mid n}^{} d
    \]
    Ex. $n=12, \quad\sum_{d\mid 12}^{} f(d) = f(1)+f(2)+f(3)+f(4)+f(6)f(12)$

    Conversely, given arithmetic function $f$ we can define $F(n) = \sum_{d\mid n}^{} f(d)$

    \underline{Recall}
    \[
        \sum_{d\mid n}^{} \phi(d) = n
    \]

    \begin{theorem}
        Let $n = p_1^{e_1}\dots p_r^{e_r}$, then 
        \begin{enumerate}
            \item $\tau(n) = (e_1 + 1)(e_2 + 1)\dots(e_r + 1)$
            \item $\sigma(n) = \frac{p_1^{e_1+1} - 1}{p_1 - 1} + \dots + \frac{p_r^{e_r+1} - 1}{p_r - 1}$
            \item $\phi(n) = (p_1^{e_1} - p_1^{e_1-1})\dots(p_r^{e_r} - p_r^{e_r - 1})$
        \end{enumerate}
        \begin{align*}
            \sigma(p^e) &= (1+p+p^2+\dots+p^{e-1}+p^e) \\
            &= \frac{p^{e+1}-1}{p-1}
        \end{align*}
        If $d\mid n = p_1^{e_1} - p_r^{e_r}$, then $d=p_1^{k_1}\dots p_r^{k_r}$ where $0\le k_i \le e_i$
    \end{theorem}

    \begin{theorem}
        If $f$ is multiplicative, then 
        \[
            F(n) = \sum_{d\mid n}^{} f(d)
        \]
        is multiplicative. 
    \end{theorem}

    \begin{corollary}
        $\tau(n)$, $\sigma(n)$ are multiplicative.
    \end{corollary}

    \begin{theorem}
        If $F$ is multiplicative, then $f$ is multiplicative.
    \end{theorem}

\section{Mobius Function}
    Let $n$ be a positive integer. 

    \begin{align*}
        \mu(n) =
        \begin{cases}
            1 \quad\text{ if } n = 1 \\
            0 \quad\text{ if } p^2\mid n = \text{ for some prime } p \\
            (-1)^r \quad\text{ if } n = p_1\dots p_r \text{ distinct primes}
        \end{cases}
    \end{align*}
    Ex. $\mu(2) = -1, \quad\mu(p) = -1, \quad\mu(12) = 0, \quad\mu(6) = 1$

    \begin{theorem}
        $\mu$ is multiplicative. 
        \begin{proof}
            Let $\gcd(a,b) = 1$. If for some prime $p$ we have $p^2\mid a$ or $p^2\mid b$, 
            then $p^2\mid ab$, so $\mu(a)\mu(b) = 0 = \mu(ab)$. 

            Now suppose $a=p_1\dots p_r, \quad b=q_1\dots q_k$ are square-free. then
            \begin{align*}
                \mu(ab) &= \mu(p_1\dots p_r q_1\dots q_k)  \\
                &= (-1)^{r+k} \\
                &= (-1)^r(-1)^k  \\
                &= \mu(a)\mu(b)
            \end{align*}
            
            $n = \sum_{d\mid n}^{} f(d)$, what is $f(d)$? $f(d) = \phi(d)$. 
            \[
                F(n) = \sum_{d\mid n}^{} f(d) = 
                \begin{cases}
                    1 \quad n = 1 \\
                    0 \quad \text{otherwise}
                \end{cases}
            \]
            What is $F(n) = \sum_{d\mid n}^{} \mu(d)$?

            Ex. $F(10) = \sum_{d\mid 10}^{} \mu(d) = \mu(1) + \mu(2) + \mu(5) + \mu(10) = 1-1-1+1 = 0$.

            Ex. $F(12) = \sum_{d\mid 12}^{} \mu(d) = \mu(1) + \mu(2) + \mu(3) + \mu(4) + \mu(6) + \mu(12) = 1-1-1+0+1+0 = 0$.

        \end{proof}
    \end{theorem}

    \begin{theorem}
        Let $F(n) = \sum_{d\mid n}^{} \mu(d)$. 
        Then 
        \[
            F(n) = 
            \begin{cases}
                1 \quad\text{if } n = 1 \\
                0 \quad\text{if } n > 1
            \end{cases}    
        \]
        \begin{proof}
            We have that $F$ is multiplicative. Since $\mu$ is multiplicative, let us compute
            \[
                F(p^k) = \mu(1) + \mu(p) + \mu(p^2) + \dots + \mu(p^k) = 1-1+0+\dots+0 = 0
            \]
            If $n = p_1^{e_1}\dots p_r^{e_r}$, then $F(n) = F(p_1^{e_1})\dots F(p_r^{e_r}) = 0$
        \end{proof}
    \end{theorem}

    \subsection{Mobius Inversion Formula}
    \begin{theorem} [Mobius Inversion Formula]
        Let $F(n) = \sum_{d\mid n}^{} f(d)$. Then 
        \[
            f(n) = \sum_{d\mid n}^{} \mu\legendre{n}{d}F(d)
        \]
        \begin{proof} [(Idea)]
            Use $n = 10$.
            \begin{align*}
                \sum_{d\mid 10}^{} \mu(d)F(\frac{10}{d}) &= \sum_{d\mid 10}^{}(\mu(d)\sum_{c\mid \frac{10}{d}}^{}f(c)) \\
                &= \mu(1)(f(1)+f(2)+f(5)+f(10)) \\
                &\quad+ \mu(2)(f(1) + f(5)) \\
                &\quad+ \mu(5)(f(1) + f(2)) \\
                &\quad+ \mu(10)(f(1)) \\
                &= f(1)(\mu(1)+\mu(2)+\mu(5)+\mu(10)) \\
                &\quad+ f(2)(\mu(1)+\mu(5)) \\
                &\quad+ f(5)(\mu(1)+\mu(2)) \\
                &\quad+ f(10)(\mu(1)) \\
                &= \sum_{d\mid 10}^{} f(d)(\sum_{c\mid \frac{10}{d}}^{}\mu(c))
            \end{align*}

            \begin{align*}
                \sum_{d\mid n}^{} \mu(d)F(\frac{n}{d}) &= \sum_{d\mid n}^{} (\mu(d)\sum_{c\mid \frac{n}{d}}^{}f(c)) \\
                &= \sum_{d\mid n}^{} (f(d)\sum_{c\mid \frac{n}{d}}^{}\mu(c)) = 
                \begin{cases}
                    0 \quad\text{if } \frac{n}{d} > 1 \\
                    1 \quad\text{if } n = d
                \end{cases} \\
                &= \sum_{d=n}^{} f(d)\sum_{d\mid 1}^{}\mu(c) \\
                &= f(n)
            \end{align*}

            To be more precise:
            \begin{align*}
                \sum_{d\mid n}^{} \mu(d) \sum_{c\mid\frac{n}{d}}^{} f(c) \\
                &= \sum_{d\mid n}^{}\sum_{c\mid\frac{n}{d}}^{} \mu(d)f(c) \\
                &= \sum_{d\mid n, c\mid\frac{n}{d}}^{} \mu(d)f(c) \\
                &= \sum_{c\mid n, d\mid\frac{n}{c}}^{} \mu(d)f(c)
            \end{align*}
        \end{proof}
    \end{theorem}

    Ex: 
    \begin{align*}
        \tau(n) = \sum_{d\mid n}^{} 1 \\
        \rightarrow 1 = \sum_{d\mid n}^{} \mu(\frac{n}{d})\tau(d) \\
        \sigma(n) = \sum_{d\mid n}^{} d \\
        n = \sum_{d\mid n}^{}\mu(\frac{n}{d})\sigma(d) (=\sum_{d\mid n}^{}\phi(d))
    \end{align*}