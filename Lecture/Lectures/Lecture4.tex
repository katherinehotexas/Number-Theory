\chapter{Lecture 4}
\date{September 5, 2024}

\section{Bezout, Euclid's Lemma}
    \begin{enumerate}
        \item If $a|c$ and $b|c$, must $ab|c$? \\
        False: $a=b=c=2$, $2|2$, $2|2$ but $4\nmid 2$ 
        \item If $a|bc$ and $a\nmid b$, must $a|c$? \\
        False: $a=4, b=c=2$
    \end{enumerate}
    
    But\dots Proposition: Let $a,b,c\in\mathbb{Z}$
    \begin{enumerate}
        \item If $a|c, b|c$ and $\gcd(a,b)=1$, then $ab|c$.
        \begin{proof}
            By Bezout, there exist integers $x,y$ s.t. $ax+by=1$. 
            Then, $acx+bcy = c$. \\
            By definition, there exist $r,s\in\mathbb{Z}$ s.t. $c=ar=bs$. 
            Thus, 
            \begin{align*}
                a(bs)x + b(ar)y &= c \\
                ab(sx+ry) &= c
            \end{align*}
            So, $ab|c$.
        \end{proof}

        \item If $a|bc$, and $\gcd(a,b)=1$, then $a|c$. (Euclid's Lemma)
        \begin{proof}
            Again, there exist $x,y\in\mathbb{Z}$ s.t. $ax+by=1$.
            Then $acx+bcy=c$. \\
            Since $a|bc$, we have $bc=ar$ for some $r\in\mathbb{Z}$.
            Hence 
            \begin{align*}
                acx+ary &= c \\
                a(cx+ry) &= c
            \end{align*}
            So, $a|c$ as desired.
        \end{proof}
        
    \end{enumerate}

\section{Prime Numbers}
    \begin{definition}
        A prime $p$ is an integer greater than $1$ that is only divisible by $1$ and $p$.
    \end{definition}

    \begin{theorem} [Euclid's Lemma]
        If $p$ is prime and $p|ab$ $(a,b\in\mathbb{Z})$, then $p|a$ or $p|b$ (or both).
        \begin{proof}
            Suppose $p\nmid a$. Since $p$ is prime, this implies that $\gcd(p,a)=1$. \\
            Then by Euclid's Lemma, we have $p|b$.
        \end{proof}
    \end{theorem}

    \begin{corollary}
        If $p$ is prime and $p|(a_1a_2\dots a_n)$ then 
        $p|a_k$ for some $k, 1\leq k\leq n$.
        \begin{proof} [Proof by Induction] 
            Base case $(n=1)$. Tautology *(If A then A) \\
            \underline{Inductive step}: Assume that for some $n\geq 1$, if $p$ divides the product 
            of any collection of $n$ integers $a_1\dots a_n$, then $p|c_k$ for some $k$. \\
            Suppose $p|a_1a_2\dots a_{n}a_{n+1}$.
            By Euclid's Lemma, $p|a_1a_2\dots a_n$ OR $p|a_n+1$. \\
            In the latter case, we are done. \\
            Hence assume now that $p|a_1a_2\dots a_n$. By IH, $p|a_k$ for some 
            $k, 1\leq k\leq n$ as desired.
        \end{proof}
    \end{corollary}

    \begin{corollary}
        If $p,q_1,q_2,q_n$ are primes, and $p|q_1q_2\dots q_n$, then $p=q_k$ for some $k$.
        \begin{proof}
            By the previous result, $p|q_k$ for some k. Since $q_k$ is prime and $p>1$, 
            we have $p=q_k$.
        \end{proof}
    \end{corollary}

    \begin{theorem} [Fundamental Theorem of Arithmetic, FTA]
        Every integer $n>1$ can be expressed as a product of primes. Moreover, this expression
        is unique up to reordering the factors.
        \begin{proof} [Proof by Induction on n]
            Base case $(n=2)$. \\
            \underline{Induction step}: Assume that any integer ($>1$) less than or equal
            to n satisfies FTA. \\
            Now consider $n+1$. \\
            If $n+1$ is prime, we are done. Otherwise, assume $n+1=ab$ for some $1<a,b<n+1$. 
            By IH, a and b can be expressed as a product of primes, hence so can $n+1$. 
            This proves the existence statement. \\\\
            For uniqueness, take the same IH. 
            Suppose that we can express $n+1$ as
            \[
                n+1=p_1p_2\dots p_r = q_1q_2\dots q_s
            \]
            where $p_r,q_s$ are prime. \underline{Without loss of generality}, assume
            \[
                p_1\leq p_2\leq\dots\leq p_r \text{, and } q_1\leq q_2\leq\dots\leq q_s
            \]
            Note $p_1|q_1q_2\dots q_s$, so $p_1=q_i$ for some $i$.
            By the same argument, $q_1=p_j$ for some $j$. \\
            Since $p_1\leq p_j$ and $q_1\leq q_2$, this implies $p_1=q_1$. 
            By cancelling, we have $p_2\dots p_r = q_2\dots q_s$. \\
            Since $p_2\dots p_r = q_1\dots q_s \leq n$, we can apply IH to conclude that
            $r=s$ and $p_i = q_i$ for all i.
        \end{proof}
    \end{theorem}

    \begin{theorem}
        There exist infinitely many primes.
        \begin{proof} [Proof (Euclid)]
            Assume that $p_1\dots p_n$ is a list of n primes. \\
            Consider the integer $N=p_1\dots p_n+1$.
            Note that no $p_i$ can divide N, otherwise 
            \begin{align*}
                p_i &| (N-p_1\dots p_n) \\
                p_i &| 1 \\
                & nooooo
            \end{align*}
            But N is divisible by some prime p with $p\neq p_1,\dots,p_n$. 
            Thus, there are infinitely many primes.
        \end{proof}
    \end{theorem}

    % \begin{theorem} [Prime Number Theorem]
    %     Let pi x = # of primes $\leq x$.
    %     Then pi x grows asymptotically like $\frac{x}{\log(x)}$.
        
    % \end{theorem}

