\chapter{Lecture 14}
\date{October 15, 2024}

\section{Recap}
    If $\gcd(a,n) = 1$, the order of $a$ is the smallest positive exponent
    $k$ such that $a^k\equiv 1\pmod{n}$
    \begin{itemize}
        \item If $a^m\equiv 1\pmod{n}$, then $\ord{a}\mid m$
        \item $a,a^n,\dots,a^{\ord{n}}$ are all incongruent $\pmod{n}$
        \item If $\ord{a}=\phi(n)$, then $a$ is called a \underline{primitive root} and
        $a,\dots,a^{\phi(n)}\pmod{n}$ are congruent to all the integers between $1$ and $n$, 
        coprime to $n$
    \end{itemize}

\section{All primes have a primitive root}
    \begin{theorem}
        Let $p$ be prime and $d\mid p-1$. Then there are exactly $\phi(d)$
        integers (that are mutually incongruent $\pmod{p}$) that have
        order $d\pmod{p}$. In particular there are $\phi(p-1)$ primitive roots.
        \begin{lemma}
            If $d\mid p-1$, then $x^d\equiv 1\pmod{p}$ has exactly $d$ incongruent
            solutions $pmod{p}$.
            \begin{proof}
                $x^{p-1}-1\equiv x^{dk}-1 = (x^d - 1)(x^{d(k-1)}+\dots+x^d+x)$
            \end{proof}
        \end{lemma}
        \begin{proof} [Proof of Thm.]
            Define $\psi(d) = \#$ of integers $1\le x\le p-1$ having order $d\pmod{p}$. \\

            \underline{WTS}: $\psi(d) = \phi(d)$ for $d\mid p-1$ \\
            Instead, let's prove $\psi(d)\le \phi(d)$ when $d\mid p-1$. 
            If there are no integers with order $d$, then
            \[ \psi(d) = 0\le \phi(d) \]
            Hence assume there exists at least one integer $a$ with $\ord_{p}{a} = d$. \\

            \underline{Claim}: If $b$ has order $d$, then $b\equiv a^h\pmod{p}$ for some $h$.
            Why? If $b$ has order $d$, then $b$ satisfies:
            \[ x^d\equiv 1\pmod{p} \quad*\]
            which has exactly $d$ incongruent solutions. On the other hand,
            the integers $a,a^2,a^3,\dots,a^d$ are all incongruent $\pmod{p}$
            and they all satisfy $*$, since 
            \[ (a^{i})^d \equiv (a^d)^i \equiv 1^i \equiv 1 \pmod{p} \]
            Since $*$ has exactly $d$ solutions $\pmod{p}$, we must have 
            $b\equiv a^h\pmod{p}$ for some $h$, $1\le h\le d$. \\

            Now, we need to determine which $a^k$ has $\ord{a^k} = d$. 
            But $\ord{a^k} = \frac{d}{\gcd(h,d) = d}$
            precisely when $\gcd(h,d) = 1$. Hence there are exactly
            $\phi(d)$ exponents $h$ such that $a^h$ has order $d$. 
            Thus, we find $\psi(d)=\phi(d)$. We have shown for $d\mid p-1$,
            $\psi(d)$ is either $0$ or $\phi(d)$. But we know $\psi(d)\le \phi(d)$.

            Consider the sum
            \[ \sum_{d\mid p-1}^{}\psi(d). \]
            Note every integer $a$ between $1\le a\le p-1$ has some $\ord{a}$ 
            that divides $p-1$.
            Since each integer between $1$ and $p-1$ is counted exactly once,
            we have 
            \[ \sum_{d\mid p-1}^{} \psi(d) = p-1 \]
            \begin{mdframed}
            \begin{example}
                p = 7 
                \begin{align*}
                    \ord{1} &= 2 \\
                    \ord{2} &= 3 \\
                    \ord{3} &= 6 \\
                    \ord{4} &= 3 \\
                    \ord{5} &= 6 \\
                    \ord{6} &= 2
                \end{align*}
                \begin{align*}
                    \sum_{d\mid p-1}^{}\psi(d) &= \sum_{d\mid 6}^{} \psi(d) \\
                    &= \psi(1)+\psi(2)+\psi(3)+\psi(6) \\
                    &= 1+1+2+2 \\
                    &= 6 \\
                    &= p-1
                \end{align*}
            \end{example}
            \end{mdframed}

            Recall 
            \[ \sum_{d\mid p-1}^{}\phi(d) = p-1 \]
            Hence
            \[ \sum_{d\mid p-1}^{}\psi(d) = \sum_{d\mid p-1}^{}\phi(d), \quad\psi(d)\le \phi(d) \]  
            Thus $\psi(d) = \phi(d) \quad\forall\quad d\mid p-1$.
        \end{proof}
    \end{theorem}

    \underline{Note}: Once you have a primitive root $g$, then all the other primitive roots
    are congruent to $g^h$ where $\gcd(h,p-1) = 1$.

\section{How to find a primitive root}
    \begin{definition}
        Let $g$ be a primitive root of $p$ (or $n$ if $n$ has a primitive root).
        If $1\le a\le p-1$, the smallest positive exponent $k$ with $a\equiv g^k\pmod{p}$
        is called the \underline{index of $a\pmod{p}$} relative to $g$, 
        denoted $\ind(a)$.
    \end{definition}