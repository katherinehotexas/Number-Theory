\chapter{Lecture 23}
\date{November 19, 2024}

\section{Diophantine Equations}
    Ex:
    \begin{itemize}
        \item Linear $\quad ax+by = c$
        \item $x^2 + y^2 = p\quad$ (solvable when $p\equiv 1\pmod{4}$)
                "Easy" for any particular $p$ by brute force bc finitely many possibilities
        \item $x^2-y^2 = 1\quad$ infinitely many possibilities $(x+y)(x-y) = 1 \longrightarrow (x+y)\mid 2$ and $(x-y)\mid 2 = (1,0),(-1,0)$
        \item $x^2-2y^2 = 1\quad$ has soln $(x,y)=(3,2)$
    \end{itemize}

    In general, $x^2 - Dy^2 = 1$ is called \underline{Pell's Equation}. 
    How to find integer solutions?

\section{Diophantine Approximation}
    How to approximate irrational numbers by rational numbers in the "best" way

    Ex: $\pi\approx\frac{22}{7}$ is the best approximation among all rational 
    numbers with denominator $\le 7$ (much bigger actually)

\section{Continued Fractions}
    \begin{definition}
        A (simple) finite continued fraction is a rational number expressed as 
        \[
            a_0 + \frac{1}{a_1 + \frac{1}{a_2 + \frac{1}{\dots+\frac{1}{a_{n-1} + \frac{1}{a_n}}}}}
        \]
        where $a_i\in\ZZ, a_i > 0$ for $i\ge 1$.
    \end{definition}
    Ex. 
    \begin{align*}
        \frac{43}{19} &= 2 + \frac{5}{19} \\
        &= 2 + \frac{1}{\frac{19}{5}} \\ 
        &= 2 + \frac{1}{3 + \frac{4}{5}} \\
        &= 2 + \frac{1}{3 + \frac{1}{\frac{5}{4}}} \\
        &= 2 + \frac{1}{3 + \frac{1}{1 + \frac{1}{4}}}
    \end{align*}

    Notice: Euclidean Algorithm uses $a_i$ values from continued fraction
    \begin{align*}
        43 &= 19(2) + 5 \\
        19 &= 5(3) + 4 \\
        5  &= 4(1) + 1 \\
        4  &= 1(4)
    \end{align*}

    \underline{Notation}: 
    \[
        = [2; 3, 1, 4]
    \]

    \begin{theorem}
        Every rational number has a continued fraction representation. 
        \begin{proof}
            Eucliean Algorith applied to $\frac{a}{b}$ gives
            \begin{align*}
                a &= a_0 b + r_1 \\ 
                b &= a_1 r_1 + r_2 \\
                r_1 &= a_2 r_2 + r_3 \\
                &\dots \\
                r_{n-1} &= a_n r_n \\
                \frac{a}{b}   &= a_0 + \frac{r_1}{b} = a_0 + \frac{1}{\frac{b}{r_1}} \\
                \frac{b}{r_1} &= a_1 + \frac{r_2}{r_1} \\
                &\rightarrow \frac{a}{b} = a_0 + \frac{1}{a_1 + \frac{r_2}{r_1}} \dots
            \end{align*}
            By continuity, we obtain continued fraction.
        \end{proof}
    \end{theorem}

    \begin{definition}
        An \underline{infinite} continued fraction is an expression of the form 
        \[
            [a_o; a_1, a_2, \dots] = a_0 + \frac{1}{a_1 + \frac{1}{a_2 + \frac{1}{a_3 + \dots}}}
        \]
        $a_i\in\ZZ, a_i > 0$ for $i\ge 1$.
    \end{definition}

    \underline{Ex}: $\pi$
    \[
        \pi = 3 + \frac{1}{\frac{1}{0.14159}} = 3 + \frac{1}{7 + \frac{1}{15 + \frac{1}{1 + \frac{1}{\dots}}}}
    \]
    No obvious pattern\dots

    \underline{Ex}: $e$
    \[
        e = 2 + \frac{1}{1 + \frac{1}{2 + \frac{1}{1 + \frac{1}{1 + \frac{1}{4 + \frac{1}{1 + \frac{1}{1 + \frac{1}{6 + \frac{1}{1 + \frac{1}{\dots}}}}}}}}}} = [2;1,2,1,1,4,1,1,6,1,1,8,1,\dots]
    \]

    \underline{Ex}: $[1;1,1,1,1,\dots] = 1 + \frac{1}{1 + \frac{1}{1 + \frac{1}{1 + \dots}}}$

    Let $x = 1 + \frac{1}{1 + \frac{1}{1 + \frac{1}{1 + \dots}}}$
    \begin{align*}
        \rightarrow x = 1 + \frac{1}{x} \\
        x^2 - x - 1 = 0 \quad\rightarrow\quad x = \frac{1 + \sqrt{5}}{2} = \phi
    \end{align*}
    (Golden rule)

    \begin{theorem}
        \dots
        \begin{enumerate}
            \item A continued fraction is infinite iff it represents an irrational number
            \item The continued fraction represenation of an irrational number is unique
            \item A rational number has exactly two continued fraction representations: 
                \[
                    [a_o; a_1, \dots, a_n] = [a_0; a_1, \dots, a_{n-1}, 1] \quad\text{where}\quad a_n \ne 1
                \]
        \end{enumerate}
    \end{theorem}

    \begin{definition}
        The \underline{$k^{th}$ convergent} of $[a_0; a_1, a_2, \dots]$ is 
        \[
            C_k = [a_0; a_1, a_2, \dots, a_k]
        \]
    \end{definition}

    \underline{Ex:} For $\pi = [3;7,15,1,\dots]$
    \begin{align*}
        C_0 &= 3 \\
        C_1 &= 3 + \frac{1}{7} = \frac{22}{7} \\
        C_2 &= 3 + \frac{1}{7 + \frac{1}{15}} = \dots
    \end{align*}

    \underline{Ex:} $\frac{19}{51} = [0;2,1,2,6]$
    \[
        = 0 + \frac{1}{2 + \frac{1}{1 + \frac{1}{2 + \frac{1}{6}}}}
    \]
    \begin{align*}
        C_0 &= 0 \\
        C_1 &= 0 + \frac{1}{2} = \frac{1}{2} \\
        C_2 &= 0 + \frac{1}{2 + \frac{1}{1}} = \frac{1}{3} \\
        C_3 &= 0 + \frac{1}{2 + \frac{1}{1 + \frac{1}{2}}} = \frac{3}{8} \\
        C_4 &= \frac{19}{51}
    \end{align*}

    \[
        \begin{matrix}
            & & a_i \\
            C_0 & 0 & 0 \\ 
            C_1 & \frac{1}{2} & 2 \\
            C_2 & \frac{1}{3} & 1 \\
            C_3 & \frac{3}{8} & 2 \\
            C_4 & \frac{19}{51} & 6
        \end{matrix} \\
        51 = 8*6 + 3
    \]

    Define:
    \begin{align*}
        p_0 &= a_0, q_0 = 1 \\
        p_1 &= a_1 a_0 + 1, q_1 = a_1 \\
        p_k &= a_k p_{k_1} + p_{k-2}, q_k = a_k q_{k-1} + q_{k-2}
    \end{align*}

    \begin{theorem}
        $C_k = \frac{p_k}{q_k}$
        \begin{proof} [Proof by Induction]
            \underline{Base case} for $k = 0$: 
            \[
                C_0 = a_0 = \frac{a_0}{1} = \frac{p_0}{q_0}
            \]
            \underline{Base case} for $k = 1$:
            \[
                C_1 = a_0 + \frac{1}{a_1} = \frac{a_1 a_0 + 1}{a_1} = \frac{p_1}{q_1}
            \]

            \underline{Inductive step}: Assume $C_k = \frac{p_k}{q_k}$ for some $k\ge 2$. 
            WTS: $C_{k+1} = \frac{p_{k+1}}{q_{k+1}}$.
            \[
                C_{k+1} = [a_0;a_1,\dots,a_k,a_{k+1}] = [a_0;a_1,\dots,a_k,\frac{1}{a_{k+1}}]
            \]
            is a continued function of length $k$. 
            \begin{align*}
                C_{k+1} &= \frac{(a_k + \frac{1}{a_k + 1})p_{k+1} + p_{k+2}}{(a_k + \frac{1}{a_k + 1})q_{k+1} + q_{k+2}} \\
                &= \frac{(a_k + \frac{1}{a_k + 1})p_{k+1} + p_{k+2}}{(a_k + \frac{1}{a_k + 1})q_{k+1} + q_{k+2}} \\
                &= \frac{p_k + 1}{q_k + 1}
            \end{align*}
        \end{proof}
    \end{theorem}